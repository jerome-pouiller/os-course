%
% This document is available under the Creative Commons Attribution-ShareAlike
% License; additional terms may apply. See
%   * http://creativecommons.org/licenses/by-sa/3.0/
%   * http://creativecommons.org/licenses/by-sa/3.0/legalcode
%
% Copyright 2010 Jérôme Pouiller <jezz@sysmic.org>
%

% Pour faire une version imprimable, avec les notes sans overlay
% \PassOptionsToClass{notes=show,handout}{beamer}
% Pour en faire un article:
% \documentclass[10pt,ucs,usepdftitle=false,a4paper]{article}
% \usepackage{beamerarticle}

\documentclass[10pt,ucs,usepdftitle=false]{beamer}

% Pour mettre deux pages sur une
% (Préférer l'utilisation d'un post processing)
%\usepackage{pgfpages}
%\pgfpagesuselayout{2 on 1}[a4paper,border shrink=5mm]
%\pgfpagesuselayout{4 on 1}[a4paper,landscape, border shrink=5mm]
%\setbeameroption{show notes on second screen=right}

\input{style}

% Apparait sur chaque slide:
%\logo{\pgfimage[height=5mm]{pics/logo}}

\title{Systèmes d'exploitation}
\hypersetup{pdftitle={Systèmes d'exploitation}}
% \subtitle{Sous-titre}
\author[Sysmic - J. Pouiller]{Jérôme Pouiller \email{j.pouiller@sysmic.org}}
\hypersetup{pdfauthor={Sysmic - Jérôme Pouiller}}
\institute[Sysmic]{}
%\institute[Sysmic]{\hspace*{1cm}\pgfimage[height=1.5cm]{pics/logo}}
% Plus complet:
% \institute[Sysmic]{
%   \inst{1} \hspace*{1cm}\includegraphics[height=1.5cm]{pics/logo}
%   \and
%   \inst{2} \includegraphics[height=1.5cm]{pics/logo-quiris}
% }
%\date[Juin 2012]{Juin 2012}
\date{}
% Pour le PDF seulement:
\subject{Systèmes d'exploitation}
\keywords{}

\begin{document}

{
\setbeamertemplate{background canvas}{}
\begin{frame}[plain]
  \maketitle
  \begin{textblock}{10}(6,11)
    %\includegraphics[height=30mm,width=30mm]{sandwich}
    \begin{quote}
      \rmfamily\textit\textbf\color{darkgray}{\large
      ``I'm doing a (free) operating system (just a hobby, won't be big and
      professional like gnu) for 386(486) AT clones.''}
        \vskip3mm\hspace*\fill{\small--- Linus Torvalds (1991-08-25)}
    \end{quote}
  \end{textblock}
  \note[item]{Parler de moi, de mon CV, freelance, sysmic, expertise, Polytech Paris, Tours, Insa Rennes}
\end{frame}
}

  % -360 slides- -> 240 slides
  % 0-2 Intro, Shell (ca va déborder de 15min)
  % 2-4 Filesystems (ca va déborder de 10min)
  % 4-6 Création d'executable et les Makefile (tout juste 2h)
  % 6-8 Gestion Monotache (1h30)
  % 8-10 Gestion Multitache (2h tout juste. A étoffer)
  % 10-12 Gestion de la mémoire et debug de la mémoire(1h), Communication interprocess (A supprimer). Architecture et virtualisation (1h)

  %
% This document is available under the Creative Commons Attribution-ShareAlike
% License; additional terms may apply. See
%   * http://creativecommons.org/licenses/by-sa/3.0/
%   * http://creativecommons.org/licenses/by-sa/3.0/legalcode
%
% Created: 2012-07-28 10:50:36+02:00
% Main authors:
%     - Jérôme Pouiller <jezz@sysmic.org>
%

\part{Introduction}

\begin{frame}[fragile=singleslide]{Programme}
  \begin{itemize} 
  \item Utilisation des systèmes Unix
    \begin{itemize}
    \item Savoir utiliser l'environnement
    \item  Connaître  quelques  concepts  annexes  (file  descriptors,
      pattern matching, etc...)
    \item Rappeler les concepts de base d'un OS
    \end{itemize}
  \item Les systèmes de fichiers
    \begin{itemize}
      \item Organisation des fichiers
      \item Format des filesystems
    \end{itemize}
  \item La création d'exécutables
    \begin{itemize} 
    \item La compilation
    \item Le chargement
    \end{itemize}
  \item La gestion des tâches
    \begin{itemize}
    \item Les systèmes monotâche
    \item Les systèmes multitâches 
    \end{itemize}
  \item La gestion de la mémoire
    \begin{itemize}
    \item Les différents segments
    \item Les algorithmes de gestion de la mémoire
    \item Les techniques de debug
    \end{itemize}
  \item Les architectures d'OS
  \end{itemize}
\end{frame}

  %
% This document is available under the Creative Commons Attribution-ShareAlike
% License; additional terms may apply. See
%   * http://creativecommons.org/licenses/by-sa/3.0/
%   * http://creativecommons.org/licenses/by-sa/3.0/legalcode
%
% Created: 2012-07-28 10:50:36+02:00
% Main authors:
%     - Jérôme Pouiller <jezz@sysmic.org>
%

\section{Le shell}

\begin{frame}
  \partpage
\end{frame}

\begin{frame}
  \tableofcontents[currentpart]
\end{frame}

\begin{frame}[fragile=singleslide]{Historique}
  \begin{itemize}
  \item Mode de communication bas niveau privilègié
  \item  Lèger,  simple  à  implémenter, puissant.   Parfois  l'unique
    manière de communiquer avec le système.
  \item Le shell ``Unix'' est plus commun. Beaucoup d'autres interface
    en ligne de commande s'en inspire.
  \item Première version du shell Unix tel qu'on le connait écrite par
    Ken Thompson chez Bell Labs en 1971 (bien antérieur à Linux)
  \item Remplacé par le shell de Stephen Bourne en 1977
  \item Par ordre d'apparition: sh, csh, tcsh, ksh, bash, zsh, ash,
    dash
  \item Normalisé par la Posix 2 en 1992
  \item Fonctionne de manière plus ou moins identique sur tous les OS
    (Linux, Androïd, iOS, Windows/Cygwin, etc...)
  \item On peut faire beaucoup de chose avec la ligne de commande
  \item Il est possible de faire des script en shell. Il arrive ce
    soir le seul langage de script disponible sur le système.
  \end{itemize}
\end{frame}

\subsection{Bases}


\begin{frame}[fragile=singleslide]{Bases de shell}
  \begin{itemize}
  \item Lancer une commande (=lancer un programme)
    \begin{lstlisting}
$ ls
    \end{lstlisting} %$
  \item Séparation des arguments par des espaces
    \begin{lstlisting}
$ mkdir dir1 dir2
    \end{lstlisting} %$
  \item Conséquence: les espaces sont des caractère spéciaux en shell
  \item Les arguments sont recu par le programme par es deux arguments
    \c{argc} et \c{argv}
  \end{itemize}
\end{frame}

\begin{frame}[fragile=singleslide]{Convention}
  Par convention,  nous préfixons dans ces slides  les commandes shell
  par :
  \begin{itemize}
  \item  \verb+$+  pour les  commandes  à  éxecuter par  l'utilisateur
    normal
  \item \verb+%+ pour les commande à executer par root
  \item \verb+>+ pour les commandes non-shell
  \end{itemize}
\end{frame}

\begin{frame}[fragile=singleslide]{Les argument optionnels}
  \begin{itemize}
  \item Souvent les arguments optionnels commence par '-'
  \item Les options peuvent avoir des arguments
  \item On distingue les options longues commencant par '-\--'
    \begin{lstlisting}
$ ls --all
$ ls --sort=time
$ ls --sort time
    \end{lstlisting}
  \item ... les option courte tenant sur un seul caractère et
    commancant par un simple '-' (attention tout de même aux
    exceptions)
    \begin{lstlisting}
$ ls -l -a
$ ls
    \end{lstlisting}
  \item Il est possible de concaténer les option courtes
    \begin{lstlisting}
$ ls -la
    \end{lstlisting} %$
    \item Les options ne sont \emph{normalement} pas dépendante de leur emplacement
    \begin{lstlisting}
$ ls -la
    \end{lstlisting} %$
  \item Ce principe est normalisé car toutes ces commandes utilisent
    la fonction Posix getopt (mais rien ne le garanti)
  \end{itemize}
\end{frame}

\begin{frame}[fragile=singleslide]{La documentation}
  \begin{itemize}
  \item   \cmd{man COMMAND} permet d'accèder à la documentation de la commande
  \item Les pages de man sont divisée en 9 sections (d'après \emph{man(1)})
    \begin{enumerate}
    \item Executable programs or shell commands
    \item System calls (functions provided by the kernel)
    \item Library calls (functions within program libraries)
    \item Special files (usually found in /dev)
    \item File formats and conventions (e.g. /etc/passwd)
    \item Games
    \item Miscellaneous (including macro packages and conventions),
      e.g.  man(7), groff(7)
    \item System administration commands (usually only for root)
    \item Kernel routines [Non standard]
    \end{enumerate}
  \item Une même entrée peut être présente dans plusieurs section, il
    possible de préciser la section en la placant en argument avant la
    commande:
    \begin{lstlisting}
$ man read
$ man 2 read
    \end{lstlisting}
  \item Les références des pages de man sont donnés avec le numéro de
    section entre parenthèses.  Ainsi, \textit{wait(2)} signifie que
    vous pouvez accéder à la documentation avec la commande
    \verb+man 2 wait+.
  \item \cmd{man -l}  permet d'afficher un fichier ``local''
  \item \c{man man} pour plus d'information
  \end{itemize}
\end{frame}

\begin{frame}[fragile=singleslide]{Les chemins}
  Il est possible d'utiliser des chemins:
  \begin{itemize}
  \item absolus, commencant par un \c{/}
    \begin{lstlisting}
$ mkdir /tmp/tete
    \end{lstlisting}
  \item relatifs, commencant par un autre caractère
    \begin{lstlisting}
$ mkdir ../../tmp/titi
    \end{lstlisting}
  \end{itemize}
  Les chemins relatifs, s'interprête à partir du \emph{répertoire
    courant} (lié au processus actuel et hérité par les processus
  fils).
  \begin{itemize}
  \item \c{pwd} affiche le répertoire courant
  \item \c{cd} modifie le répertoire courant
  \item Dans un chemin, "." correspond au répertoire courant
  \item ... \cmd{mkdir foo} est identique à \cmd{mkdir ././foo}
  \item ".." correspond au répertoire parent
  \item Beaucoup de commande prennant en parametre un répertoire
    utilise le répertoire courant si le paramètre n'est pas spécifié
    (ex: \c{ls})
 \item  cf. \emph{path\_resolution(7)}
  \end{itemize}
Note: Les fichiers commencant par \c{.} sont considérés comme des fichiers cachés
\end{frame}

\begin{frame}[fragile=singleslide]{Le PATH}
  \begin{itemize}
  \item Normalisées par Posix, plus ou moins regroupée dans un projet
    nommé coreutils
  \item Une commande est recherchée dans la variable \c{$PATH}
  \item ... par défaut: /bin /sbin /usr/bin et /usr/sbin
  \item Si on spécifie le chemin (la commande contient \c{/}),
    \c{$PATH} n'est pas utilisé
  \item Par conséquent, pour lancer une binaire dans le répertoire
    courant:\c{./a.out}
  \item Ajouter \c{.} dans \c{$PATH} est une mauvaise pratique
  \item Mecanisme géré par la fonction Posix execvpe(3)
  \item Certaine commande ont un comportement différent suivant
    \c{argv[0]} (exemple: \c{test} et \c{[})
  \end{itemize}
\end{frame}

\begin{frame}[fragile=singleslide]{Le contenu des fichiers}
  \begin{itemize}
    \item Ne pas oublier qu'un fichier n'est qu'un vecteur d'octet
    \item Les fichiers dit "texte" ont simplement la particularité de
      n'avoir que des octets supérieurs à 0x20 (et les octet > 0x7F
      s'interprete suivant la région)
    \item Le principe de "type" de fichier est finalement assez flou.
    \item \c{file} permet de repérer le format des fichier
    \item L'utilisation d'une norme de nommage ou d'une extention peut
      aussi aider, mais ca n'est pas une pratique native sur la
      plupart des OS.
    \end{itemize}
\end{frame}

\begin{frame}[fragile=singleslide]{Les file descriptor}
  \begin{itemize}
  \item Lorsqu'un programme souhaite accèder à un fichier, il va
    utiliser la fonction Posix \c{open(3)} qui lui retourne un nombre
    appellé \emph{file descriptor}.
  \item Il s'agit d'un identifiant pour une structure dans l'OS.
  \item Le file descriptor peut être passé à d'autre fonctions du
    système comme \c{read(3)} ou \c{write(3)}.
  \item Nous verrons plus tard que le concept de file descriptor va
    plus loin.
  \item Les couche basses de l'OS ouvre automaitquement trois file
    descriptor lors qu'un programme est lancé:
    \begin{itemize}
    \item Entrée standard (numéro 0), accessible en
      lecture. Normalement au clavier.
    \item Sortie standard (numéro 1), accessible en
      écriture. Normalement relié à l'écran
    \item Sortie d'erreur (numéro 2), accéssible en
      écriture. Normalement relié à l'écran
    \end{itemize}
  \end{itemize}
\end{frame}

\begin{frame}[fragile=singleslide]{Les redirections}
  Il est possible de demander au shell de rediriger les entrée est les
  sortie d'une commande avec les metacaractère \c{<} \c{>} et \c{|}:
  \begin{itemize}
  \item Commande standard:
    \begin{lstlisting}
$ echo foo
    \end{lstlisting}
  \item Sortie standard vers un fichier
    \begin{lstlisting}
$ echo foo > file
    \end{lstlisting}
  \item Un fichier vers l'entrée standard
    \begin{lstlisting}
$ cat -n < file
    \end{lstlisting} %$
  \item Sortie d'erreur vers un fichier
    \begin{lstlisting}
$ ls toto 2> file
    \end{lstlisting} %$
  \end{itemize}
\end{frame}

\begin{frame}[fragile=singleslide]{Les Redirections}
  \begin{itemize}
  \item Sortie standard d'une commande vers l'entrée d'une autre
    \begin{lstlisting}
$ echo bar foo | wc
    \end{lstlisting}
  \item Couplage des redirections
    \begin{lstlisting}
$ cat -n < file1 | wc > file3
    \end{lstlisting} %$
  \item L'espace n'est pas obligatoire et les redirections ne sont pas
    forcement à la fin de la ligne
    \begin{lstlisting}
$ >file2 cat<file1 -n
    \end{lstlisting} %$
  \item Certaine commande detecte que la sortie est redirigée et se
    comporte différement
    \begin{lstlisting}
$ ls
$ ls | cat -n
$ ls > file
    \end{lstlisting}
  \end{itemize}
\end{frame}

\subsection{Les variables}

\begin{frame}[fragile=singleslide]{Les variables locales}
  \begin{itemize}
  \item Affectation:
    \begin{lstlisting}
$ FOO=foo
    \end{lstlisting}
  \item En shell, l'espace est un metacaractère, donc, ces
commandes son fausses:
\begin{lstlisting}
$ FOO = foo
$ FOO=foo bar
\end{lstlisting}
\item Il est possible de les concaténer avec \c{+=}
    \begin{lstlisting}
$ FOO+=bar
    \end{lstlisting}
  \item La syntaxe \c{$\{VAR\}} permet de récupérer le contenu d'une
    variable. Elle peut-être abbregée \c{$VAR} elle suivit d'un
    caractère non-alpha-numérique
     \begin{lstlisting}
$ echo ${FOO}
$ echo $FOO
$ echo ${FOO}_bar $FOO_bar
     \end{lstlisting}
     \item Sous zsh, \c{vared} permet d'éditer intéractivement une variable
   \end{itemize}
 \end{frame}

\begin{frame}[fragile=singleslide]{Les variables d'environnement}
  \begin{itemize}
  \item Tous les processus possède un ensemble de variable d'envionement.
  \item Elle se trouvent dans l'espace mémoire du processus (cf. \c{environ(7)})
  \item On y accéder facilement avec \c{getenv(3)} et \c{setenv(3)}
  \item Par défaut, un processus hérite de l'environnement de son
    parent (cf. \c{exev(3)})
  \item Un programme peut modifier son comportement en fonction du
    contenu de l'environnement
  \item \c{export} liste les variable d'environnement
  \item \c{export VAR} transforme une variable locale en variable
    d'environement, ou instancie la variable.
  \item Il est possible de lancer une commande avec une varibale
    d'envionnement particulière:
    \begin{lstlisting}
$ LANG=fr_FR.utf8 ls non-existant
ls: impossible d'accéder à non-existant: Aucun fichier ou dossier de ce type
    \end{lstlisting}
  \item ... ou en utilisant la commande \c{env}, qui possède plus d'options
    \begin{lstlisting}
$ env LANG=fr_FR.utf8 ls non-existant
ls: impossible d'accéder à non-existant: Aucun fichier ou dossier de ce type
    \end{lstlisting}
  \item Les variables d'environnement auront un impact sur tous les
    sous-processus lancés
  \item Leur fonctionnement est très différent des variable shell,
    mais elle sont gérée avec la même syntaxe
  \end{itemize}
  Les variables d'environnement importantes:
  \begin{itemize}
  \item \c{PATH}: les chemins ou les commandes doivent être recherchée
  \item \c{LANG}, \c{LOCALE} et \c{LC_*}: les informtion
    d'internationnalisation
  \item \c{DISPLAY}: l'addresse du serveur d'affichage pour les
    commande graphique
  \item \c{TERM}: le type de terminal utilisé (nécessaire pour le bon
    affichage des couleur et des outils fenetrés)
  \item \c{LS_COLOR}: contient la configuration de coloration de la
    command \c{ls --color}
  \item \c{PAGER}, \c{EDITOR}, \c{BROWSER}: Les outils à utilisé pour
    visualiser, éditer et aller sur le web.
  \end{itemize}
\end{frame}

\begin{frame}[fragile=singleslide]{La syntaxe évoluée des variable}
  La syntax des  variables peut être plus évoluée:
  \begin{itemize}
  \item \c{$\{VAR#foo\}} ou \c{$\{VAR/foo/bar\}} pour effectuer des
    modifcation sur les variables
  \item \c{VAR=( a b c )} pour affecter un tableau
  \item \c{$\{VAR[3]\}} pour lire une value dans un tableau
  \item Dans un script ou dans une fonction shell, les variable
    \c{$1}, \c{$2}, ... correspondent aux arguments. \c{$*} et \c{$@}
    signifient ``Tous les arguments''
  \item \c{$((21 * 2))} et \c{$[43 - 1]} sont remplacé par le résultat
    de l'expression aritthmétique
  \item \c{$(echo toto)} ou \c{`echo toto`} sont remplacés par le
    résultat de la commande \c{echo toto}.
  \end{itemize}
\end{frame}

\begin{frame}[fragile=singleslide]{L'escaping}
  \begin{itemize}
  \item Sans surprise, il est possible d'échapper un caractère spécial avec \c{\\}
    \begin{lstlisting}
$ perl -e print\ \"Hello\ World\\n\"\;
    \end{lstlisting}
  \item Il est aussi possible de \emph{quoter} un argument
    \begin{itemize}
    \item Le \emph{double quote} \c{"} echappe la plupart des
      caractères sauf les variable et le \c{"} de fin:
      \begin{lstlisting}
$ perl -e "print \"Hello World\n\";"
      \end{lstlisting}
    \item Le \emph{simple quote} \c{'} escape tous les caractère
      sauf le carcatère \c{'} de fin. Il ne peux pas être echappé:
      \begin{lstlisting}
$ perl -e 'print "Hello World\n";'
      \end{lstlisting}
    \item Le \emph{backquote} n'est pas un quoting, il correspond à \c{$()}
    \end{itemize}
  \end{itemize}
\end{frame}

\begin{frame}[fragile=singleslide]{Le globbing}
  \begin{itemize}
  \item Langage composé de trois metacaractères:
    \begin{itemize}
    \item \c{\?}:n'importe quel caractère
    \item \c{*}: n'importe quel caractère répété 0 ou plusieurs fois
    \item \c{[]}: N'importe lequel des caractère compris entre les crochets
    \end{itemize}
  \item Le shell essaie de faire correspondre tous les arguments
    contenant ces metacaractère avec les fichier du répertoire courant
  \item ... il remplace ensuite le pattern par la liste des fichier correspondants
    \begin{lstlisting}
$ wc -l *.c
    \end{lstlisting}
  \item Les commandes recoivent la liste des fichiers en argument, pas le pattern
  \item ... deux exception notables: \c{find -name} et \c{dpkg -l}. Il
    est necessaire de correctement les quoter pour que le pattern soit
    effectivement transmis à la commande
  \end{itemize}
\end{frame}

\begin{frame}[fragile=singleslide]{Les expressions régulières}
  \begin{itemize}
  \item Ressemble de loin au globbing
  \item Plus de metacaractères:
    \begin{itemize}
    \item \c{.}: N'importe quel caractère
    \item \c{[]}: N'importe lequel des caractères contenu entre les crochets
    \item \c{*} Le caractère précédants répété 0 ou plusieurs fois
    \item \c{+} Le caractère précédant répété 1 ou plusieurs fois
    \item \c{\{X,Y\}} Le caractère précédant répété entre X et Y fois.
    \item \c{()} Mémorise un groupe qui peut être référencé avec \c{\\X}
    \item \c{^} \c{$} Début et fin de ligne
    \item ... voir  regex(7) pour la spécifiction complète
    \item D'un point de vu formel, les expression réguli`ere peuvent
      décrire des grammaire régulière (de Type 3 dans la hiérarchie de
      chompsky). Il est possible de démontrer qu'il est existe ne
      bijection entre l'ensemble des expressions régulière et
      l'ensemble des automates à état fini (A comparer avec bison, un
      outils capable d'exprimer des grammaire de Type 2,
      \emph{context-free}).
    \item Les deux implémentation les plus répandues sont celles de
      gnu (Gnu Regular Expression Processor) et de perl. Il peut
      exister des différence entre ces implémentations.
      \begin{lstlisting}
$ echo a123b45 | grep -E '^a.*b.*$'
a123b45
$ echo a1111b1 | grep -E '^t(.)*t\1$'
a1111b1
      \end{lstlisting}
    \item Beaucoup de commande prennent des expression régulière en
      entrée. Attention à ce qu'elle ne soit pas interprétée comme du
      globbing par le shell
    \item Les expression régulière, c'est bon, mangez-en.
    \end{itemize}
  \end{itemize}
\end{frame}

\begin{frame}[fragile=singleslide]{Le terminal}
  \begin{itemize}
    \item Lourd historique
    \item Il existe(ait) énormément de terminaux différant. Chaque
      terminal à ses spécificitée:
      \begin{itemize}
      \item caractère permettant le retour à la ligne,
      \item possibilité de déplacer le curseur,
      \item possibilité de souligner,
      \item de metter des caractères en gras,
      \item terminaux en couleur,
      \item séquence retournée par les touches spéciales (backspace,
        delete, flèche, touches de fonction, composition de
        touches,...)
      \end{itemize}
    \item Le minitel est(était) un terminal compatible VT102
    \item Les temrinaux virtuel choisissent quelle norme ils
      implémente (u peuvent en faire une nouvelle)
    \item La variable d'environnement \c{$TERM} indique aux commandes
      le type de terminal. De nos jours, la valeur XTERM est de loin
      la plus répandue.
    \item \c{<Ctrl+V>} permet d'afficher la séquence de touche recue
      par le shell sans l'interpréter
    \item Les couleurs se font avec des séquence d'échappement plus ou
      moins standardisée: \c{\x1B[32m} pour le rouge, etc...
    \item La bibliothèque readline, très largement utilisée gère tout
      cet aspect. Readline inclut une base de donnée des
      fonctionnalité de tous les terminaux existant.
  \end{itemize}
\end{frame}

\begin{frame}[fragile=singleslide]{Outils dérivés}
  Quelques outils pour travailler avec les expression régulières
  \begin{itemize}
  \item ed, grep, sed, awk, perl -pe: Outils de traitement de texte
    automatique
  \item ed, ex, vi, vim: Un éditeur de texte
  \end{itemize}
\end{frame}

\begin{frame}[fragile=singleslide]{Jobs control}
  \begin{itemize}
  \item Il est possible de travailler avec plusieurs commande en parallèle
  \item \c{<CTRL+Z>} suspend la commande courante. En fait, c'est le
    noyau qui transforme cette combinaison de touche en signal et qui
    suspend le programme si celui-ci lui permet.
  \item \c{fg} continue l'éxecution de la commande en \emph{foreground}
  \item \c{bg} continue l'éxecution de la commande en \emph{background}
  \item Il est possible de lancer une commande directement en
    backgound en la terminant pas \c{&}
  \item \c{jobs} liste les jobs en cours d'éxection
  \end{itemize}
  \begin{lstlisting}
$ cp -r bigdir newdir
<CTRL+Z>
$ sleep 100 &
$ jobs
[1]  - suspended  cp -r bigdir newdir
[2]  + running    sleep 100
  \end{lstlisting}
\end{frame}

\begin{frame}[fragile=singleslide]{Ecrire des scripts}
  \begin{itemize}
  \item Les scripts possèdent la meme syntaxe que la ligne de commande
  \item Commencent par le chemin de l'interpréteur prefixé de \c{#!}
    \begin{lstlisting}
#!/bin/sh
#!/usr/bin/perl
#!/bin/sed
#!/usr/bin/make -f
    \end{lstlisting}
  \item L'OS appelle l'interpreteur et passer le fichier de script et
    ses arguments en paramètre. (Essayez avec votre propre
    application)
  \item Il est possible de les lancer en les passant directement à
    l'interpreteur de commande
  \item Doivent avoir les droits en execution
    \begin{lstlisting}
$ bash script.sh
$ chmod +x script.sh
$ ./script.sh
    \end{lstlisting} %$
  \item Il est bien sur possible d'appeller d'autres scripts
  \item Il est possible de sourcer d'autres script (!= appeller)
    \begin{lstlisting}
source lib.sh
. lib.sh
    \end{lstlisting}
  \item Il est possible de déclarer des fonctions
    \begin{lstlisting}
function bar {
}
foo() {
}
    \end{lstlisting}
  \item Il est possible de séparer deux ommande par \emph{\;}
  \item La commande \emph{test(1)} ou \c{\[(1)} permet d'nterpreter des condition
  \item Il existe aussi des structure de controle:
    \begin{lstlisting}
if test -n $VAR; then
  echo '$VAR exist'
else
  echo '$VAR does not exist'
fi
while [ -z $VAR ]; do
  VAR+=$(cat file)
done
    \end{lstlisting}
  \end{itemize}
\end{frame}

\begin{frame}[fragile=singleslide]{Quelques derniers trucs}
  \begin{itemize}
  \item  En début d'argument \c{\~/} sera remplacé par Le chemin de votre \emph{home}
  \item La syntaxe \c{a\{1,2\}b} sera remplacé par \c{a1b a2b}. Ca
    n'est pas du globbing, car ca ne matche pas avec le répertoire
    courant
  \item Complétion
    \begin{lstlisting}
$ cd /h<TAB>/j<TAB>/c<TAB>
    \end{lstlisting}
  \item Les shell modernent sont capables de faire de la completion avancée
    \begin{lstlisting}
$ man l<TAB>
    \end{lstlisting}
  \item Alias
    \begin{lstlisting}
$ alias ll="ls -l --color=auto"
$ alias cp="cp -i"
$ alias mv="mv -i"
$ alias rm="rm --one-file-system"
    \end{lstlisting} %$
  \item Il est possible de mettre des commande dans \c{\~/.bashrc} ou
    \c{\~/.zshrc} qui seront éxecutée à chaque démarrage du shell.
  \item Man de la commande courante (sous Zsh uniquement)
    \begin{lstlisting}
$ rm -<M-h>
    \end{lstlisting} %$
  \end{itemize}
\end{frame}

\begin{frame}[fragile=singleslide]{Travailler réseau en deux mots}
  \begin{itemize}
  \item \c{/sbin/ifconfig -a} donne la configuration des cartes réseaux
  \item \c{lo} correspond à la carte réseau virtuelle de localhost
  \item \c{route -n} affiche la table de routage du système (kernel)
  \item \c{/etc/resolv.conf} et \c{/etc/nsswitch.conf} contiennent la
    configuration du service de résolution de nom (dont le DNS) (libc)
  \item ... de nos jours, on utilise souvent un proxy DNS et la
    configuration du DNS se retrouve dans
    \c{/var/run/nm-dns-dnsmasq.conf}
  \item \c{iptables} gère les règle de filtrage
  \item Bien que très utilisée, ces commandes peuvent être remplacée
    par iproute2 (et sa cmmande \c{ip})
  \item \c{netstat -n} ou \c{netstat -ln} permet d'obtenir l'état des
    connexion réseau
  \end{itemize}
\end{frame}

\subsection{Un shell à distance}

\begin{frame}[fragile=singleslide]{Travailler à distance}
  Protocoles les plus utilisés:
  \begin{itemize}
  \item Telnet
    \begin{itemize}
    \item \cmd{telnetd} et \cmd{telnet}
\begin{lstlisting}
host$ telnet -l root target
target%
\end{lstlisting} %$
    \item   Pas   sécurisé,   attention   a   votre   mot   de   passe
      \note[item]{faire    une   démonstration    avec    telnetd   et
        \texttt{tcpdump  -i lo -A  port telnet}  (il faut  regarder le
        dernier caractère de chaque paquet envoyé)}
    \item \verb/<CTRL+]>/ permet d'accéder à l'interface de commande
    \end{itemize}
  \item Ssh
    \begin{itemize}
    \item \cmd{sshd} et \cmd{ssh}
\begin{lstlisting}
host$ ssh root@target
target%
\end{lstlisting} %$
    \item Sécurisé
    \item Pleins de bonus de sécurisé
    \item   Il   est   possible   de  forcer   la   déconnexion   avec
      \verb/<RET><~><.>/   et   de   suspendre  une   connexion   avec
      \verb/<RET><~><CTRL+Z>/
    \end{itemize}
  \end{itemize}
\end{frame}

\subsection{Utilisation de clefs numériques}

\begin{frame}[fragile=singleslide]{Utiliser des clef ssh}
  \begin{itemize}
  \item Possibilité de créer des clefs pour \cmd{ssh}
\begin{lstlisting}[language=sh]
host$ ssh-keygen -t dsa
\end{lstlisting} %$
  \item Toujours mettre un mot de passe sur votre clef
  \item Recopiez votre clef dans \verb+~/.ssh/authorized_keys+
\begin{lstlisting}[language=sh]
host$ ssh-copy-id root@target
\end{lstlisting} %$
  \end{itemize}
\end{frame}

\begin{frame}[fragile=singleslide]{Utiliser des clef ssh}
  \begin{itemize}
  \item Utiliser ssh-agent
\begin{lstlisting}[language=sh]
host$ ssh-agent
host$ SSH_AUTH_SOCK=/tmp/agent.3391; export SSH_AUTH_SOCK;
host$ SSH_AGENT_PID=3392; export SSH_AGENT_PID;
host$ echo Agent pid 3392;
\end{lstlisting} %$
  \item Enregistrer votre passphrase auprès de l'agent
\begin{lstlisting}[language=sh]
host$ ssh-add
\end{lstlisting} %$
  \item Forwarder votre agent
\begin{lstlisting}[language=sh]
host$ ssh -A root@target
target%
\end{lstlisting} %$
  \end{itemize}
\end{frame}

\begin{frame}[fragile=singleslide]{Forwarder des ports}
  \begin{itemize}
  \item \c{ssh -L 1080:far-far-host:80 far-host}: Une demande sur le
    port 1080 de ma machine \c{locale} est forwarder à \c{far-host} qui se
    connecte sur \c{far-far-host} sur le port 80. Très utile si
    far-far-host n'est pas directment accessible par internet.
  \item \c{ssh -R 1080:close-host:80 far-host}: Une demande sur le
    port 1080 de \c{far-host} est forwarder à ma machine \c{locale} qui se
    connecte sur \c{close-host} sur le port 80. Très utile si je dois
    acceder à \c{close-host} à partir de far-host
  \item \c{ssh -D 1080 far-host} Idem que l'option \c{-L}. En
    revanche, je peux demander à crée un tunnel vers n'importe quel
    host dynamique. On utilise le protocole socks utilisé par les
    proxy. Certaine commande intègre le gestion de ce protocole. Pour
    les autre, on peut utiliser \c{socksify}
  \item L'étape suivante est de créer une interface réseau virtuelle
    passant par ce canal et de lui affecter des route. On obtenient
    ainsi un VPN.
    \item cf. \emph{ssh(1)}
  \end{itemize}
\end{frame}

\subsection{Les utilisateurs}

\begin{frame}[fragile=singleslide]{Les utilisateurs}
  \begin{itemize}
  \item Il y a plusieurs utilisateurs sur le système. Il sont gérés
    dans \c{/etc/passwd}
  \item Il y a des groupe sur le système, gérés dans \c{/etc/group}
  \item Une utilisateur peur appartenir à plusieurs groupes et
    plusieurs utilisateurs peuvent appartenir au meme groupe
  \item Ces utilsateur et ces groupe sont géré avec leur UID et
    leur GID, /etc/passwd et /etc/group s'occuppent de faire la
    correspondances avec les nom lors de l'affichage
  \item \emph{root} (UID 0) est l'utilisateur privilègié
  \item Certaine fonction ne peuvent être executé que par root (par
    exemple: modifier l'horloge, la configuration réseau)
  \item Il existe plusieurs manières pour devenir \c{root}
    \begin{itemize}
    \item \c{login}
    \item \c{su}
    \item \c{sudo}
    \end{itemize}
  \item Chaque fichier du système est associé à un propriétaire et à un groupe
  \item Chaque fichier possède 3 + 3 x 3 droits (modifiable par \emph{chmod(1)}):
    \begin{itemize}
    \item Ecriture (w)
    \item Lecture (r): Pour un répertoire, cela permet de lister le contenu
    \item Execution (x): Pour un répertoire, cela permet d'accéder au
      fichiers contenus dans le répertoire. Pour un fichier cela permet de
      l'éxecuter
    \end{itemize}
  \item Ces trois droits sont répétés pour
    \begin{itemize}
    \item Le propriétaire (u)
    \item Le groupe (g)
    \item Le reste du monde (o)
    \end{itemize}
    \begin{lstlisting}
$ ls -ld /bin/ls ~/.bashrc /tmp /bin/ping
-rwxr-xr-x  1 root root 105840 Apr  1 05:09 /bin/ls
-rwxrwxrwx  1 jezz jezz     16 Feb 21 23:03 /home/jezz3/.bashrc
-rwsr-xr-x  1 root root  35712 Nov  8  2011 /bin/ping
drwxrwxrwt 13 root root   4096 Jul 28 20:42 /tmp
   \end{lstlisting}
 \item Il est pratique d'écrire ses droit sous la forme octal: 755, 644, etc...
 \item Il existe des système de gestion de droits plus fin (Access Control List), mais pas aussi utilisé.
   \item Il existe en plus:
\begin{itemize}
  \item SetUid/SetGid (comme ping): La commande prend les droits du propriétaire/groupe lorsqu'elle s'éxecute
  \item Sticky Bit (comme /tmp): Il est possible de créer des fichier mais, seul le propriétaire du fichier peut l'effacer.
\end{itemize}
\end{itemize}
\end{frame}

\begin{frame}[fragile=singleslide]{Gestion des paquets}
  \begin{itemize}
  \item La principale raison obligeant à être root pour installer un
    paquet est pour écrire les répertoires appartenant à root.
  \item Un programme est livré sous forme d'un paquet contenant: la
    binire, la doc et les éventuelles ressources
  \item Il s'agit ni plus d'une tarball (tar est utilisé dans les .deb
    alors que cpio est utilisé dans les .rpm).
  \item Lors de la decompression (\c{dpkg -i}, \c{rpm -i}, \c{ipk
      -i}), on écrit dans une base de donnée les nom des fichiers
    décompressé afin de pouvoir facilement supprimer le paquet
  \item On ajoute un fichier normalisé à cette tarball afin d'avoir
    des information supplémentaires:
    \begin{itemize}
    \item Version
    \item Description
    \item Dépendances
    \item Signature
    \item ...
    \end{itemize}
  \item On peut interroger la base de donnée des fichier installés
    avec (\c{dpkg -l}, \c{dpkg -L}, \c{dpkg -S}, \c{rpm -q})
  \item Afin de ne pas devoir récupérer chaque paquets manuellement,
    on crée des dépots centralisé et indexé. Des outils peuvent
    interroger ces base (apt-get, urpmi, yum, ...). il peuvent
    automatiquement recupérer les dépendance et appeller dpkg/rpm
  \item Lors de l'installation, on note si le paquet a été demandé
    explicitement ou si il a été installé à cause d'une
    dépendance. Lorsqu'un paquet n'a plus d'utilité, on peut le
    désinstaler automatiquement.
  \item Pleins d'autres outils utiles: \c{apt-cache}, \c{apt-get source},
    \c{apt-get source -b}, \c{apt-get builddeps}, etc...
  \end{itemize}
\end{frame}

 % 4h
  %
% This document is available under the Creative Commons Attribution-ShareAlike
% License; additional terms may apply. See
%   * http://creativecommons.org/licenses/by-sa/3.0/
%   * http://creativecommons.org/licenses/by-sa/3.0/legalcode
%
% Created: 2012-07-28 20:09:12+02:00
% Main authors:
%     - Jérôme Pouiller <jezz@sysmic.org>
%

\section{Les systèmes de fichiers}

\subsection{L'arborescence}

\begin{frame}[fragile=singleslide]{Généralité}
  Les types de fichiers
  \begin{itemize}
  \item Normal (\emph{touch(1)}, \emph{open(2)}, \emph{mknod(2)})
  \item Répertoire (\emph{mkdir(1)}, \emph{mkdir(2)})
  \item Lien symbolique (\emph{ln(1)}, \emph{symblink(2)})
  \item Pipe nommé (\emph{mkfifo(1)}, \emph{mkfifo(3)}, \emph{mknod(2)})
  \item Socket nommé (\emph{bind(2)}, \emph{mknod(2)})
  \item Fichier périphérique charactère (\emph{mknod(1)}, \emph{mknod(2)})
  \item Fichier périphérique bloc (\emph{mknod(1)}, \emph{mknod(2)})
  \end{itemize}

  Par ailleurs, il  est possible de faire pointer  une nouvelle entrée
  vers  une structure e  fichier exsitante.  Ce mécanisme  est appelle
  \emph{hard link} (\emph{ln(1)}, \emph{link(2)}). Certain système COW
  fonctionnes ainsi.
  \\
  Tous  les  caractères  sont  authorisés  sauf  \c{/}  (réservé  pour
  séparerr les  répertoires) et \c{\0}  (réservé pour indiquer  la fin
  des arguments)
\end{frame}

\begin{frame}[fragile=singleslide]{Arborescence}
  \begin{itemize}
  \item \c{/bin} \c{/sbin} \c{/usr/bin} \c{/usr/sbin}: Binaires
  \item \c{/} et \c{/usr} séparé pour des raison historiques
  \item \c{*/sbin}: Binaire normalement réservée à root
  \item \c{/lib*} \c{/usr/lib*}: Bibliothèque
  \item  \c{/etc}:  Fichiers  de  configuration système.  Beaucoup  se
    terminent en \c{*rc.conf}
  \item \c{/home}, \c{/root} Espace des utilisateurs
  \item \c{/var} Répertoire de travail des applciation systèmes
  \item \c{/var/spool}  Queue de  traitement de certains  démon (mail,
    imprimante, etc...)
  \item \c{/var/log} Logs système
  \item \c{/var/cache} Cache système de cartains outils (index de man,
    version binaire des index apt, debconf, etc...)
  \item \c{/var/run}  et \c{/run}  Fichier de communication  entre les
    services  (fichiers  de  lock,  PIDs, sockets  des  communication,
    identifiants de mémoire partagée, etc...)
  \item \c{/var/lib} Données de travail de certaine bibliothèque (apt,
    ...)
  \item  \c{/var/www}, \c{/var/mail},  ...  Dedié aux  partage web  et
    mails
  \item \c{/usr/share} Donnée  statiques de certains services (icones,
    fonts, internationalisation, configurations statiques, etc...)
  \item \c{/usr/share/man} Pages de man
  \item \c{/usr/share/doc} Autre documentation, Licences
  \item  \c{/usr/include} Headers des  bibliothèques C  (installés par
    les version \c{*-dev} des packets)
  \item  \c{/usr/local} \c{/opt}  Application installée  en  dehors du
    service de packets normal (compilés localement)
  \item \c{/tmp} et  \c{/var/tmp} Répertoire tempsoraire. \c{/tmp} est
    vidé à chaque redémarrage
  \end{itemize}
  Le répertoire \c{/dev}
  \begin{itemize}
  \item Fichiers spéciaux, \emph{file devices}
  \item Communiquent avec des drivers (sous Unix, tout est fichier)
  \item \emph{dd(1)}  permet un control  plus fin et est  souvent plus
    approprié  que  \emph{cat}  et  \emph{echo}  pour  accèder  à  ces
    fichiers
  \item Certains  périphérique nécessite l'utilisation  d'autre appels
    système qui nous verrons plus tard
  \item Dans  tous les cas,  ces appels systèmes passent  utilisent un
    file descriptor comme identifiant
  \item Quelques exemples:
    \begin{itemize}
    \item \c{/dev/ttyS0}: Premier port série
    \item \c{/dev/sda}: Premier disque
    \item \c{/dev/sr0}: Lecteur CD
    \item \c{/dev/mem}: Mémoire physique
    \item \c{/dev/zero}: Périphérique virtuel qui ne donne que des 0
    \item \c{/dev/random}: Source d'entropie
    \item \c{/dev/null}: Trou noir
    \item \c{/dev/psaux}  et \c{/dev/input/*}: Periphériques d'entrées
      (souris, clavier, touchsreen, etc...)
    \item \c{/dev/snd/}: Cartes son
    \item \c{/dev/rtc0}: Horloge (plus spécial à accèder)
    \item  \c{/dev/video0}, \c{/dev/nvidia}:  Webcam,  carte vidéo  ne
      s'accèdent pas directement (nous y reviendrons)
    \end{itemize}
    N'apparaissent pas:
    \begin{itemize}
    \item Les bus (cas très rare et anormaux ou on fait des implementation
      en userland),
    \item Les cartes  et les périphériques réseaux (à  l'heure du cloud et
      des environnement distribué, ca peut avoir son importance)
    \item  Les carte  video n'apparaissent  pas toujours.  Certains driver
      sont implémentés en userspace
    \end{itemize}
  \end{itemize}
  Il s'agit  d'un standard, pour les systèmes  spécialisé, beaucoup de
  ces répertoire n'existeront pas.

  \emph{debootstrap(1)} permet de crée une nouvelle arboresence et d'y
  décompresser   les   fichiers   minimum  au   fonctionnement   d'une
  distribution Debian

  Lorsque  vous  installez  (ou   compilez)  un  paquet,  vous  pouvez
  spécifier de l'installer à partir d'une autre racine.

  Il  est possible  de s'éxecuter  sur une  autre racine  à  l'aide de
  \emph{chroot(1)}   ->   Peut-etre    à   mettre   dans   la   partie
  virtualisation?
\end{frame}

\subsection{Les filesystèmes}

\begin{frame}[fragile=singleslide]{Les filesystèmes}
  \begin{itemize}
  \item Une  partition de disque est \emph{montée}  sur un \emph{point
      de montage}
  \item Un point de montage est un répertoire (vide de préférence)
  \item \emph{mount(1)} \emph{mount(2)} \emph{mount(8)}
  \item Il existe diffŕent type de file systemes: vfat, ntfs, iso9660,
    ext2, ext3, ext4, xfs, btrfs, reiserfs, cramfs, squashfs, jffs2
  \item  Il est  possible de  monter  un fichier  normal plutot  qu'un
    fichier périphérique avec \c{-o loop} (une image disque ou une iso
    par exemple)
  \item \c{tmpfs} est mappé de sur de la mémoire RAM
  \item  \c{procfs}  (monté sur  \c{/proc})  et  \c{sysfs} (monté  sur
    \c{/sys}) sont des file systems virtuels
  \item Il permettent d'obtenir des informations sur l'état du noyau.
  \item Filesystem réseau: nfs, samba
  \item Filesystems plus complexes, implémneté en userland par FUSE (à
    l'aide de \c{/dev/fuse}): sshfs, ftpfs, etc..
  \end{itemize}
\end{frame}

\subsection{Implémentations}

\begin{frame}[fragile=singleslide]{table des patition}
  \begin{itemize}
  \item Inventée par Microsoft à peu près en même temps que la FAT
  \item  CHS (Cylindre-Head-Sector)  est  obsolete, de  nos jours,  on
    accede au disque en  tuilisant leur LBA (Logical Block Addressing)
    (les disque IDE doient implementer cette compatibilité, je ne dois
    pas que ca soit encore present dans le protocole sata )
  \item Les 512 premiers octets représente la table
  \item Les 446 premier  octet contiennent le \emph{Bootcode} (un boot
    loader s'appuyant sur le bios encore utilisé de nos jours)
    %% Ajouter figure
  \item  Ensuite un tableau  de 4  entré de  16 octets  reṕrésention 4
    partitions principales
  \item Chaque entré  contient le type, l'addresse (LBA)  de départ et
    la taille de la partition (répété dans le filesystème)
  \end{itemize}
\end{frame}

\begin{frame}[fragile=singleslide]{Vfat}
  \begin{itemize}
  \item Accronyme de File Allocation Table
  \item Inventé par Microsoft en 1977
  \item Disque divisé en cluster de  $512 * 2^n$ octets (en fait $2^n$
    secteurs de 512 octets)
  \item  La structure  volume ID  est  placée à  l'adresse zero,  elle
    contient la taille des cluster, l'addresse du répertoire racine et
    la taille de la FAT (nous  verrons que la taille de la partition =
    taille de fat en double mots * taille du cluster en octets)
  \item Les répertoires
    \begin{itemize}
    \item Un cluster répertoire  est un vecteur de structures pointant
      sur des fichiers ou d'autres répertoire.
    \item Les entrée contiennent un plus des information comme le nom,
      la taille et les attribut des fichiers
    \item Il  existe des valeurs  spéciales pour indiquer  les entrées
      supprimées et la fin du listing
    \item On peut ainsi parcourir les fichier du disque
    \end{itemize}
  \item Et si le fichier ou le répertoire ne tient pas sur un cluster?
    \begin{itemize}
    \item La FAT intervient
    \item La FAT  est un tableau de d'entiers de  32bits (ou 16bits ou
      12 bits)
    \item Chaque entrée représent un cluster (entrée 1 == cluster 1)
    \item  Si  l'entrée  vaut   0,  le  cluster  est  disponible  pour
      l'allocation
    \item Si  l'entrée vaut  un nombre, le  cluster est occuppé  et la
      valeur de l'entrée pointe sur le cluster suivant
    \item Si l'entrée vaut 0xFFFFFFFF,  le cluster est occupé et c'est
      le dernier cluster du fichier ou du répertoire
    \end{itemize}
  \item Les défauts:
    \begin{itemize}
    \item  La  fragmentation,  particulièrement  vraie si  un  fichier
      grossi progressivement
    \item Le temps d'accès entre la  FAT et les données et l'entrée du
      répertoire (pour modifier la taille)
    \item  La liste simplement  chainée dans  la FAT  qui oblige  à la
      parcourir pout accer à la fin du fichier
    \item Pas de pointeur vers le répertoire parent
    \end{itemize}
  \end{itemize}
  cf. \url{http://www.pjrc.com/tech/8051/ide/fat32.html}
\end{frame}

\begin{frame}[fragile=singleslide]{ext2}
  Inventé par Rémy Card au LIP6
  \begin{itemize}
  \item Deux type de structures principales:
    \begin{itemize}
    \item Des inodes (index-node) (128bits)
    \item Des blocks (paramètrable de l'ordre de 4Ko de nos jours)
    \end{itemize}
  \item Les disque  est divisé en \emph{block group}  (une centaine de
    Mo)
  \item Le système de block group permet de rapprocher les données des
    index
  \item  ... et  de  plus facilement  récupérer  le disque  en cas  de
    problème
  \item Le premier  block groupe se trouve à  l'adresse 1024 (l'espace
    avant est réservé pour un éventuel bootloader)
  \item Chaque \emph{block group} contient
    \begin{itemize}
    \item    Une    copie    du    superblock   (sauf    si    l'opton
      \emph{sparce\_block} est active)
    \item Des donnée concernant  ce block (taille de l'espace d'inode,
      taille  de l'espace  de block,  nombre de  block/inode utilisée,
      etc...)
    \item Un espace (fixe) d'inode
    \item Un espace (fixe) de block
    \item Un pointeur vers un block  de bitmap des inode qui permet de
      connaitre les inodes alloués
    \item Une pointeur vers un block de bitmap des block qui permet de
      connaitre les blocks alloués
    \end{itemize}
  \item  Remarque: Le bitmap  de block  doit tenir  sur un  block, par
    conséquent, il ne  peut y avoir plus de  taille\_de\_bloc * 8 blocs
    par groupe  de bloc. (ausi vrai  dans une moindre  mesure pour les
    inodes). C'est  généralement ca qui va déterminier  la taille d'un
    groupe.
  \item Remarque:  En connaissant, la  taille des block group,  le nom
    d'ino  par block group  et l'offset  de la  table d'inode  dans un
    group, il  est possible d'indexer n'importe quel  inode (idem pour
    les blocks)
  \item Chaque fichier est associé à un inode
  \item Un inode contient:
    \begin{itemize}
    \item Des informations comme  la taille, les date de modification,
      des création et d'accès, le type, etc...
    \item Dans  le cas fichier  spéciaux, toutes les  information sont
      contenus dans l'inode
    \item Un tableau de 12 pointeur vers les blocks contenant le corps
      du fichier (permet d'indexer les 50 premiers Ko)
    \item Un  pointeur vers un  block contenant des pointeurs  vers le
      coprs du fichiers (permet d'indexer les 4Mo suivants)
    \item un pointeur vers un block de second niveau (qui contient des
      pointeurs de pointeurs) (permet d'indexer les 4Go suivant)
    \item  un  pointeur vers  un  block  de  troisième niveau  (permet
      d'indexer les 4To suivant)
      % \item  Le jours  ou ont aura  besoin d'indexer des  fichier de
      %   4Po, on ajoutera  une inférence (limite théorique de l'ordre
      %   de 10^40)
    \item Plus le fichier est  gros, plus le nombre d'indirection sera
      important
    \item  Le système  alloue en  priorité les  fichiers dans  le même
      block group que son inode (limite la fragmentation)
    \end{itemize}
  \item Un fichier peut-être un répertoire
    \begin{itemize}
    \item Il  contient un tableau  de structures contenant  l'inode du
      fichier, la taille du nom et le nom de l'entrée
    \item Un répertoire contient  systématiquement une entrée \c{.} et
      une entrée \c{..}
    \item Une entrée dont l'inode est a zero à été supprimée
    \item Le parcours  des réperoitre se fait en O(n),  un index à été
      ajouté sur ext3 pour le faire en $O(log_2(n))$
    \end{itemize}
  \item Remarque: les  première inodes du disque sont  réservée pour :
    la liste des badblocks, le répertoire racine, le journal, etc...
  \item  cf.   \emph{tune2fs(1)},  \emph{debugfs(1)},  \emph{stat(1)},
    \url{http://www.nongnu.org/ext2-doc/ext2.html}
  \end{itemize}
  % Presnter le fonctionnement de fsck.ext2 (c'est interressant)
  % Exo: cacher des fichier dans de l'ext2 (il faut trouver de l'espace, ecrire, et activer la bitmap)
\end{frame}


\begin{frame}[fragile=singleslide]{Journalisation}
  \begin{itemize}
  \item Provient des technoloies des bases de donnée
  \item Permet de garantir l'atomicité des opération sur le disque
  \item On  écrit d'abord  dans le  journal ce que  l'on se  prépare à
    faire
  \item  Une fois  l'action écrite,  on  écrit ensuite  un message  de
    \emph{commit}
  \item Si le système plante, il  lit le journal, si une operation est
    associée à un commit, il execute l'operation, sinon, on estime que
    les information  concernant l'operation sont  peut-être erronée et
    on execute pas l'action
  \end{itemize}
  Evidement, il faut de temps en temps écrire réelement les donnée sur
  le disque.
  \begin{itemize}
  \item On commence par écrire les données
  \item On s'assure qu'elle ont été correctemnt écrite
  \item  Même si  le système  plante à  ce moment,  on  pourra rejouer
    l'entrée du journal
  \item on supprime l'entrée du journal
  \item Pour des raisons de performance, les entrée du journal ne sont
    pas forcement  écrite dans l'ordre. Il faut  alors faire attention
    aux contrainte de précedances
  \end{itemize}
  \begin{lstlisting}
$ debugfs -R logdump /dev/sda8
  \end{lstlisting}
  \begin{itemize}
  \item Il  est possible de  journaliser toutes les données  écrite ou
    seulement les meta donnée
  \item Dans le  second cas, on garanti que  filesystème sera cohérent
    mais pas les donnée à l'intérieur du fichier
  \item Linux implémente aussi  un more \emph{ordered} qui garanti que
    les donnée sont  mise à jour sur le disque avant  de mettre à jour
    les meta  donnée. Ainsi,  il est possible  de perdre  des données,
    mais pas d'avoir des donnée corrompues.
  \end{itemize}
\end{frame}

\begin{frame}[fragile=singleslide]{Copy On Write}
  \begin{itemize}
  \item Certain file système sont dit Copy-On-Write
  \item Il sont montés read-only
  \item Si  un fichier dit  être modifé, une  copie est faite  (sur un
    autre espace ou en mémoire) et la copie est modifiée.
  \item Les futur accès ce feront sur la copie
  \item Permet de faire démarrer des  système sur CD (avec un plus une
    compression dans ce cas)
  \item Permet  de faire fonctionner plusieurs système  sur une unique
    partition montée en lecture seule (virtualisation)
  \item Permet de démarer des  machine en réseau avec un unique dsique
    partagé
  \end{itemize}
\end{frame}
  % 4h (2h pour l'arboressence + 2h pour les algos)
  %
% This document is available under the Creative Commons Attribution-ShareAlike
% License; additional terms may apply. See
%   * http://creativecommons.org/licenses/by-sa/3.0/
%   * http://creativecommons.org/licenses/by-sa/3.0/legalcode
%
% Created: 2012-07-28 20:09:12+02:00
% Main authors:
%     - Jérôme Pouiller <jezz@sysmic.org>
%

\part{La création d'executables}

{
\setbeamertemplate{background canvas}{}
\begin{frame}[plain]
  \partpage
  \begin{textblock}{10}(6,12)
    \begin{quote}
      \rmfamily\textit\textbf\color{darkgray}{\large
      ``There are only 10 kinds of people in this world: those who know binary and those who don't.''}
      %\vskip3mm\hspace*\fill{\small--- William Shakespeare, Hamlet}
    \end{quote}
  \end{textblock}
\end{frame}
}

\begin{frame}
  \tableofcontents
\end{frame}

%% A reformuler, revoir
\section{Compilation}

\begin{frame}[fragile=singleslide]{Compilation}
  Compilation normale:
  \begin{lstlisting}
host$ gcc -c hello.c -o hello.o
host$ gcc hello.o -o hello
host$ ./hello 1
Hello World
  \end{lstlisting} %$
\end{frame}

\subsection{Fonctionnement de la compilation}

\begin{frame}[fragile=singleslide]{Le format ELF}
  \begin{itemize}
  \item La  plupart des fichiers  manipulés par le compilateur  et les
    outils associés sont au format ELF
  \item Ce format est une série de sections et de tables
  \item  Certaines sections seront chargée en mémoires
  \item  Certaines  section  demande  à etre  simplement  allouées  en
    mémoire
  \item  \man{objdump(1)} permet  d'obtenir des  informations  que un
    fichier ELF
  \item  \man{objcopy(1)}  permet   d'extraire  des  sections  ou  de
    modifier un fichier au format ELF
  \item \man{readelf(1)}  et \man{nm(1)}, mais  \cmd{objdump} contient
    toute leurs fonctionnalités
  \end{itemize}
\end{frame}

\begin{frame}[fragile=singleslide]{La compilation}
  La compilation:
  \begin{itemize}
  \item  Execute   le  préprocesseur  (fichier   \file{.i}),  puis  le
    compilateur  vérifie la syntaxe,  le typage,  converti le  code en
    assembleur  (fichier  \file{.s})   puis  en  code  objet  (fichier
    \file{.o})
  \item L'option \file{-c} est utilisé.
  \item Le compilateur doit connaitre les signature des fonction (afin
    de vérifier correctement le typage).
  \item  Les fichiers  headers (fichiers  \file{.h})  sont nécessaires
    pour cette phase
  \item \file{-I}  permet de spécifier des  chemins supplémentaires où
    rechercher des fichiers headers (par défaut: \file{/usr/include})
  \item Si un fichier est  inclu entre double-quotes, il est recherché
    dans le même répertoire que le source.
  \item \cmd{-Wall}, \cmd{-Wextra} recommandés
  \item \cmd{-DMACRO} peut etre utilisé
  \end{itemize}
\end{frame}

\begin{frame}[fragile=singleslide]{Les symboles de debug}
  \begin{itemize}
  \item \cmd{-g}  permet d'ajouter une section  contenant des symboles
    de debug
  \item Utilisés  par les debuggueurs pour  récupérer des informations
    supplémentaire ou pour faire le lien avec les sources
  \item  Les sources  ne sont  pas  incluse dans  les informations  de
    debug.  Seul une  association entre  les adresses  du code  et les
    lignes du source est incluse.
  \item  Attention  aux  modifications   des  code  posterieurs  à  la
    compilation
  \item Cette section n'est pas chargée en mémoire.
  \item Le format utilisé s'apelle \emph{dwarf} (Debug with Arbitrary
    Record Format).
  \item \cmd{strip}  permet de retirer  les section de debug  et autre
    sections utilisé seulement les opération de link
  \end{itemize}
\end{frame}

\begin{frame}[fragile=singleslide]{Les symboles de debug}
  L'option \cmd{-g} ne change pas le code généré:
  \begin{lstlisting}
$ gcc -g -c main.c -o main-dbg.o
$ gcc -c main.c -o main-rel.o
$ ls -l
-rw-rw-r-- jpo jpo 2125 Aug 3 16:05 main-rel.o
-rw-rw-r-- jpo jpo 3720 Aug 3 16:05 main-dbg.o
$ strip *.o
$ ls -l
-rw-rw-r-- jpo jpo 2096 Aug 3 16:05 main-rel.o
-rw-rw-r-- jpo jpo 2096 Aug 3 16:05 main-dbg.o
  \end{lstlisting}
\end{frame}


\subsection{Edition de liens}

\begin{frame}[fragile=singleslide]{Inlining}
  Le compilateur peut effectuer quelques optimisations:
  \begin{itemize}
  \item Les  fonction et les  variables marquées \c{static}  ne seront
    pas exportés, et donc, pas utilisé par les autres fichier objets.
  \item  Le compilateur peut décider d'\emph{inliner} ces fonctions
  \item Si toutes les appels à une fonctions statique ont été inlinés,
    il peut supprimer la fonction du fichier objet.
  \end{itemize}
\end{frame}

\begin{frame}[fragile=singleslide]{L'édition de liens}
  \begin{itemize}
  \item  Le compilateur  ne connait  pas forcement  les  addresses des
    fonctions et des variables globale marquées \c{extern}
  \item  Les endroits  ayant  besoin de  ces fonctions/variables  sont
    remplacés par  des 0 et une  entrée est ajoutée dans  la table des
    \emph{relocation} du fichier objet (cf. \cmd{objdump -R})
  \item  On appelle  le \emph{linker}  (\c{gcc} sans  l'option \c{-c})
    pour résoudre les symboles
  \item Le linker  connait toutes les fonctions, et  toutes les tables
    de relocation.
  \item Il peut déplacer les  addresses de chargement des fonctions et
    des variables de  manière à les mettre au  plus près des fonctions
    qui  les   appellent  (optimisation  de   l'utilisation  du  cache
    d'instruction)
  \item Une  fois l'agencement des fonctions défini,  le linker résoud
    toutes les entrées des tables de relocations
  \end{itemize}
\end{frame}

\begin{frame}[fragile=singleslide]{Placement des symboles de debug}
  \begin{itemize} 
  \item Il  est possible de passer  des options au linker  à partir de
    \cmd{gcc} en utilisant \cmd{-Wl,}
  \item  Il est  possible  de placer  les  symboles de  debug dans  un
    fichier séparé.
  \item Avec l'option \cmd{--build-id=uuid}, il est possible d'ajouter
    un identifiant à la binaire
  \end{itemize} 
  \begin{lstlisting}
$ gcc -Wl,--build-id=sha1 -g main.c -o main
$ cp main -o main.dbg
$ strip --only-keep-debug main.dbg
$ strip main
$ objcopy --add-gnu-debuglink=main.dbg main
$ file main main.dbg
main:     ELF 64-bit... BuildID[sha1]=0x3adaaacf2906a5d2bfb7d415035d2e2, stripped
main.dbg: ELF 64-bit... BuildID[sha1]=0x3adaaacf2906a5d2bfb7d415035d2e2, not stripped
$ objdump -h main main.dbg
  \end{lstlisting}
\end{frame}

\begin{frame}[fragile=singleslide]{Placement des symboles de debug}
  Il est possible de livrer  la binaire à laproduction tout en gardont
  les  symboles de debug  dans un  dépôt séparé.  Il sera  possible de
  demander  au  debugueur  de  les  utiliser pour  résoudre  des  dump
  mémoire. L'identifiant  unique nous permet de garantir  que les deux
  fichiers correspondent bien. On  pourait aussi sauver le code source
  de l'application afin d'avoir toutes les informations pour debuguer.
\end{frame}

\section{Les bibliothèques}

\begin{frame}[fragile=singleslide]{Les bibliothèques}
  \begin{itemize}
  \item  Afin de  simplifier  le déploiment  des  utilitaires, il  est
    possible  d'empaqueter  un  ensemble  de fichier  objet  dans  une
    bibliothèque dite statique (fichier \file{.a}). cf. \man{ar(1)}.
  \item Dans ce cas, cela ne change rien à la procedure de link
  \item Il  est aussi possible d'utiliser  des biliothèques dynamiques
    (fichier \file{.so}). Nous y reviendrons.
  \item  Il  est  possible  de  spécifier  le  chemin  complet  de  la
    bibliothèque (dynamique ou statique)  ou de laisser le compilateur
    la trouver automatiquement avec la syntaxe \cmd{-lbrary}.  On peut
    dans ce cas lui précisier des chemins supplémentaire ou rechercher
    la bibliothèque avec \cmd{-L}
  \item  Par  défaut,  le  compilateur recherchera  les  bibliothèques
    dynamiques, sauf si l'option \cmd{-static} est utilisée
  \item  Le  linker suit  en  fait  une  des recettes  contenues  dans
    \file{/usr/lib/ldscripts/}  (en fonction  des  options passées  au
    linker).   Il   est  possible  de  fabriquer   son  propre  format
    (\cmd{ld -T}) (utile pour générer des firmwares).
  \item Voir \file{u-boot.lds}.
  \end{itemize} 
 % http://www.osdever.net/bkerndev/Docs/basickernel.htm
\end{frame}

\begin{frame}[fragile=singleslide]{Les bibliothèques}
  \begin{itemize} 
  \item  Les  symboles déclaré dans les scripts
    de link sont placé aux endroits demandé en mémoire:
    \begin{lstlisting}
void *etext;
printf("My code end at %p", &etext);
    \end{lstlisting} 
  \item \cmd{-Map=file} permet de sauver le mapping mémoire utilisé
    \begin{lstlisting}
$ gcc -g -Wl,-Map=main.map main.c -o main
    \end{lstlisting} 
  \end{itemize}
\end{frame}

\begin{frame}[fragile=singleslide]{Libtool}
  Problème:
  \begin{itemize}
  \item Les  fichiers objets des bibliothèques  statiques et dynamique
    ne  se  compilent  pas  avec  les mêmes  options  (en  particulier
    \cmd{-fPIC})
  \item  Certaines architectures ne  permettent pas  les bibliothèques
    dynamiques
  \item La  création de bibliothèques  portables peut devenir  un vrai
    casse tête
  \end{itemize}
  \emph{libtool} est outil permettant de faciliter ce travail.
\end{frame}

\subsection{Les bibliothèques dynamiques}

\begin{frame}[fragile=singleslide]{Les bibliothèques dynamiques}
  L'usage de bibliothèques dynamique permet:
  \begin{itemize}
  \item de ne charger qu'un exemplaire de la bibliothèque pour tout le
    système
  \item de simplifier la redistribution du programme
  \item de simplifier les mises à jour de la bibliothèque
  \end{itemize}
\end{frame}

\begin{frame}[fragile=singleslide]{Les bibliothèques dynamiques}
  Coté bibliothèque:
  \begin{itemize}
  \item  Elle contient  une table  des  symboles exportés  et de  leur
    emplacement (dans la table \emph{.dynsym})
  \item A  la construction, elle  doit être linkée  avec \cmd{-shared}
    pour indiquer  que sont  chargement sera différent  d'un programme
    standard (principalement, aucune fonction main n'est nécessaire)
  \end{itemize}
\end{frame}

\begin{frame}[fragile=singleslide]{Les bibliothèques dynamiques}
  Coté éxecutable:
  \begin{itemize}
  \item Le  linker va ajouter une table  d'indirection (la \emph{.got}
    \emph{Global  Offset Table})  pour tous  les symboles  se trouvant
    dans des bibliothèques
  \item Le linker doit passer  par cette indirection pour appeller une
    fonction exportée par une biliothèque dynamique.
  \item Il ne peut pas faire  tenir ce morceau de code à l'emplacement
    laissé par le compilateur (qui  ne fait pas la différence entre un
    symbole statique et un symbole dynamique)
  \item Il crée  donc un petit morceau de code  qu'il placera dans une
    section \emph{Procedure Linkage Table} (\emph{.plt})
  \item Cette  procedure permet simplement  de brancher vers  la table
    d'indirection.
  \end{itemize}
\end{frame}

\begin{frame}[fragile=singleslide]{Les bibliothèques dynamiques}
  \begin{center}
   \pgfimage[width=7cm]{pics/plt_after}
\end{center}
\begin{lstlisting}
$ objdump -d | grep -A 10 "section \.plt"
    \end{lstlisting}
\end{frame}

\begin{frame}[fragile=singleslide]{Les bibliothèques dynamiques}
  A l'éxécution
  \begin{itemize}
  \item Un interpreteur (normalement \file{/lib/ld.so}) est appellé
  \item  Il charge  les bibliothèques  nécessaires (inscrites  dans la
    table \emph{.dynamic}) en mémoire
  \item  Il  remplit la  GOT  avec  des  pointeurs vers  une  fonction
    permettant la résolution du symbole.
  \item Lorsque ce symbol est appellé la première fois, cette fonction
    est appellée.
  \item La fonction résoud le symbol et place son adresse dans la GOT
  \item Cette méthode s'appelle \emph{lazy resolving}
  \item   Cela   peut   être   désactivé  en   passant   la   variable
    d'environnement \cmd{LD_BIND_NOW}
  \end{itemize}
\end{frame}

\begin{frame}[fragile=singleslide]{Les bibliothèques dynamiques}
  \begin{center}
    \pgfimage[width=7cm]{pics/plt_before}
  \end{center}
\end{frame}

\begin{frame}[fragile=singleslide]{Les bibliothèques dynamiques}
  \begin{itemize}
  \item Il est possible de  demander le chargement manuel et dynamique
    des    bibliothèques    et    des    symboles    avec    \c{libdl}
    (\man{dl\_open(3)}, \man{dl\_sym(3)})
  \item Afin d'accélérer le  chargement, \file{ld.so} utilise un index
    de bibliothèques présentes sur le système.
  \item  Lorsque vous  ajoutez une  bibliothèque, vous  devez appeller
    \man{ldconfig(1)} pour mettre à jour ce cache.
  \item la  variable \cmd{LD_LIBRARY_PATH} permet  d'ajouter un chemin
    de recherche de biliothèque pour \file{ld.so}.
  \end{itemize}
\end{frame}

\begin{frame}[fragile=singleslide]{Utilisation de \c{LD_PRELOAD}}
  \begin{itemize}
  \item La  variable d'envionnement \c{LD_PRELOAD}  permet de demander
    le chargement d'une bibliothèque avant les autres
  \item Les symboles exportées par celle-ci seront prioritaires.
  \item Exemple:
    \begin{lstlisting}
unsigned int sleep(unsigned int s) {
    static unsigned int (*real_sleep)(unsigned int s) = NULL;

    if (!real_sleep)
        real_sleep = dlsym(RTLD_NEXT, ``sleep'');

    usleep(5);
    return real_sleep(s);
}
    \end{lstlisting}
  \end{itemize}
\end{frame}

\section{Résultats de la compilation}
% Section: Redistribution? A deplacer apres les outils de compilation?

\begin{frame}[fragile=singleslide]{A qui fournir quoi?}
  \begin{itemize}
  \item La poubelle
    \begin{itemize}
    \item Les dependances
    \item Les objets (ELF)
    \end{itemize}
  \item Les sources
    \begin{itemize}
    \item Les sources
    \item Les headers
    \item Le système de compilation
    \end{itemize}
  \item Les developpeurs externes
    \begin{itemize}
    \item Les headers
    \item Les bibliothèque statique (archve crée avec \man{ar(2)})
    \end{itemize}
  \item L'utilisateur final
    \begin{itemize}
    \item Les bibliothèque dynamique (ELF)
    \item Les binaires (ELF)
    \end{itemize}
  \end{itemize}
\end{frame}

\begin{frame}[fragile=singleslide]{Identifier le résultat}
  Un bon moyen de reconnaitre  les binaires est d'utiliser la commande
  \man{file(1)}:
  \begin{lstlisting}
host$ file ...
hello-dyn:        ELF 64-bit LSB executable, x86-64, version 1 (SYSV), dynamically linked (uses shared libs), for GNU/Linux 2.6.15, not stripped
hello-static: ELF 64-bit LSB executable, x86-64, version 1 (GNU/Linux), statically linked, for GNU/Linux 2.6.15, not stripped
/lib/x86_64-linux-gnu/librt-2.15.so: ELF 64-bit LSB shared object, x86-64, version 1 (SYSV), dynamically linked (uses shared libs), for GNU/Linux 2.6.24, stripped
/usr/lib/x86_64-linux-gnu/librt.a: current ar archive
\end{lstlisting} %$
\end{frame}

\begin{frame}[fragile=singleslide]{Redistribution et licences}
  \begin{itemize}
  \item GPL. Est-ce du travail dérivé?
    \begin{itemize}
    \item Le resultat d'un copier-coller?
    \item Une compilation statique?
    \item Une compilation dynamique?
    \end{itemize}
  \item LGPL
  \end{itemize}
\end{frame}

\section{Makefiles}
% A étoffer?

\begin{frame}[fragile=singleslide]{Makefile}
  \begin{itemize} 
  \item Vieux système (qui a dit ``pourri''?)
  \item Beaucoup d'implémentations différentes
  \item Heureusement,  gmake (GNU Make) est quasiment  le seul utilisé
    de nos jours.
  \item  \cmd{make}  est finalement  un  interpréteur  qui éxecute  le
    fichier  nommé  \file{Makefile}  ce  trouvant dans  le  répertoire
    courant
  \item  Il existe  de nombreuses  alternatives au  Makefile,  mais ce
    dernier  a imposé des  standards qui  se retrouvent  dans beaucoup
    d'autres systèmes (nous en verrons quelques uns plus tard)
  \end{itemize} 
\end{frame} 

\begin{frame}[fragile=singleslide]{Bases}
  \begin{itemize} 
  \item Principe du fichier \file{Makefile}:
    \begin{lstlisting} 
cible: depends
       rules
file.o: file.c dep.h
       gcc file.c -c -o file.o
    \end{lstlisting} 
  \item \cmd{make} crée  un graphe de dépendance et  vérifie les dates
    de modifications  des fichierx afin  de n'éxecuter que  les règles
    nécessaires
  \item Notez que les  dépendances avec les fichiers \file{.h} doivent
    être déclarées
  \item Un \file{Makefile} bien écrit ne reconstruit que le nécessaire
    et n'oublie jamais de reconstruire le nécessaire.
  \item  Il  est  possible  spécifier  à  make une  ou  des  cibles  à
    fabriquer. Sinon, il fabriquera la première cible seulement.
  \end{itemize}
\end{frame}

\begin{frame}[fragile=singleslide]{Les variables}
  \begin{itemize} 
  \item Il est possible d'utiliser des variables
    \begin{lstlisting}
CC = gcc
...
        $(CC) file.c -c -o file.o
    \end{lstlisting} %$
  \item Remarquez que la syntaxe des variable est différente du shell.
  \item ... or, nous appellons du shell à partir des \file{Makefile}
  \item  Les  variables  peuvent  être  surchargées sur  la  ligne  de
    commande:
    \begin{lstlisting}
$ make CC=special-gcc
    \end{lstlisting} %$
  \item  Il  existe de  nombreuses  variables standards,  prédéfinies:
    \c{CC}, \c{CFLAGS},  \c{LDFLAGS}, \c{CXX}, \c{CXXFLAGS}, \c{MAKE},
    ...
  \item  Il existe  aussi des  variables locales  aux  règles: \c{$<},
    \c{$@}, ....
  \end{itemize} 
\end{frame}

\begin{frame}[fragile=singleslide]{Makefile}
  \begin{itemize} 
  \item  Il  est  possible  de définir  des  règles
    génériques:
    \begin{lstlisting} 
%.o: %.c
       gcc $< -c -o $@
file.o: dep.h
    \end{lstlisting} 
  \item De plus, il existe de nombreuses règles implicite qui facilite
    le travail
  \item  Utilisez  au maximum  les  règles  implicites facilite  votre
    travail:
    \begin{lstlisting}
host$ vi toto.c
host$ touch Makefile
host$ make toto
    \end{lstlisting} %$
    \item Un bon Makefile est un Makefile court
  \end{itemize}
\end{frame}

\begin{frame}[fragile=singleslide]{Les règles \emph{phony}}
  \begin{itemize} 
  \item  Il  est  possible  de définir  des  règles  \emph{virtuelles}
    (\emph{phony}) pour simplifier certains traitements
    \begin{lstlisting}
all: exec1 exec2
    \end{lstlisting} 
  \item  Parmis  les règles  \emph{phony}  très répandues:  \c{clean},
    \c{distclean},   \c{install},    \c{all},   \c{check},   \c{dist},
    \c{distcheck}, \c{mrproper}...
  \end{itemize} 
\end{frame}
 % 2h
  %
% This document is available under the Creative Commons Attribution-ShareAlike
% License; additional terms may apply. See
%   * http://creativecommons.org/licenses/by-sa/3.0/
%   * http://creativecommons.org/licenses/by-sa/3.0/legalcode
%
% Created: 2012-07-28 20:09:12+02:00
% Main authors:
%     - Jérôme Pouiller <jezz@sysmic.org>
%

\part{Systèmes de compilation}

\begin{frame}
  \partpage
\end{frame}

\begin{frame}
  \tableofcontents
\end{frame}

% (Re-ajouter l'utilisation des systeme de compilations)


% A placer au debut de la section compilation
\begin{frame}[fragile=singleslide]{Règle d'or}
  \begin{center}
    \huge{Jamais d'espaces dans les chemins de compilation}
  \end{center}
\end{frame}

\subsection{Un projet base de Makefile}

\begin{frame}[fragile=singleslide]{Qu'est-ce qu'un cross-compiler?}
  \begin{itemize}
  \item Permet de compiler vers une cible différente du host
  \item Les binaires produites ne sont pas éxecutable sur le host
  \item La cible de la compilation est généralement décrites dans le nom
  \end{itemize}
  \begin{lstlisting}
$ /opt/arm-sysmic-linux-uclibcgnueabi/usr/bin/arm-buildroot-linux-uclibcgnueabi-gcc source.c
$ PATH+=:/opt/arm-sysmic-linux-uclibcgnueabi/usr/bin/
$ arm-buildroot-linux-uclibcgnueabi-gcc source.c
$ file a.out
a.out: ELF 32-bit LSB executable, ARM, version 1 (SYSV), dynamically linked (uses shared libs), not stripped
  \end{lstlisting}
\end{frame}

\begin{frame}[fragile=singleslide]{Cas générique}
  On utilise les variables prédéfinies de gmake:
  \begin{lstlisting}
make CC=arm-linux-gcc
  \end{lstlisting}
  Mieux dans un sous répertoire séparé:
  \begin{lstlisting}
mkdir arm
cd arm
make -f ../Makefile VPATH=.. CC=arm-linux-gcc
  \end{lstlisting}
  Exemple avec memstat.
\end{frame}

\begin{frame}[fragile=singleslide]{Créer un projet sans autotools}
  Utiliser au maximum les règles implicites facilite votre travail
  \begin{lstlisting}
host$ touch Makefile
host$ make hello
  \end{lstlisting} %$
\end{frame}

\begin{frame}[fragile=singleslide]{Créer un projet sans autotools}
  Utiliser les règles implicites facilite votre travail
  \lstinputlisting{samples/hello/Makefile.1}
  Testons:
\begin{lstlisting}
host$ make CC=arm-linux-gcc CFLAGS=-Wall
\end{lstlisting} %$
\end{frame}

\begin{frame}[fragile=singleslide]{Créer un projet sans autotools}
  \cmd{VPATH} vous permet de géré la compilation \emph{out-of-source}.
  Remarques que pour que \verb+VPATH+ fonctionne correctement, vous devez avoir
  correctement utilisé le quoting pour les directive d'inclusion (\verb+<+ pour
  les entête systèmes et \verb+"+ pour les entêtes du projet)
  % Pas besoin d'ajouter VPATH = . Du coup, c'est la meme chose que Makefile.1
  %\lstinputlisting{samples/hello/Makefile.2}
  Testons:
\begin{lstlisting}
host$ cd build
host$ make -f ../Makefile VPATH=.. CC=arm-linux-gcc
\end{lstlisting} %$
\end{frame}

\begin{frame}[fragile=singleslide]{Créer un projet sans autotools}
  \cmd{gcc} peut  générer les dépendances de vos  fichiers.  On génère
  ainsi  des morceaux  de Makefile  que l'on  inclut. Il  ne  faut pas
  oublier   d'ajouter  la  dépendance   entre  \cmd{hello.d}   et  les
  dépendances de \cmd{hello.c}
  \lstinputlisting{samples/hello/Makefile.3}
  \note{ Voir http://www.makelinux.net/make3/make3-CHP-2-SECT-7.html}
\end{frame}

\begin{frame}[fragile=singleslide]{Créer un projet sans autotools}
  Les  Makefile permettent d'utiliser  des fonctions  de substitutions
  qui  peuvent  nous aider  à  rendre  notre  système plus  générique.
  \lstinputlisting{samples/hello/Makefile.4}
\end{frame}

\begin{frame}[fragile=singleslide]{Créer un projet sans autotools}
  Nous pouvons  ajouter des alias  pour nous aider dans  les commandes
  complexes
  \lstinputlisting[firstline=10]{samples/hello/Makefile.5}
  \note[item]{TODO: Ajouter des \#ifdef CONFIG pour faire dans Kconfig}
  \note[item]{Parler de apt-get source (-b), dpkg -L, dpkg -l, dpkg-buildpackage}
\end{frame}

\subsection{A base d'Autotools}

\begin{frame}[fragile=singleslide]{Historique des Autotools}
  \note[item]{Faire l'historique de configure/Makefile/autotools}
  \begin{enumerate}
  \item Makefile
  \item Makefile + hacks pour effectuer de la configuration
  \item Makefile.in + configure
  \item Makefile.in + configure.ac
  \item Makefile.am + configure.ac
  \end{enumerate}
\end{frame}

\begin{frame}[fragile=singleslide]{Compiler avec autotools}
  \begin{itemize}
  \item C'est le cas le plus courant
  \item Pour une compilation classique:
\begin{lstlisting}
host$ ./configure
host$ make
host% make install
\end{lstlisting} %$
  \item Exemple avec dropbear.
  \item Compilation \emph{out-of-source}. il est nécessaire d'appeller
    le \file{configure} à partir du répertoire de build.
    \begin{lstlisting}
host$ mkdir build
host$ cd build
host$ ../configure
host$ make
host% make install
    \end{lstlisting} %$
  \end{itemize}
\end{frame}

\begin{frame}[fragile=singleslide]{Compiler avec autotools}
  Obtenir de l'aide:
\begin{lstlisting}
host$ ./configure --help
\end{lstlisting} %$

  Parmis les fichiers générés:
  \begin{itemize}
  \item \file{config.log}  contient la sortie  des opération effectuée
    lors de l'appel de \cmd{./configure}.  En particulier, il contient
    la ligne de commande utilisée. Il est ainsi possible de facilement
    dupliquer la configuration.
\begin{lstlisting}
host$ head config.log
\end{lstlisting} %$
  \item     \cmd{config.status}     permet     de    regénérer     les
    Makefile.  \cmd{config.status} est  automatiquement appellé  si un
    Makefile.am est modifié.
  \end{itemize}
  \note{Ajouter un exemple avec tcpdump ou dmalloc}
\end{frame}


\begin{frame}[fragile=singleslide]{Créer un projet avec autotools}
  Fonctionnement des autotools:
  \begin{itemize}
  \item Préparation
\begin{lstlisting}
% apt-get install automake autoconf
\end{lstlisting}
  \item Déclaration de notre programme et de nos sources pour \cmd{automake}
\begin{lstlisting}
$ vim Makefile.am
\end{lstlisting} %$
\begin{lstlisting}
bin_PROGRAMS = hello
hello_SOURCES = hello.c hello.h
\end{lstlisting}
%  \item Les \cmd{autotools}  imposent l'existence de certains fichiers
%    de documentation
%\begin{lstlisting}
%$ touch NEWS README AUTHORS ChangeLog
%\end{lstlisting} %$
  \end{itemize}
\end{frame}

\begin{frame}[fragile=singleslide]{Créer un projet avec autotools}
  \begin{itemize}
  \item  Création  d'un  template  pour \cmd{autoconf}  contenant  les
    macros utiles pour notre projet
\begin{lstlisting}
$ autoscan
$ mv configure.scan configure.ac
$ rm autoscan.log
$ vim configure.ac
\end{lstlisting}
  \item Personnalisation du résultat
\begin{lstlisting}
...
AC_INIT([hello], [1.0], [bug@sysmic.org])
AM_INIT_AUTOMAKE([foreign])
...
\end{lstlisting}
  \item      Génération      du      \file{configure}      et      des
    \file{Makefile.in}. C'est cette version qui devrait être livée aux
    packageurs.
\begin{lstlisting}
$ autoreconf -iv
\end{lstlisting} %$
    \note[item]{Bon, pas de TP la dessus, ca pas très utile}
  \end{itemize}
\end{frame}

\begin{frame}[fragile=singleslide]{Créer un projet avec autotools}
  \begin{itemize}
  \item Compilation
\begin{lstlisting}
$ ./configure --help
$ mkdir build
$ cd build
$ ../configure --host=arm-linux --build=i386 --prefix=$(pwd)/../install
$ make
$ make install
\end{lstlisting} %$
  \end{itemize}
\end{frame}

\begin{frame}[fragile=singleslide]{Créer un projet avec autotools}
  La cible \verb+distcheck+ :
  \begin{enumerate}
  \item Recopie les fichiers référencé dans Autotools
  \item Retire les droits en écriture sur les sources
  \item Lance une compilation \emph{out-of-source}
  \item Installe le projet
  \item Lance la suite de test
  \item Lance un distclean
  \item Vérifie que tous les fichiers créés sont effectivement supprimés
  \item Crée une tarball correctement nommée contenant les sources
  \end{enumerate}
\end{frame}

\begin{frame}[fragile=singleslide]{Créer un projet avec autotools}
  Si \cmd{automake}  est appellé avec  \verb+-gnits+, \verb+distcheck+
  effectue des vérification supplémentaires sur la documentation,
  etc...
  \\[2ex]
  La fonctionnalité \verb+distcheck+ est  le point fort souvent énoncé
  des autotools.
\begin{lstlisting}
$ make distcheck
$ tar tvzf hello-1.0.tar.gz
\end{lstlisting} %$
\end{frame}

%%% A partir de la, je ne sais pas si je le fais

\subsection{A base kmake}

%% A placer apres le système de Makefile?
%% Peut-être ajouter léxemple d'eolane juste avant
\begin{frame}[fragile=singleslide]{Compiler un programme tiers}{Kconfig}
  \begin{itemize}
  \item Système de compilation du noyau
  \item Très bien adapté à la cross-compilation
  \item Pour configurer les options:
    \begin{itemize}
    \item En ncurses
\begin{lstlisting}
host% apt-get install libncurses5-dev
host$ make ARCH=arm CROSS_COMPILE=arm-linux- menuconfig
\end{lstlisting} %$
    \item En Qt3
\begin{lstlisting}
host% apt-get install libqt3-mt-dev
host$ make ARCH=arm CROSS_COMPILE=arm-linux- xconfig
\end{lstlisting} %$
    \end{itemize}
  \item Ne pas oublier d'installer les headers des bibliothèques
    \item Exemple avec busybox
  \end{itemize}
\end{frame}

\begin{frame}[fragile=singleslide]{Compiler un programme tiers}{Kconfig}
  \begin{itemize}
  \item Pour cross-compiler
\begin{lstlisting}
host$ make ARCH=arm CROSS_COMPILE=arm-linux-
\end{lstlisting} %$
  \item Pour compiler \emph{out-of-source}
\begin{lstlisting}
host$ mkdir build
host$ make ARCH=arm CROSS_COMPILE=arm-linux- O=build
\end{lstlisting} %$
  \end{itemize}
\end{frame}

\begin{frame}[fragile=singleslide]{Compiler un programme tiers}{Kconfig}
  Principaux points importants:
  \begin{itemize}
  \item Adapté au environnements embarqué
  \item Adapté aux environnements avec beaucoup de configuration
  \item Initié par le Kernel Linux
  \item  Pas un  système de  compilation réel. Composé de :
    \begin{itemize}
    \item Kconfig, Système de gestion de configuration
    \item  Kmake, règles  Makefile  bien étudiées.  Chaque projet  les
      adapte à ces besoins
    \end{itemize}
  \item Application de la règle: "Pas générique mais simple à hacker"
  \item Dépend principalement de \cmd{gmake}
  \item Mode verbose: \verb+V=1+
  \item Permet d'effectuer des recherche
  \end{itemize}
\end{frame}

\begin{frame}[fragile=singleslide]{Compiler un programme tiers}{Kconfig}
  Test avec busybox:
  \begin{itemize}
  \item Préparation
\begin{lstlisting}
host$ wget http://busybox.net/downloads/busybox-1.18.3.tar.bz2
host$ tar xvjf busybox-1.18.3.tar.bz2
\end{lstlisting} %$
  \item Récupération d'une configuration par défaut
\begin{lstlisting}
host$ make help
host$ make ARCH=arm CROSS_COMPILE=arm-linux- O=build defconfig
\end{lstlisting} %$
  \item Personnalisation de la configuration
\begin{lstlisting}
host% apt-get install libncurses5-dev
host$ make ARCH=arm CROSS_COMPILE=arm-linux- O=build menuconfig
\end{lstlisting} %$
  \end{itemize}
\end{frame}

\begin{frame}[fragile=singleslide]{Compiler un programme tiers}{Kconfig}
  Test avec busybox:
  \begin{itemize}
  \item Compilation
\begin{lstlisting}
host$ make ARCH=arm CROSS_COMPILE=arm-linux-
\end{lstlisting} %$
  \item Installation
\begin{lstlisting}
host$ make ARCH=arm CROSS_COMPILE=arm-linux- install
\end{lstlisting} %$
  \end{itemize}
\end{frame}

\subsection{A base de Cmake}

\begin{frame}[fragile=singleslide]{Cmake}
  \begin{itemize}
  \item Aucune des  solution précédentes ne fonctionne sur  les OS non
    posix (rappel: Cygwin = Couche posix pour Windows)
  \item Cmake ressemble à Automake + Autoconf
  \item Un fichier (CMakeLists.txt) décrit la compilation
    \begin{lstlisting}
cmake_minimum_required (VERSION 2.6)
project (HELLO)
add_executable (hello hello.c)
    \end{lstlisting}
  \item Exemple avec yajl
  \item Cmake génère des fichiers  pour les différent types de système
    de compilation: Makefile, XCode, projet VisualStudio, etc...
    \begin{lstlisting}
$ mkdir build
$ cd build
$ ccmake ..
$ make
    \end{lstlisting}
\end{itemize}
\end{frame}

\begin{frame}[fragile=singleslide]{Cmake}
  Points positifs:
  \begin{itemize}
  \item Portabilité
  \item Syntaxe cohérente (c'est loin d'être le cas de Makefile)
  \item Extensible
  \item  Son  abstration permet  une  prise  en  main facile  pour  un
    débutant
  \end{itemize}
  Sa portabilité rend le niveau d'abstraction de Cmake assez élevé:
  \begin{itemize}
  \item Peut dérouter les habitués
  \item Processus de compilation  complexe à debugguer (c'est aussi le
    cas d'Autotools)
  \item  Faible intégration  avec  les système  de compilation  natifs
    (contrairement à Autotools)
  \item  Certaine action  simple nativement  peuvent  devenir complexe
    dans Cmake
  \item  Nécessite d'installer  cmake  sur la  machine de  compilation
    (Contrairement à Autotools/Kmake)
  \end{itemize}
\end{frame}

\begin{frame}[fragile=singleslide]{Que choisir?}
  \begin{itemize}
  \item  Projet nécessitant  une  bonne integration  avec les  système
    Posix et avec les système Microsoft: Cmake
  \item  Petit  projet, avec  redistribution  restreinte: Makefile  ou
    CMake
  \item Petit projet, mais redistribution large: Autotools
  \item Gros projet de forte complexité: Kmake
  \end{itemize}
\end{frame}
 %2h
  %%
% This document is available under the Creative Commons Attribution-ShareAlike
% License; additional terms may apply. See
%   * http://creativecommons.org/licenses/by-sa/3.0/
%   * http://creativecommons.org/licenses/by-sa/3.0/legalcode
%
% Created: 2011-08-14 17:43:38+02:00
% Main authors:
%     - Jérôme Pouiller <jezz@sysmic.org>
%

\part{Gestion des évènements}

\begin{frame}
  \partpage
\end{frame}

\begin{frame}
  \tableofcontents
\end{frame}

\begin{frame}{Les OS monotâches}
  \begin{itemize}
  \item Permettent  surtout de fournir une  interface de programmation
    commune pour différents programmes ou drivers
  \item Le plus connu est sûrement MS-DOS
  \item Les OS génériques sont maintenant tous multitâches
  \item  Les  derniers  restant   se  trouvent  sur  des  applications
    spécialisés : Consoles de jeux, calculatrice, etc...
  \item Ils ont l'intérêt d'être simple à développer
  \end{itemize}
\end{frame}

\begin{frame}{Une définition}
  \begin{itemize}
  \item \textbf{Temps de réponse} : temps entre un évènement et la fin
    du traitement de l'évènement.
  \end{itemize}
\end{frame}

\section{Scrutation des évènements}
\begin{frame}[fragile]{Scrutation des évènements}
  \begin{itemize}
  \item Aussi appelé \emph{polling}
  \item Boucle infinie
  \item On teste des valeurs des entrées à chaque tour de boucle
  \end{itemize}
  \begin{lstlisting}
#define sensor1 *((char *) 0x1234)
#define sensor2 *((char *) 0xABCD)

int main() {
  while (1) {
    if (sensor1)
      action1();
    if (sensor2)
      action2();
  }
}
  \end{lstlisting}
\end{frame}

\begin{frame}{Scrutation}
  \begin{itemize}
  \item Temps de réponse au  évènements en pire cas facile à calculer:
    Pire temps pour parcourir la boucle
  \item Simple à programmer lorsqu'il  y a peu de périphériques (ayant
    des temps  de réaction  similaires). On peut  les scruter  en même
    temps
  \item Utilisation du CPU sous optimal. Beaucoup de temps est utilisé
    pour lire la valeur des  entrée. Ceci est particulièrement vrai si
    les évènements sont peu fréquents
  \item Si certains évènements entraînent des temps de traitement long
    ou si il y a beaucoup  d'entrées à scruter, le temps de réponse en
    pire cas peut rapidement devenir très grand
  \item Tous les évènements sont traités avec la même priorité
  \item  Mauvaise modularité  du code.  Ajouter des périphériques
    revient à repenser tout le système
  \end{itemize}
\end{frame}

\section{Gestion d'interruptions synchrones}

\begin{frame}{Interruptions synchrones}
  \begin{itemize}
  \item Appelé aussi Background/Foreground
  \item Gestion des évènements dans les interruptions
  \end{itemize}
  \begin{center}
    \pgfimage[width=6cm]{pics/model_bgfg}
  \end{center}
\end{frame}

\begin{frame}[fragile]{Interruptions synchrones}
  Concrètement:
  \begin{lstlisting}
#define PTR_DATA ((char *) 0x1234)

void isr() {
  action1(*PTR_DATA);
  *PTR_DEVICE_ACK = 1;
}

int main() {
  enable_interrupt(isr, 0x1);
  while(1) {
    ; // Optionnal background computing
  }
}
  \end{lstlisting}
\end{frame}

\begin{frame}{Interruptions synchrones}
  \begin{itemize}
  \item Temps de réponse au évènements plutôt bon
  \item Temps de réponse assez simple à calculer. Somme de
    \begin{itemize}
    \item Temps de traitement de l'évènement
    \item Temps de traitement des évènements de priorité supérieures
    \item Temps du changement de contexte (plus ou moins constant)
    \item Pire intervalle de temps ou les interruptions sont désactivées
    \end{itemize}
  \item[$\rightarrow$] Dans  un système simple, ça peut  se calculer à
    la louche
  \item Le  temps de réponse en pire  cas des calculs en  tâche de fond
    est quasiment  identique au traitement  par scrutation (attention
    tout de même à la fréquence maximum des interruptions)
  \end{itemize}
\end{frame}

\begin{frame}{Qu'est-ce qu'une interruption?}
  Il existe trois type d'interruptions:
  \begin{itemize}
  \item Les interruptions matérielles:
    \begin{itemize}
    \item   \textbf{IRQ}   (aussi   appelé   Interruption   externe).
      Asynchrone.   Exemples:  clavier,   horloge,  bus,  DMA,  second
      processeur, etc...
    \item  \textbf{Exception}.   Asynchrone  ou Synchrone.   Exemples:
      Division  par zéro,  Erreur arithmétique,  Erreur d'instruction,
      Erreur d'alignement, Erreur de page, Breakpoint matériel, Double
      faute,  etc...

      \note{Un  overflow arithmétique ne  produit pas  d'exception, il
        lève le flags ``retenue''\\}

      \note{Un  breakpoint  logiciel change  une  instruction par  une
        interruption  logicielle. Un  break point  software  n'est pas
        possible en ROM alors que le breakpoint hardware oui\\}
    \end{itemize}
  \item \textbf{Logicielle}. Déclenchée par une instruction. Synchrone.
  \end{itemize}
\end{frame}

\begin{frame}{Fonctionnement d'une interruption}
  \begin{center}
    \pgfimage[width=10cm]{pics/interuption-1}
  \end{center}
\end{frame}

\begin{frame}{Fonctionnement d'une interruption}
  Quand une  interruption est levée:
  \begin{itemize}
  \item le CPU sauve en  partie ou en totalité le contexte d'exécution
    (principalement le pointeur d'instruction) sur la pile
  \item Le CPU passe en mode superviseur (nous y reviendrons)
  \item  Le CPU  recherche dans  l'IVT (\emph{Interrupt  Vector Table}
    aussi  appelée  IDT,  \emph{Interrupt  Description  Table})  l'ISR
    (\emph{Interruption Service Routine}) associée
  \item Le CPU place le pointeur d'instruction sur l'ISR
  \item  L'ISR traite  l'évènement (fin  de traitement  d'une E/S,
    etc...)
  \item L'ISR acquitte la  réception de l'interruption indiquant qu'une
    nouvelle  donnée peut-être  traitée.
  \item L'ISR restaure le (un) contexte
  \end{itemize}
\end{frame}

\begin{frame}{Fonctionnement d'un PIC}
  Le PIC (Programmable Interrupt Controller) est un composant matériel
  permettant  la gestion  des  IRQ.  Il peut-être  intégré  au CPU  ou
  externe (ou à cheval entre les deux...). Il permet en particulier:
  \begin{itemize}
  \item Activer ou de désactiver des IRQ
  \item De masquer temporairement une IRQ
  \item De mettre en queue une interruption temporairement masquée
  \item De contrôler la priorité des interruptions
  \end{itemize}
  Il arrive fréquemment  qu'un PIC soit multiplexé sur  une seule ligne
  d'IRQ. Dans  ce cas, le premier  étage d'ISR lit un  registre du PIC
  pour connaître  le numéro de l'IRQ.   (Cas notoire du  8259A sur les
  architectures x86)

  \note {Il existe aussi des APIC (Advanced PIC). Sur PC notamment}
\end{frame}

\begin{frame}{Exemple}
  Exemple classique d'intégration d'un PIC multiplexé sur une IRQ:
  \begin{center}
    \pgfimage[width=10cm]{pics/interuption-3}
  \end{center}
\end{frame}

\begin{frame}{Exemple}
  \begin{enumerate}
  \item Le périphérique \emph{Timer} lève sa ligne d'IRQ
  \item Le PIC reçoit l'interruption et lève une IRQ du processeur
  \item  Le processeur  complète  l'instruction courante  et sauve  le
    registre d'instruction (PC) et le registre d'état (PSW)
  \item La tâche courante devient interrompue (Nous y reviendrons)
  \item Le premier étage d'ISR est appelé
  \item  Le  gestionnaire d'interruption  complète  la sauvegarde  des
    registres
  \item   Le  gestionnaire  d'interruption   demande  au   PIC  quelle
    interruption à  été appelée  et il lit  dans l'IVT quelle  ISR est
    associée
  \end{enumerate}
\end{frame}

\begin{frame}{Exemple}
  \begin{enumerate}
    \setcounter{enumi}{7}
  \item Le  gestionnaire d'interruption se branche  sur l'ISR associée
    (ici, ISR1)
  \item L'IRQ du processeur est acquittée. Les autre IRQ peuvent ainsi
    être levées
  \item  L'ISR1 lit la  valeur provenant  du \emph{Timer}  et acquitte
    l'interruption  du \emph{Timer}. Ce  périphérique peut  de nouveau
    lever des IRQ.
  \item Les registres généraux sont restaurés
  \item Le contexte d'exécution est restauré
  \item Le registre PC est restauré
  \end{enumerate}
\end{frame}

\subsection{Utilisation des interruptions}

\begin{frame}{Exemple}
  Exemple de différence d'approche entre la gestion par scrutation et
  la gestion par interruption:\\

  Prenons  l'acquisition  de   donnée  à  partir  d'un  convertisseur
  analogique/numérique asynchrone
  \begin{itemize}
  \item  Dans  le  cas  du  traitement  par  scrutation,  nous  allons
    périodiquement voir  si un résultat est arrivé.  Beaucoup de temps
    est  consommé  pour  rien   et  lorsque  le  résultat  arrive,  le
    traitement du résultat sera retardé
  \item  Une interruption  est  levée quand  une  nouvelle donnée  est
    disponible. Le processeur peut alors la traiter.
  \end{itemize}
\end{frame}

\begin{frame}{Latence des interruptions}
  \begin{itemize}
  \item Un périphérique ne génère pas d'IRQ si la précédente n'est pas
    acquittée (en principe)
  \item Vu  que les  interruptions sont souvent  multiplexées, les
    interruptions sont  souvent désactivées lors de  la première phase
    de traitement
  \item  Pour des  raisons techniques,  il est  parfois  nécessaire de
    désactiver les interruptions
  \item  Le partage  de l'information  entre les  interruptions  et le
    reste   du   programme  nécessite   parfois   de  désactiver   les
    interruptions (Nous y reviendrons)
  \end{itemize}
  Les conséquences:
  \begin{itemize}
  \item Augmente les temps de réponses
  \item Temps réponse plus difficile à calculer
  \item  Risque   de  perdre  des  interruptions  (Dans   ce  cas,  une
    interruption \emph{overrun} est (devrait être) déclenchée)
  \end{itemize}
\end{frame}

\begin{frame}{Précautions avec les interruption}
  \begin{itemize}
  \item Acquitter l'interruption le plus tôt possible
  \item Rester le moins de temps possible dans une interruption
  \item Accéder à un minimum de données pour éviter d'avoir à partager
    des données avec le background
  \item Transférer un maximum de traitement hors de l'interruption
  \item[$\rightarrow$] Gestion des interruptions asynchrones
  \end{itemize}
\end{frame}

\section{Gestion d'interruptions asynchrones}

\begin{frame}[fragile]{Interruptions asynchrones}
  \begin{itemize}
  \item  Interruption  séparée en  deux  parties:  \emph{top half}  et
    \emph{bottom half}
  \item On délègue le maximum de traitement au \emph{bottom half}
  \item Permet de décharger les interruptions
  \item Permet  de plus facilement prendre en  compte des interactions
    entre  les   évènements  (exemple,  possibilité   d'attendre  deux
    évènements avant d'effectuer une action)
  \end{itemize}
\end{frame}

\begin{frame}[fragile]{Interruptions asynchrones}
  \begin{lstlisting}
#define PTR_DATA ((char *) 0x1234)

int gotit = 0;
void isr() {
    gotit++;
    *PTR_DEVICE_ACK = 1;
}

int main() {
  enable_interrupt(isr, 0x1);
  while(1) {
     if (gotit) {
       gotit--;
       action1();
     }
  // Optionnal background computing
  }
}
  \end{lstlisting}
  \note{Il y a un bug à cause  du partage de gotit, mais on en parlera
    plus tard}
\end{frame}

\section{Protection des structures de données}

\begin{frame}[fragile]{Exemple de partage de données}
  Imaginons le code suivant:
  \begin{lstlisting}
#define PTR_DATA ((char *) 0x1234)
int a = 0;
char t[255];
void isr() {
  t[a++] = *PTR_DATA;
  *PTR_DEVICE_ACK = 1;
}
void main() {
  enable_interrupt(isr, 0x1);
  while(1) {
    if (a)
      action1(t[--a]);
    // Optionnal background computing
  }
}
  \end{lstlisting}
\end{frame}

\begin{frame}[fragile]{Exemple de partage de données}
  Prenons le  cas où  \verb+f+ traite l'interruption  précédente (donc
  \verb+a = 1+) et qu'une nouvelle interruption est déclenchée:
  \begin{columns}
    \begin{column}{5cm}
      \begin{lstlisting}[showlines=true,emptylines=10]
--a; // a = 0;





action1(t[a]);
// Lecture de t[1] au lieu de t[0]!
       \end{lstlisting}
     \end{column}
     \begin{column}{5cm}
      \begin{lstlisting}[showlines=true,emptylines=10,escapeinside=\{\}]

t[a] = *PTR_DATA;
// t[0] est {é}cras{é}!
a++;
// a = 1 maintenant
*PTR_DEVICE_ACK = 1;


       \end{lstlisting}
    \end{column}
  \end{columns}
  Au lieu de lire correctement  la première valeur retournée par l'ISR
  puis la seconde, nous lirons  tout d'abord une valeur aléatoire puis
  la valeur retournée par la seconde interruption.
\end{frame}

\begin{frame}{Comment éviter le problème?}
  Les problèmes d'accès concurrents se traduisent très souvent par des
  \emph{races  conditions}.   C'est à  dire  des problèmes  aléatoires
  produit par une séquence particulière d'évènements
  \begin{itemize}
  \item   Les  \emph{races  conditions}   sont  souvent   difficiles  à
    reproduire et à identifier
  \item Les  \emph{races conditions} peuvent être latente  dans le code
    et se déclarer suite à une modification de l'environnement externe
  \item Une race condition coûte chère (difficulté de correction, peut
    potentiellement atterrir en production)
  \end{itemize}
  Comment s'en protéger?
  \begin{itemize}
  \item  Ne  pas  partager de données avec les interruptions
  \item Utiliser des accès atomiques
  \item  Utiliser  des  structures  adaptées: buffers  circulaires  et
    queues
  \item  Désactiver  les  interruptions  lors  d'accès  à  des  données
    partagées
  \end{itemize}
\end{frame}

\subsection{Buffer circulaire}

\begin{frame}[fragile]{Buffer circulaire}
  \begin{lstlisting}
char buf[SIZE];
void init() {
  w = r = 0;
}
void write(char c) {
  if ((w + 1) % SIZE == r )
    ; //buffer is full
  buf[w] = c;
  w = (w + 1) % SIZE;
}
void read() {
  if (w == r)
    ; //buffer is empty
  ret = buf[r];
  r = (r + 1) % SIZE;
}
  \end{lstlisting}
\end{frame}

\subsection{Queue}

\begin{frame}{Queue}
  Même fonctionnement  que le buffer circulaire, mais  avec un tableau
  de structure.

  Il est aussi possible de  faire des \emph{queue} d'objets de tailles
  différentes.  Dans  ce cas, faire très attention  à l'allocation des
  objets.   L'allocation  dynamique est  rarement  une opération  très
  bornée  dans  le temps  et  doit  être  utilisée avec  précaution  à
  l'intérieur des interruptions et des tâches temps réelles.
\end{frame}

\subsection{Désactivation des interruptions}

\begin{frame}{Désactivation des interruptions}
  Si l'utilisation de Buffer circulaires ne résout pas le problème, il
  est possible de désactiver les interruptions.

  La désactivation des interruptions peut entraîner des latences dans la
  gestion  des  interruptions  et  des  pertes  d'interruptions  le  cas
  échéant.
\end{frame}

\begin{frame}{Cas des interruption en milieu multicoeurs}
  \begin{itemize}
  \item  On ne  désactive que  les interruptions  locales (sur  le CPU
    courant)
  \item Une interruption peut se produire sur un autre coeur
  \item Nécessité d'utiliser un mécanisme supplémentaire d'exclusion
  \item \emph{Spin lock} souvent utilisé pour ce cas.
  \item  Pas beaucoup  d'autres  choix.  Par  conséquent les  sections
    critiques dans les interruptions doivent être très limitées
  \end{itemize}
\end{frame}

\subsection{Spin Lock}

\begin{frame}[fragile]{Spin Lock}
  Attente active.\\
  Nécessite une instruction assembleur  permettant un accès en lecture
  et une écriture  en une instruction: \\
  \texttt{test\_and\_set} affecte le registre d'état en fonction de la
  valeur  du registre  et affecte  la valeur  1 au  registre.
  \begin{lstlisting}
void lock(int m) {
  while(atomic_test_and_set(m))
     ;
}

void unlock(int m) {
  m = 0;
}
  \end{lstlisting}
\end{frame}

\section{Les limites du monotâche}

\begin{frame}{Problèmes de la gestion des interruptions asynchrones}
  Nous n'avons pas résolu notre problème récurent:
  \begin{itemize}
  \item  Le partage  de l'information  entre les  interruptions  et la
    boucle principale entraîne des latences
  \end{itemize}
  On retrouve certains problèmes que l'on avait avec la scrutation:
  \begin{itemize}
  \item  Ne permet  pas de  prioriser  les traitements  dans la  boucle
    principale
  \item Interaction entre les évènements complexe
  \end{itemize}
  \note{Parler du mot clef volatile\\}
  \note{Montrer ici  le code du HC08  de la Cobalt.  Commencer par une
    description du but,  du hard: 3 ADC pour  Joystick, deux encodeur,
    clavier matricé,  bus can, prise coaxiale, 20ms,  sytème pour deux
    Cobalt   sur   un  même   réseau   montrer  doc/cobalt/schema   de
    principe.pdf.   Montrer MC68HC908GZ16.h  (montrer  les registres),
    MC68HC908GZ16.c   (montrer   volatile),   link.prm  (montrer   les
    vecteurs),   start.c  (copie  ROM   vers  RAM),   interrupt.c  (en
    particulier intGenlock), main.c (fonction loop)\\}
\end{frame}


\section{Le temps partagé}

\begin{frame}{Concurrence}
  \begin{itemize}
  \item   Des   tâches   concurrentes   sont   des   tâches   exécutées
    séquentiellement sur un seul processeur en entrelaçant l'exécution
    de chaque tâches
  \item  Pour les tâches,  le temps  partagé est  transparent.  Chaque
    tâche à l'impression d'avoir le CPU pour elle-seule
  \item  On  trouvera aussi  le  terme  de  \emph{multitâches} ou  de
    \emph{temps partagé}
  \end{itemize}
\end{frame}

\begin{frame}{Concurence}
   La programmation concurrente N'EST  PAS de la programmation parallèle
  (même les système multicoeurs sont souvent concurrent et parallèle):
  \begin{center}
    \pgfimage[width=10cm]{pics/concurentVsParallel}
  \end{center}
\end{frame}

\begin{frame}[fragile]{Programmation multitâche}
\begin{lstlisting}
#include <unistd.h>

int main() {
  int r;

  r = fork();
  if (r < 0) {
     // Error
  } else if (r > 0) {
    // Parent
  } else /* r == 0 */ {
    // Child
  }
}
\end{lstlisting}
\end{frame}

\begin{frame}{Concurrence}
  Migration d'un système avec gestion asynchrone des interruptions vers
  un système multitâches:
  \begin{center}
    \pgfimage[width=10cm]{pics/model_multitask}
  \end{center}
\end{frame}

\begin{frame}{Tâches concurrentes}
  Pour les  systèmes plus complexes ou pour  facilité la réutilisation,
  un   système   multitâche   est   plus   approprié   qu'un   système
  \emph{Foreground/Background}.
  \begin{itemize}
  \item Facilite la gestion des évènements
  \item Permet de prioriser les traitements
  \end{itemize}
  \note{Parler  des différentes  états des  tâches ici.  Il  manque un
    slide avec du code.}
\end{frame}

\begin{frame}{Etats des tâches}
  \begin{center}
    \begin{tikzpicture}[scale=2]
  \tikzstyle{block} = 
    [rectangle, draw, fill=blue!20, 
     text width=5em, text centered, 
     rounded corners, minimum height=4em]
  \tikzstyle{io} = [ellipse, draw, fill=red!20]

  \node[block]             (Ru)  {Running};
  \node[block,above=of Ru] (Wa)  {Waiting};
  \node[io,   right=of Ru] (Out) {};
  \node[block,below=of Ru] (Int) {Interrupted};
  \node[block,left=of Ru]  (Re)  {Ready};
  \node[io,   left=of Re]  (In)  {};
  \draw[->, line width=1pt] (Re)  -- (Ru);
  \draw[->, line width=1pt] (Ru)  -- (Wa);
  \draw[->, line width=1pt] (Ru)  -- (Int);
  \draw[->, line width=1pt] (Int) -- (Ru);
  \draw[->, line width=1pt] (Int.north west) -- (Re.south);
  \draw[->, line width=1pt] (Wa.south west)  -- (Re.north);
  \draw[->, line width=1pt] (Ru)  -- (Out);
  \draw[->, line width=1pt] (In)  -- (Re);
\end{tikzpicture}

  \end{center}
\end{frame}


\subsection{Changement de contexte}

\begin{frame}{Le changement de contexte}
  Chaque tâche possède une pile en mémoire. Une liste globale contient:
  \note{Faire un schéma}
  \begin{itemize}
  \item les états de toutes les tâches
  \item l'emplacement de la pile en mémoire
  \item le contexte d'exécution, c'est-à-dire une sauvegarde des registres
  \end{itemize}
  Lors du changement de contexte
  \begin{itemize}
  \item  on  sauvegarde  le   contexte  de  la  tâche  précédente,  en
    particulier son pointeur de pile et son pointeur d'instruction
  \item on restaure le contexte de la nouvelle tâche
  \item on restaure le pointeur d'instruction
  \end{itemize}
  Dans la  pratique, il y a  des petites subtilités  dépendantes de la
  manière dont le changement de contexte à été amené.
\end{frame}

\begin{frame}{Le changement de contexte}
  \begin{center}
    \pgfimage[width=7cm]{pics/context_switch}
  \end{center}
\end{frame}

\begin{frame}{Multitâche non-préemptif}
  Le changement de contexte  peut-être volontaire par les tâches. Dans
  ce   cas,  la   tâche   ayant  terminé   son  traitement   appellera
  explicitement   la  fonction   \emph{schedule}   qui  effectuera   la
  changement  de   contexte.  le  système  est   dit  non-péemptif  ou
  multitâche collaboratif.
\end{frame}

\begin{frame}{Multitâche non-préemptif}
  Ce type  de système implique une  latence difficilement quantifiable
  entre un évènement et sont traitement:
  \begin{center}
    \pgfimage[width=7cm]{pics/preemptive-no}
  \end{center}
\end{frame}

\begin{frame}{Multitâche non-préemptif}
  \begin{enumerate}
  \item  Une tâche  non prioritaire  est en  cours d'exécution  et est
    interrompue par un évènement (une IRQ)
  \item L'ISR est appellé
  \item Le traitement  l'IRQ rend une tâche de  haute priorité prête à
    être exécutée
  \item  A  la fin  de  l'ISR,  le système  rend  le  CPU  à la  tâche
    non-prioritaire
  \item Quand  la tâche non-prioritaire termine  sont traitement, elle
    appelle \texttt{schedule}
  \item L'ordonnanceur donne la main à la tâche de forte priorité
  \item La tâche de haute priorité peut (enfin) s'exécuter
  \end{enumerate}
\end{frame}

\begin{frame}{Multitâche préemptif}
  Un  système  multitâche préemptif  va  être  capable  de changer  de
  contexte lors des interruptions:
  \begin{center}
    \pgfimage[width=10cm]{pics/preemptive-yes}
  \end{center}
\end{frame}

\begin{frame}{Multitâche préemptif}
  \begin{enumerate}
  \item  Une tâche  non prioritaire  est en  cours d'exécution  et est
    interrompue par un évènement (une IRQ)
  \item L'ISR est appelée
  \item Le traitement  l'IRQ rend une tâche de  haute priorité prête à
    être exécutée
  \item A la fin de l'ISR, le système appel le scheduler
  \item Le scheduler donne la main a la tâche de haute priorité
  \item  Quand  la tâche  prioritaire  termine  sont traitement,  elle
    appelle \texttt{schedule}
  \item   Vu  qu'il  n'y   a  plus   tâche  prioritaire   à  exécuter,
    l'ordonnanceur redonne la main à la tâche de faible priorité
  \end{enumerate}
\end{frame}

\begin{frame}{Le changement de contexte sur interruption}
  \begin{center}
    \pgfimage[width=10cm]{pics/interuption-2}
  \end{center}
\end{frame}

\begin{frame}[fragile]{Round robin}
  Examinons  le cas de  deux tâches  de priorité  égales n'effectuant
  jamais de relanchement volontaire:
  \begin{lstlisting}
task1() {
  for(;;) ;
}
task2() {
  for(;;) ;
}
  \end{lstlisting}
\end{frame}

\begin{frame}[fragile]{Round robin}
  Dans ce cas, si aucune interruption ne se produit, la première tâche
  à avoir pris la main ne la rendra jamais. Afin de reprendre la main,
  on  utilise une  interruption  d'horloge.  Celle-ci  garanti que  le
  système  pourra périodiquement  reordonnancer les  tâches.  La période
  l'horloge utilisée est appelée quantum de temps ou HZ dans le cas de
  Linux.

  Dans   ce  cas-ci,   l'ordonnanceur  devra   donner  une   période  à
  \emph{task1} puis une période à  \emph{task2} et ainsi de suite.  Ce
  comportement s'appelle \emph{Round-Robin} ou \emph{Tourniquet}.
  \begin{center}
    \begin{tikzpicture}[scale=0.5]
     \timeline{10}{-3.0}{-1.0/A, -2.5/B}
     \fill[cgreen] (0,-1.5) \hi 1 \lo 1 \hi 1 \lo 1 \hi 1 \lo 1 \hi 1 \lo 1 \hi 1 \lo 1;
     \fill[cred]   (0,-3.0) \lo 1 \hi 1 \lo 1 \hi 1 \lo 1 \hi 1 \lo 1 \hi 1 \lo 1 \hi 1;
\end{tikzpicture}
  \end{center}
\end{frame}



% \section{Biographie}

% \begin{frame}{Biographie}
%   \begin{itemize}
%   \item G. Bois, M. De Nanclas, L. Filion, \emph{Real time systems concepts}, Ecole Polytechnique de Montréal
%   \item VxWorks, \emph{VxWorks Programers Guide V5.5}
%   \item Wikipedia: PIC, APIC, Interrupts
%   \end{itemize}
% \end{frame}
 %2h
%  %                                                                                                               
% This document is available under the Creative Commons Attribution-ShareAlike
% License; additional terms may apply. See
%   * http://creativecommons.org/licenses/by-sa/3.0/
%   * http://creativecommons.org/licenses/by-sa/3.0/legalcode
%
% Created: 2011-08-14 17:43:38+02:00
% Main authors:
%     - Jérôme Pouiller <jezz@sysmic.org>
%

\part{Le multitâches}

\begin{frame}
  \partpage
\end{frame}

\begin{frame}
  \tableofcontents[currentpart]
\end{frame}

\section{Le temps partagé}

\begin{frame}{Concurence}
  \begin{itemize}
  \item   Des    tâches   concurentes   sont    des   tâches   éxécutées
    séquentiellement sur un seul processeur en entrelacant l'éxécution
    de chaque tâches
  \item  Pour les tâches,  le temps  partagé est  transparent.  Chaque
    tâche à l'impression d'avoir le CPU pour elle-seule
  \item  On  trouvera aussi  le  terme  de  \emph{multitâches} ou  de
    \emph{temps partagé}
  \end{itemize}
\end{frame} 

\begin{frame}{Concurence}
   La programmation concurente N'EST  PAS de la programmation parallèle
  (même les système multicoeur sont souvent concurent et parallèle):
  \begin{center}
    \pgfimage[width=10cm]{pics/concurentVsParallel}
  \end{center}
\end{frame}

\begin{frame}[fragile]{Programmation multitâche}
\begin{lstlisting}
#include <unistd.h>

int main() {
  int r;

  r = fork();
  if (r < 0) {
     // Error
  } else if (r > 0) {
    // Parent
  } else /* r == 0 */ {
    // Child
  }
}
\end{lstlisting} 
\end{frame} 

\begin{frame}{Concurence}
  Migration d'un système avec gestion asynchrone des interruptions vers
  un système multitâches:
  \begin{center}
    \pgfimage[width=10cm]{pics/model_multitask}
  \end{center}
\end{frame} 

\begin{frame}{tâches concurentes}
  Pour les  systèmes plus complexes ou pour  facilité la réutilisation,
  un   système   multitâche   est   plus   approprié   qu'un   système
  \emph{Foreground/Background}.
  \begin{itemize} 
  \item Facilite la gestion des évènements
  \item Permet de prioriser les traitements
  \end{itemize} 
  \note{Parler  des différentes  états des  tâches ici.  Il  manque un
    slide avec du code.}
\end{frame} 

\begin{frame}{Etats des tâches}
  \begin{center}
    \begin{tikzpicture}[scale=2]
  \tikzstyle{block} = 
    [rectangle, draw, fill=blue!20, 
     text width=5em, text centered, 
     rounded corners, minimum height=4em]
  \tikzstyle{io} = [ellipse, draw, fill=red!20]

  \node[block]             (Ru)  {Running};
  \node[block,above=of Ru] (Wa)  {Waiting};
  \node[io,   right=of Ru] (Out) {};
  \node[block,below=of Ru] (Int) {Interrupted};
  \node[block,left=of Ru]  (Re)  {Ready};
  \node[io,   left=of Re]  (In)  {};
  \draw[->, line width=1pt] (Re)  -- (Ru);
  \draw[->, line width=1pt] (Ru)  -- (Wa);
  \draw[->, line width=1pt] (Ru)  -- (Int);
  \draw[->, line width=1pt] (Int) -- (Ru);
  \draw[->, line width=1pt] (Int.north west) -- (Re.south);
  \draw[->, line width=1pt] (Wa.south west)  -- (Re.north);
  \draw[->, line width=1pt] (Ru)  -- (Out);
  \draw[->, line width=1pt] (In)  -- (Re);
\end{tikzpicture}

  \end{center}
\end{frame} 


\subsection{Changement de contexte}

\begin{frame}{Le changement de contexte}
  Chaque tâche possède une pile en mémoire. Une liste globale contient:
  \note{Faire un schema}
  \begin{itemize} 
  \item les états de toutes les tâches
  \item l'emplacement de la pile en mémoire
  \item le contexte d'éxecution, c'est-à-dire une sauvegarde des registres
  \end{itemize} 
  Lors du changement de contexte
  \begin{itemize} 
  \item  on  sauvegarde  le   contexte  de  la  tâche  précédente,  en
    particulier son pointeur de pile et son pointeur d'instruction
  \item on restaure le contexte de la nouvelle tâche
  \item on restore le pointeur d'instruction
  \end{itemize} 
  Dans  la pratique, il  y a  des petites  subtilités dependant  de la
  manière dont le changement de contexte à été amené.
\end{frame} 

\begin{frame}{Le changement de contexte}
  \begin{center}
    \pgfimage[width=7cm]{pics/context_switch}
  \end{center}
\end{frame}

\begin{frame}{Multitâche non-préemptif}
  Le changement de contexte  peut-être volontaire par les tâches. Dans
  ce   cas,  la   tâche   ayant  terminé   son  traitement   appellera
  explicitement   la  fonction   \emph{schedule}   qui  effectura   la
  changement  de   contexte.  le  système  est   dit  non-péemptif  ou
  multitâche collaboratif.
\end{frame}

\begin{frame}{Multitâche non-préemptif}
  Ce type  de système implique une  latence difficilement quantifiable
  entre un évènement et sont traitement:
  \begin{center}
    \pgfimage[width=7cm]{pics/preemptive-no}
  \end{center}
\end{frame}

\begin{frame}{Multitâche non-préemptif}
  \begin{enumerate} 
  \item  Une tâche  non prioritaire  est en  cours d'éxecution  et est
    interrupue par un évènement (une IRQ)
  \item L'ISR est appellé
  \item Le traitement  l'IRQ rend une tâche de  haute priorité prête à
    être éxécutée
  \item  A  la fin  de  l'ISR,  le système  rend  le  CPU  à la  tâche
    non-prioritaire
  \item Quand  la tâche non-prioritaire termine  sont traitement, elle
    appelle \texttt{schedule}
  \item L'ordonnanceur donne la main à la tâche de forte priorité
  \item La tâche de haute priorité peut (enfin) s'éxécuter
  \end{enumerate} 
\end{frame} 

\begin{frame}{Multitâche préemptif}
  Un  système  multitâche préemptif  va  être  capable  de changer  de
  contexte lors des interruptions:
  \begin{center}
    \pgfimage[width=10cm]{pics/preemptive-yes}
  \end{center}
\end{frame}

\begin{frame}{Multitâche préemptif}
  \begin{enumerate} 
  \item  Une tâche  non prioritaire  est en  cours d'éxecution  et est
    interrompue par un évènement (une IRQ)
  \item L'ISR est appellé
  \item Le traitement  l'IRQ rend une tâche de  haute priorité prête à
    être éxécutée
  \item A la fin de l'ISR, le système appel le scheduler
  \item Le scheduler donne la main a la tâche de haute priorité
  \item  Quand  la tâche  prioritaire  termine  sont traitement,  elle
    appelle \texttt{schedule}
  \item   Vu  qu'il  n'y   a  plus   tâche  prioritaire   à  exécuter,
    l'ordonnanceur redonne la main à la tâche de faible priorité
  \end{enumerate} 
\end{frame} 

\begin{frame}{Le changement de contexte sur interruption}
  \begin{center}
    \pgfimage[width=10cm]{pics/interuption-2}
  \end{center}
\end{frame} 

\begin{frame}[fragile]{Round robin}
  Examinons  le cas de  deux tâches  de priorité  égales n'effectuant
  jamais de relanchement volontaire:
  \begin{lstlisting} 
task1() {
  for(;;) ;
}
task2() {
  for(;;) ;
}
  \end{lstlisting} 
\end{frame} 

\begin{frame}[fragile]{Round robin}
  Dans ce cas, si aucune interruption ne se produit, la premiere tâche
  à avoir pris la main ne la rendra jamais. Afin de reprendre la main,
  on  utilise une  interruption  d'horloge.  Celle-ci  garanti que  le
  système  pourra périodiquement  reordonnancer les  tâches.  La période
  l'horologe utilisée est appellée quantum de temps ou HZ dans le cas de
  Linux.

  Dans   ce  cas-ci,   l'ordonnanceur  devra   donner  une   période  à
  \emph{task1} puis une période à  \emph{task2} et ainsi de suite.  Ce
  comportement s'apelle \emph{Round-Robin} ou \emph{Tourniquet}.
  \begin{center}
    \begin{tikzpicture}[scale=0.5]
     \timeline{10}{-3.0}{-1.0/A, -2.5/B}
     \fill[cgreen] (0,-1.5) \hi 1 \lo 1 \hi 1 \lo 1 \hi 1 \lo 1 \hi 1 \lo 1 \hi 1 \lo 1;
     \fill[cred]   (0,-3.0) \lo 1 \hi 1 \lo 1 \hi 1 \lo 1 \hi 1 \lo 1 \hi 1 \lo 1 \hi 1;
\end{tikzpicture}
  \end{center}
\end{frame}

\section{Pagination de la mémoire}

\begin{frame}{La MMU}
  Le temps partagé  permet de simuler que chaque tâche  est la seule à
  utiliser le CPU.

  En revanche, la mémoire est partagée entre les tâches. Ainsi, si une
  tâche A écrit par erreur sur l'espace d'une tâche B:
  \begin{itemize} 
  \item  La tâche B plante
  \item  Le problème est complexe à trouver
  \item Il  n'y a  aucune moyen  pour empêcher la  tâche A  de faire
    cette action.
  \end{itemize} 
\end{frame}

\begin{frame}{La MMU}  
  Les  CPU  modernes  intègrent   un  composant  appellé  MMU  (\emph{Memory
  Management Unit}):
  \begin{itemize}
  \item  Unite de translation d'addresses mémoire
  \item  On parle d'addresses physiques et virtuelles
  \item Lorsque le  MMU est actif (cas nominal),  toutes les addresses
    du code assembleur sont des addresses virtuelles
  \item  Il est  possible de  configurer le  MMU avec  une instruction
    spéciale et  en lui  donnant un pointeur  sur un tableau  (dans la
    pratique,  il s'agit  plutot d'un  arbre) associant  les addresses
    physiques et les addresses virtuelles
  \end{itemize}
\end{frame}

\begin{frame}{La MMU (2)}  
  \begin{itemize} 
  \item  Il est  possible  de changer  les  associations simplement  en
    chargeant un pointeur sur une autre table
  \item On  défini alors une table  par tâche.  Lors  du changement de
    contexte, on change aussi de table
  \item Le CPU possède alors deux modes:
    \begin{itemize}
    \item  Utilisateur
    \item  Superviseur
    \end{itemize} 
  \item  Seul  le  mode  superviseur  (l'OS) permet  de  modifier  les
    associations de la MMU
  \end{itemize}
  \note{Nous verrons  par la suite comment passer  du mode superviseur
    au  mode utilisateur  et vice  versa\\}
  \note{Vérifier  sur wikipedia  ``addresse  virtuelle'' et  ``mémoire
    paginée''}
\end{frame} 

\begin{frame}{La MMU - gestion des exceptions}
  Toutes les addresses physiques ne sont pas associées à des addresses
  virtuelles
  \begin{itemize} 
  \item Une tâche A ne peut pas accèder à la mémoire d'une tâche B
  \item Protection contre les erreurs de programmation
  \item Permet d'assurer la sécurité des systèmes multi-utilisateurs
  \item Une tâche à l'impression d'avoir toute la mémoire pour elle
  \end{itemize} 
\end{frame} 

\begin{frame}{La MMU - gestion des exceptions}
  Toutes les addresses  virtuelles ne sont pas associées  à des addresses
  physiques
  \begin{itemize}
  \item  Lorsqu'une  tâche accède  à  une  addresse  non associée.   Une
    exception est déclenchée.  Cela permet  à l'OS de reprendre la main
    et de traiter l'erreur (souvent en tuant la tâche fautive)
  \item Lorsqu'une tâche souhaite allouer de la mémoire
    \begin{itemize}
    \item  La tâche demande à l'OS
    \item  L'OS choisi  un (ou  plusieurs) blocs  de  mémoire physique
      libres
    \item L'OS marque le bloc comme appartenant à la tâche
    \item  L'OS choisi un  espace d'adresse  virtuelle où  associer le
      bloc de mémoire
    \item L'OS met à jour la MMU
    \item L'OS retourne l'addresse virtuelle
    \end{itemize} 
  \end{itemize} 
\end{frame} 

\section{Optimisation possible grace à la MMU}

\begin{frame}{La MMU - gestion des exceptions}
  Le MMU permet à l'OS de mieux utiliser la mémoire:
  \begin{itemize} 
  \item  L'OS peut  donner  des espaces  d'addressage virtuel  contigu
    alors que la mémoire physique est fractionnée
  \item Le système n'alloue jamais la plage $[0, 1024]$
    \begin{itemize}
    \item Cela donne une plage de valeurs spéciales (ex: NULL)
    \item Ainsi, lors du debug, vous êtes certains qu'un pointeur $\in
      [0, 1024]$ est non valide
    \item En dehors des pointeurs,  les nombres que l'on manipule sont
      très  souvent <  1024.   Ce système  nous  permet de  rapidement
      repérer des casts abusifs entre des integers et des pointeurs
    \end{itemize} 
  \item ``Sun a inventé le SegFault''
  \end{itemize} 
\end{frame}

\begin{frame}{Gestion de la mémoire}
  Retarder l'association:
  \begin{itemize}
  \item Une tâche demande une allocation
  \item Le système  enregistre la demande dans le  Memory Manager mais
    ne modifie pas le MMU
  \item  Le  système  indique   à  la  tâche  que  l'allocation  s'est
    correctement déroulée
  \item Lorsque la tâche accède à cette page, une exception est levée
  \item Le système reprend la main
  \item Il remarque qu'il avait promis cette page
  \item Il alloue un bloc physique et met à jour la MMU
  \item Il rend la main à la tâche
  \item Tout est transparent pour la tâche
  \end{itemize}
\end{frame}

\begin{frame}{Gestion de la mémoire}
  Utilisation de la Swap:
  \begin{itemize}
  \item Lorsque le système n'a plus assez de mémoire
  \item Il choisit une page physique qu'il copie sur le disque dur
  \item  Il  supprime  la  page   de  la  MMU  de  la  (les)  tâche(s)
    concernée(s)
  \item Lorsque la tâche accède à la page supprimée, une exception est
    levée
  \item Le système recupère alors la page sur le disque
  \item Le système reécrit la page dans la mémoire physique
  \item Il associe l'addresse virtuelle demandée avec la nouvelle page
    physique
  \item L'OS rend la main à la tâche
  \item Tout est transparent pour la tâche
  \end{itemize}
\end{frame}

\begin{frame}{Gestion de la mémoire}
  Gestion des droits sur les pages
  \begin{itemize}
  \item    Il    est     possible    d'affecter    des    droits    en
    lecture/écriture/éxécution sur les pages gérées par la MMU
  \item Si la tâche essaye d'écrire sur une page contenant des données
    constantes, il s'agit d'un bug et une exception est levée
  \item  On garantie  que  les pages  \emph{read-only}  ne seront  pas
    modifiées
  \item Une page contenant des  données constantes (donné ou code) peut
    être mappée dans plusieurs tâches différentes
  \item En retirant les droits  en éxecution sur les pages de données,
    on améliore la sécurité du système (impossible d'exécuter une page
    contenant des données)
  \item Une  page accessible  en écriture peut  être mappée  dans deux
    tâches afin de leur permettre de partager des données
  \end{itemize} 
\end{frame}

\begin{frame}{Gestion de la mémoire}
  Simplification des accès au IO
  \begin{itemize} 
  \item La tâche demande de mapper un fichier en mémoire
  \item  Le système  alloue un  espace d'adressage  virtuel égal  à la
    taille du fichier
  \item Le fichier en lui même n'est pas chargé en mémoire
  \item Lorsque la  tâche accès à un espace  du fichier, une exception
    est levée et la page demandée est chargée de manière tranparente
  \item Le  système marque la  page comme Read-Only. Lorsque  la tâche
    tente d'écrire dans la page,  une exception est levée, la page est
    marquée \emph{dirty}  , les  droits en écriture  sont données  et le
    système rend la main
  \item Lorsque  le système  à besoin de  mémoire, il peut  écrire les
    pages modifiées sur le disque et décharger la page de la mémoire
  \item Lorsqu'une  tâche demande un fichier deja  présent en mémoire,
    on map simplement la MMU sur  la page déjà présente (il faut alors
    gérer correctement le marquage \emph{dirty} de la page)
  \item  cf.   champ  \emph{buffer}  et \emph{cache}  de  la  commande
    \texttt{free}
  \end{itemize}
\end{frame}

\begin{frame}{Gestion de la mémoire}
  Sécurisation des accès aux periphériques
  \begin{itemize}
  \item Lorsque  les registres des périphériques sont  mappés en mémoire,
    on utilise la MMU pour y accèder
  \item Il  est possible d'autoriser  l'accès à un périphérique  à une
    tâche sans lui donner d'accès au reste du système
  \item Un système utilisant  très fortement cette méthode est appellé
    micro-kernel
  \item La méthode est peu utilisée sous Linux
  \item cf. \emph{ioperm(2)}
  \end{itemize}
\end{frame}

\begin{frame}{Passage en mode superviseur}
  Un processus utilisateur ne peut pas passer en mode superviseur.

  Comment passer en mode superviseur?
  \begin{itemize} 
  \item Lorsqu'une interruption/exception est déclenchée
  \item Cela nous permet de faire fonctionner les optimisations précédentes
  \end{itemize} 

  Comment appeller une fonction du système?
  \begin{itemize} 
  \item  Les  tâches ont  besoin  de  faire  des demandes  au  système
    (exemple: allouer de la mémoire)
  \item Ces fonctions système s'appellent des \emph{appels système} ou
    \emph{syscall} (section 2 des pages de man)
  \item  Elles ont  très  peu de  points  communs avec  les appels  de
    fonctions classiques
  \item   Chaque  \emph{syscall}   est   associé  à   un  numéro   (cf
    \texttt{sys/syscall.h}                   \texttt{asm/unistd\_32.h},
    \emph{syscalls(2)})
  \end{itemize}

\end{frame}

\begin{frame}{Passage en mode superviseur}
  Pour utiliser les \emph{appels systèmes} (cf \emph{syscall(2)}):
  \begin{itemize}
  \item On place les arguments sur la pile
  \item On place le numéro de l'interruption sur la pile
  \item On déclenche une interruption logicielle (\texttt{int 0x80})
  \item  Le  CPU  passe  en  mode  superviseur  et  appelle  l'ISR  de
    l'interruption
  \item L'OS prend  la main, regarde le premier element  de la pile et
    appelle la fonction correspondante (\texttt{asm-generic/unistd.h})
  \end{itemize}
  Il  existe maintenant  des instructions  spéciales sur  les  CPU pour
  optimiser   les    \emph{syscall}   (instructions   \emph{sysenter},
  \emph{sysexit})
\end{frame}

\begin{frame}{Threads}
  Thread versus Processus
  \begin{itemize}
  \item On  appelle les tâches  ayant des contextes  mémoires différents
    des \emph{Processus} (cf. \emph{fork(2)})
  \item  Il est  possible  d'éxécuter plusieurs  tâches  dans un  même
    contexte mémoire
  \item  Ces  tâche sont  appellées  \emph{threads} ou  \emph{processus
      lègers} (cf. \emph{clone(2)})
  \item Le fonctionnement  est alors identique au mode  sans MMU, avec
    les mêmes défauts et avantages:
    \begin{itemize}
    \item  Pas  de protection  contre  les  erreurs de programmation  des
      autres threads
    \item Partage de l'information simplifiée
    \item Passage d'une thread à une autre beaucoup plus rapide
    \end{itemize}
  \end{itemize}
  \note{Attention au latence lors de l'allocation, et du swap}
\end{frame} 

\begin{frame}[fragile]{Utilisation de processus}
\begin{lstlisting}
#include <unistd.h>

int main() {
  int r;

  r = fork();
  if (r < 0) {
     // Error
  } else if (r > 0) {
    // Parent
  } else /* r == 0 */ {
    // Child
  }
}
\end{lstlisting} 
\end{frame} 

\begin{frame}[fragile]{Utilisation de threads}
\begin{lstlisting}
#include <pthread.h>

void *task(void *arg) {
  int val = (int) arg;
  // Child
}

int main() {
  int arg = 42
  pthread_t id;
  pthread_create(&id, NULL, task, (void *) arg);
  // Parent
}
\end{lstlisting} 
\end{frame} 

% \begin{frame}{Résumé sur le multiptâche. Commandes ps et pmap, et /proc}
% Memoire
%   \item pmem, \%mem, rss, rsz, rssize: Mémoire résidente, c'est à dire quantité de mémoire physqiue effectivement utilisé par le processus (par consequent, la memoire swapée n'est pas prise en compte). 
%   \item vsz,vsize : Taille de la plage d'addressage virtuelle du processus (sans les mapping de devices). Peut-être très supérieure à rss
%   \item  sz:     size in physical pages of the core image of the process. This includes text, data, and stack space. Device mappings are currently excluded; this is subject to change. See vsz and rss.

% Temps
%        %cpu, pcpu, cp      %CPU   cpu utilization of the process in "##.#" format. Currently, it is the CPU time used divided by the time the process has been running (cputime/realtime ratio), expressed as a percentage.
%                         It will not add up to 100% unless you are lucky. (alias pcpu).

%        bsdstart  START  time the command started. If the process was started less than 24 hours ago, the output format is " HH:MM", else it is "mmm dd" (where mmm is the three letters of the month). See also
%                         lstart, start, start_time, and stime.

%        bsdtime   TIME   accumulated cpu time, user + system. The display format is usually "MMM:SS", but can be shifted to the right if the process used more than 999 minutes of cpu time.

%        cputime, time   TIME   cumulative CPU time, "[dd-]hh:mm:ss" format. (alias time).

%        etime     ELAPSEDelapsed time since the process was started, in the form [[dd-]hh:]mm:ss.

%        lstart    STARTEDtime the command started. See also bsdstart, start, start_time, and stime.

%        c         C      processor utilization. Currently, this is the integer value of the percent usage over the lifetime of the process. (see %cpu).


%        args,cmd, command      COMMANDcommand with all its arguments as a string. Modifications to the arguments may be shown. The output in this column may contain spaces. A process marked <defunct> is partly dead, waiting
%                         to be fully destroyed by its parent. Sometimes the process args will be unavailable; when this happens, ps will instead print the executable name in brackets. (alias cmd, command). See
%                         also the comm format keyword, the -f option, and the c option.
%                         When specified last, this column will extend to the edge of the display. If ps can not determine display width, as when output is redirected (piped) into a file or another command, the
%                         output width is undefined. (it may be 80, unlimited, determined by the TERM variable, and so on) The COLUMNS environment variable or --cols option may be used to exactly determine the
%                         width in this case. The w or -w option may be also be used to adjust width.


%        class, cls, policy, sched     scheduling class of the process. (alias policy, cls). Field's possible values are:
%                         -   not reported
%                         TS  SCHED_OTHER
%                         FF  SCHED_FIFO
%                         RR  SCHED_RR
%                         B   SCHED_BATCH
%                         ISO SCHED_ISO
%                         IDL SCHED_IDLE
%                         ?   unknown value

%        fname     COMMANDfirst 8 bytes of the base name of the process's executable file. The output in this column may contain spaces.

%        comm      COMMANDcommand name (only the executable name). Modifications to the command name will not be shown. A process marked <defunct> is partly dead, waiting to be fully destroyed by its parent. The
%                         output in this column may contain spaces. (alias ucmd, ucomm). See also the args format keyword, the -f option, and the c option.
%                         When specified last, this column will extend to the edge of the display. If ps can not determine display width, as when output is redirected (piped) into a file or another command, the
%                         output width is undefined. (it may be 80, unlimited, determined by the TERM variable, and so on) The COLUMNS environment variable or --cols option may be used to exactly determine the
%                         width in this case. The w or -w option may be also be used to adjust width.


% Droits
%        {e,f,r,s}uid  uid effectif/filesystem/réel/sauvegardé
%        {e,f,r,s}user  utilisateur effectif/filesystem/réel/sauvegardé
%        {e,f,r,s}gid  gid effectif/filesystem/réel/sauvegardé
%        {e,f,r,s}groupe  groupe effectif/filesystem/réel/sauvegardé

%        pgid, pgrp   process group ID or, equivalently, the process ID of the process group leader. (alias pgrp).


%        eip       EIP    instruction pointer.

%        esp       ESP    stack pointer.



%        f         F      flags associated with the process, see the PROCESS FLAGS section. (alias flag, flags).

%        flag      F      see f. (alias f, flags).

%        flags     F      see f. (alias f, flag).


% Signaux:
%        blocked   BLOCKEDmask of the blocked signals, see signal(7). According to the width of the field, a 32-bit or 64-bit mask in hexadecimal format is displayed. (alias sig_block, sigmask).

%        caught    CAUGHT mask of the caught signals, see signal(7). According to the width of the field, a 32 or 64 bits mask in hexadecimal format is displayed. (alias sig_catch, sigcatch).

%        ignored   IGNOREDmask of the ignored signals, see signal(7). According to the width of the field, a 32-bit or 64-bit mask in hexadecimal format is displayed. (alias sig_ignore, sigignore).

%        label     LABEL  security label, most commonly used for SE Linux context data. This is for the Mandatory Access Control ("MAC") found on high-security systems.


%        lwp       LWP    lwp (light weight process, or thread) ID of the lwp being reported. (alias spid, tid).

%        ni, nice        NI     nice value. This ranges from 19 (nicest) to -20 (not nice to others), see nice(1). (alias nice).

%        nlwp      NLWP   number of lwps (threads) in the process. (alias thcount).

%        nwchan    WCHAN  address of the kernel function where the process is sleeping (use wchan if you want the kernel function name). Running tasks will display a dash ('-') in this column.


%        pending   PENDINGmask of the pending signals. See signal(7). Signals pending on the process are distinct from signals pending on individual threads. Use the m option or the -m option to see both.
%                         According to the width of the field, a 32-bit or 64-bit mask in hexadecimal format is displayed. (alias sig).

%        pid       PID    process ID number of the process.


%        ppid      PPID   parent process ID.

%        pri       PRI    priority of the process. Higher number means lower priority

%        psr       PSR    processor that process is currently assigned to.



%        rssize    RSS    see rss. (alias rss, rsz).

%        rsz       RSZ    see rss. (alias rss, rssize).

%        rtprio    RTPRIO realtime priority.


%        s         S      minimal state display (one character). See section PROCESS STATE CODES for the different values. See also stat if you want additional information displayed. (alias state).


%        sess      SESS   session ID or, equivalently, the process ID of the session leader. (alias session, sid).

%        sgi_p     P      processor that the process is currently executing on. Displays "*" if the process is not currently running or runnable.

%        sid       SID    see sess. (alias sess, session).

%        sig       PENDINGsee pending. (alias pending, sig_pend).


%        sigcatch  CAUGHT see caught. (alias caught, sig_catch).

%        sigignore IGNOREDsee ignored. (alias ignored, sig_ignore).

%        sigmask   BLOCKEDsee blocked. (alias blocked, sig_block).

%        size      SZ     approximate amount of swap space that would be required if the process were to dirty all writable pages and then be swapped out. This number is very rough!

%        spid      SPID   see lwp. (alias lwp, tid).

%        stackp    STACKP address of the bottom (start) of stack for the process.

%        start     STARTEDtime the command started. If the process was started less than 24 hours ago, the output format is "HH:MM:SS", else it is "  mmm dd" (where mmm is a three-letter month name). See also
%                         lstart, bsdstart, start_time, and stime.

%        start_timeSTART  starting time or date of the process. Only the year will be displayed if the process was not started the same year ps was invoked, or "mmmdd" if it was not started the same day,
%                         or "HH:MM" otherwise. See also bsdstart, start, lstart, and stime.

%        stat      STAT   multi-character process state. See section PROCESS STATE CODES for the different values meaning. See also s and state if you just want the first character displayed.

%        state     S      see s. (alias s).


%        thcount   THCNT  see nlwp. (alias nlwp). number of kernel threads owned by the process.

%        tid       TID    see lwp. (alias lwp).


%        tname     TTY    controlling tty (terminal). (alias tt, tty).

%        tpgid     TPGID  ID of the foreground process group on the tty (terminal) that the process is connected to, or -1 if the process is not connected to a tty.

%        tt, tty        TT     controlling tty (terminal). (alias tname, tty).

%        ucmd      CMD    see comm. (alias comm, ucomm).

%        ucomm     COMMANDsee comm. (alias comm, ucmd).

%        wchan     WCHAN  name of the kernel function in which the process is sleeping, a "-" if the process is running, or a "*" if the process is multi-threaded and ps is not displaying threads.
% \note{On doit être capable de décrire toute les entrée de la commande ps}

% \end{frame} 

\begin{frame}{Multitâche, MMU et Temps réel} 
  \begin{itemize}
  \item Le multitâche permet une meilleure gestion de la concurence
  \item MMU a de multiple avantages (sécurité, optimisation)
  \item En revanche le fonctionnement  de la MMU entraine de multiples
    exceptions
  \item Une  allocation mémoire  peut d'un seul  coup changer  tout le
    mapping
  \item Un accès en mémoire  peut être immédiat 100 fois mais demander
    un accès aux disque la 101eme fois
  \item  Il devient  difficile de  garantir le  temps de  calcul d'une
    fonction
  \item Les  fonctions systèmes \texttt{mlock}  et \texttt{mlock\_all}
    permettent  de  demander  à  Linux  de garder  des  pages  (ou  la
    totatlité en mémoire)
  \item  Il ne  faut pas  oublier d'allouer  une pile  suffisante avant
    d'appeller \texttt{mlock\_all}
    \note{Ajouter du code à ce sujet} 
  \item Néanmoins, cela ne change pas que l'allocation dynamique ne se
    fait pas en temps constant
  \end{itemize} 
\end{frame} 

\begin{frame}[fragile]{Utilisation de \texttt{mlock}}
\begin{lstlisting}
#include <sys/mman.h>

void alloc_stack_1k() {
  char t[1024];
}

int main() {
  alloc_stack_1k();
  mlockall(MCL_CURRENT | MCL_FUTURE);
}
\end{lstlisting} 
\end{frame} 

% \begin{frame}{Bibliographie}
% \begin{itemize} 
%   \item Thomas Petazoni et ??? ???
%   \item Documentation/ dans les sources du noyau
% \end{itemize} 
% \end{frame}

\note{Montrer l'arborescence du kernel: mm, kernel, include, arch, drivers, scripts, tools, Documentation, }


%  %
% This document is available under the Creative Commons Attribution-ShareAlike
% License; additional terms may apply. See
%   * http://creativecommons.org/licenses/by-sa/3.0/
%   * http://creativecommons.org/licenses/by-sa/3.0/legalcode
%
% Copyright 2010 Jérôme Pouiller <jezz@sysmic.org>
%

\section{Gestion de la mémoire}


  %%
% This document is available under the Creative Commons Attribution-ShareAlike
% License; additional terms may apply. See
%   * http://creativecommons.org/licenses/by-sa/3.0/
%   * http://creativecommons.org/licenses/by-sa/3.0/legalcode
%
% Copyright 2012 Jérôme Pouiller <jezz@sysmic.org>
%

\part{Communication inter-tâches}

\section{Problèmatique des accès concurents}

\subsection{Protection des structures de données}

\begin{frame}[fragile]{Exemple de partage de données}
  Le partage de données entre les tâches posent les mêmes problèmes que
  le partage de données avec les interruptions

  Prennons deux tâches \c{f1} et \c{f2}:
  \begin{lstlisting}
int a = 0;
char t[255];
void f1() {
  t[a] = data1;
  a++;
}
void f2() {
  t[a] = data2;
  a++;
}
       \end{lstlisting}
\end{frame} 

\begin{frame}[fragile]{Exemple de partage de données}
Prennons le cas ou \verb+f1+ est préemptée par \verb+f2+:
  \begin{columns}
    \begin{column}{5cm}
      \begin{lstlisting}[showlines=true,emptylines=10]
t[a] = data1; // t[0];




a++;
// a = 2 maintenant
       \end{lstlisting}
     \end{column}
     \begin{column}{5cm}
      \begin{lstlisting}[showlines=true,emptylines=10,escapeinside=\{\}]

t[a] = data2; 
// t[0] est {é}cras{é}!
a++;
// a = 1 maintenant


       \end{lstlisting}
    \end{column}
  \end{columns}

  Au lieu d'écrire  les deux données l'une après  l'autre, la valeur de
  \verb+data1+ est perdue alors  que \verb+t[1]+ contiendra une valeur
  aléatoire.
\end{frame} 

\subsection{Protection des ressources matériel}

\begin{frame}[fragile]{Exemple de ressource partagée}
  Cas d'un  périphérique réseau avec des registres  mappés en mémoire.
  Le  registre  \c{0xABC0}  permet  de  placer la  donnée  à  envoyer.
  L'écriture  d'un 1  sur  le registre  \c{0xABC4} permet  d'effectuer
  l'envoi:
\begin{lstlisting} 
void send(int data) {
  *0xABC0 = data;
  *0xABC4 = 1;
}
  \end{lstlisting} 
\end{frame} 

\begin{frame}[fragile]{Exemple de ressource partagée}
  Etudions le cas de l'éxecution simultanée de cette fonction par deux
  tâches.  La première tâche appelle \c{send} avec \cmd{data = 42}:
  \begin{lstlisting} 
*0xABC0 = 42
  \end{lstlisting} 
  La  tâche est  préemptée.  La  seconde tâche  appelle  \c{send} avec
  \cmd{data = 10}:
\begin{lstlisting} 
*0xABC0 = 10
*0xABC4 = 1
\end{lstlisting} 
  \c{10} est envoyé. La tâche 1 reprend la main:
\begin{lstlisting} 
*0xABC4 = 1
\end{lstlisting} 
  \c{10} (au lieu de \c{42}) est de nouveau envoyé.\\[3mm]

  Ce cas  est aussi valable dans  le cas d'une  interruption (bien que
  plus rare dans la pratique).
\end{frame} 

\subsection{Réentrance}

\begin{frame}{Comment éviter le problème?}
  Nous devons élargir ce que nous avons précédement vu pour le partage
  d'informations avec les interruptions.

  Les problèmes d'accès concurrents se traduisent très souvent par des
  \emph{races  conditions}.   C'est à  dire  des problèmes  aléatoires
  produit par une séquence particulière d'évènements
  \begin{itemize} 
  \item   Les  \emph{races  conditions}   sont  souvent   difficiles  à
    reproduire et à identifier
  \item Les  \emph{races conditions} peuvent être latente  dans le code
    et se déclarer suite à une modification de l'environnement externe
  \item Une race condition coûte chère (difficulté de correction, peut
    potentiellement atterrir en production)
  \end{itemize} 
\end{frame}

\begin{frame}{Comment s'en protèger?}
  Comment s'en protèger?
  \begin{itemize} 
  \item  Ne  pas  utiliser  de  variables globales  ou  de  ressources
    partagées
  \item Utiliser des accès atomiques
  \item   Placer  des   accès   aux  ressources   partagée  dans   des
    \emph{sections critiques}
  \end{itemize} 
  Une  fonction  pouvant  être  appellée simultanénement  depuis  deux
  contextes de tâches différentes est dite \emph{réentrante}
\end{frame} 

\begin{frame}{Partage de ressources critiques} 
  Une ressource critique ne peut :
  \begin{itemize}
  \item être utilisée simultanément par plusieurs tâches
  \item être réquisitionnée par une autre tâche
  \end{itemize}
  Notion de section critique :
  \begin{itemize}
  \item  séquence  d'instructions pendant  lesquelles  on utilise  une
    ressource critique
  \item sans  problème dans le cas d'un  ordonnancement non préemptif,
    mais  c'est rarement  le cas  dans un  environnement temps  réel
  \item[⇒]  évaluation  du  temps  de réponse  très  difficile,  sinon
    impossible (abondante littérature !)
  \end{itemize}
\end{frame}

\subsection{Partage de ressource entre deux tâches: Fonctionnement d'un mutex}

\begin{frame}[fragile]{Fonctionnement d'un mutex}
  Nécessite une instruction assembleur  permettant un accès en lecture
  et en écriture  en une instruction: \\
  \texttt{test\_and\_set} affecte le registre d'état en fonction de la
  valeur  du registre  et affecte  la valeur  1 au  registre.  On peut
  développer la fonction \c{lock} à partir de là:
  \begin{lstlisting} 
void lock(mutex_t *m) {
  while (test_and_set(m))
    schedule();
}

void unlock(mutex_t *m) {
  m = 0;
  schedule();
}
  \end{lstlisting} 
\end{frame}

\begin{frame}[fragile]{Fonctionnement d'un mutex}
  Un peu mieux:
  \begin{lstlisting} 
void lock(mutex_t *m) {
  while (test_and_set(m)) {
    this_task.reason = m;
    this_task.state = stop;
    schedule();
  }
}

void unlock(mutex_t *m) {
  m = 0;
  foreach (i in tasks)
    if (i.state == stop && i.reason == m)
      i.state = run;
  schedule();
}
  \end{lstlisting}
\end{frame} 

\section{Problème liés aux partage de ressources}

\begin{frame}{Problèmes associés aux sections critiques}
  Voici les problèmes à prendre en considération lors de l'utilisation
  de sections critiques:
  \begin{itemize}
  \item Dead Lock
  \item Latence induite
  \item Inversion de priorité
  \end{itemize} 
\end{frame}

\subsection{Dead Lock}

\begin{frame}[fragile]{Dead lock}
  \begin{itemize} 
  \item Aussi appellé \emph{étreinte fatale}
  \item Deux tâches utilisent  deux ressources imbriquées dans l'ordre
    inverses
  \end{itemize} 
  \begin{columns}
    \begin{column}{5cm}
      Tache 1:
      \begin{lstlisting}[showlines=true,emptylines=10]
lock(m1);



lock(m2);
      \end{lstlisting} 
    \end{column}
    \begin{column}{5cm}
      Tache 2:
      \begin{lstlisting}[showlines=true,emptylines=10]

lock(m2);
// Deadlock ici:
lock(m1);

      \end{lstlisting} 
    \end{column}
  \end{columns}
  \begin{center}
    % AABBAXXX
\begin{tikzpicture}[scale=0.35]
  \timeline{9}{-3.5}{-1.0/A, -2.5/B}
  \fill[color=black!25] (5, -1.5) rectangle +(4, 1);
  \fill[color=black!25] (4, -3.0) rectangle +(5, 1);
  \fill[cgreen] (0,-1.5)  \hi 2 \lo 2 \hi 1 \lo 4;
  \fill[cred]   (0,-3.0)  \lo 2 \hi 2 \lo 1 \lo 4;
  \pattern[pattern=north east lines] (1, -1.5) rectangle +(8, 1);
  \pattern[pattern=north west lines] (3, -3.0) rectangle +(6, 1);
  \pb{0}{-1}{cgreen};
  \pb{2}{-2.5}{cred};
\end{tikzpicture}

  \end{center}
\end{frame} 

\begin{frame}[fragile]{Dead lock}
  \textbf{Remarque:} \\
  Le code suivant:
  \begin{lstlisting} 
lock(m);
lock(m); 
  \end{lstlisting} 
  entraine un  \emph{double lock},  un type particulier  de \emph{Dead
    lock}
\end{frame} 

\begin{frame}[fragile]{Mutex dans une interruption}
  \textbf{Remarque:} \\
  Ne jamais utiliser de mutex dans une interruption.
  \begin{itemize} 
  \item Si  la ressource  est occuppée par  la tâche qui  vient d'être
    préemptée, le \texttt{lock()} s'éxécutera dans le même contexte
  \item[$\rightarrow$] Double lock
  \item De plus,  le blocage d'un mutex peut entrainer
    une  très  important latence  ce  qui  est  en contradiction  avec
    l'objectif de rester le minimum de temps dans une interruption
  \item[$\rightarrow$]  Règle  générale:   Il  ne  faut  pas  appeller
    \texttt{schedule} dans une interruption.
  \end{itemize} 
\end{frame} 

\section{Autres mécanismes de gestion d'accès concurrents}

\note{Pour chacun  des mécanisme,  donner les fonction  dasn plusieurs
  API,  faire  un exemple  ou  mieux,  donner  le code:  Posix,  Java,
  Xenomai, ucosII, vxWorks}

\subsection{Désactivation de l'ordonnanceur}

\begin{frame}{Désactivation de l'ordonnanceur}
  \begin{itemize} 
  \item On demande à l'OS de ne plus être préemptif
  \item Horriblement dangereux
  \item A  priori, à  ne jamais  utiliser sauf pour  faire des  cas de
    tests
  \end{itemize} 
\end{frame} 

\subsection{Sémaphore}

\begin{frame}{Sémaphore}
  \begin{itemize} 
  \item  Différence entre un  mutex et  un semaphore  binaire: presque
    aucune.
  \item Parfois le sémaphore  binaire est utilisé pour implémentéer le
    mutex.
  \item Toutefois,  d'un point de  vue sémantique, on  pourrait pemser
    que  le  mutex  permet  d'avoir  un morceau  de  code  mutuelement
    exclusif tandis que le sémaphore  est une section de code limité à
    une ressource.
  \item  Généralement,  les algorithmes  d'héritage  de priorité  sont
    implémentés sur les mutex mais pas sur les sémaphore.
%  \item  Notons  aussi  que  les  algorithmes  d'héritage  de  priorité
%    sont encore plus complexe sur les sémaphores
  \end{itemize} 
\end{frame} 

\subsection{Mutex réentrant}

\begin{frame}{Mutex Réentrant}
  \begin{itemize} 
  \item Idem  mutex, mais  si la même  tâche tente de  revérouiller le
    même mutex, le mutex est non-bloquant.
  \item Dans le cas  d'un mutex non-réentrant, ceci entraine forcement
    un dead-lock.
  \item Un sémaphore est maintenu pour connaitre le nombre de passage.
  \end{itemize} 
\end{frame} 

\begin{frame}[fragile=singleslide]{Les API}
  \begin{itemize} 
  \item   Posix:   \c{pthread_mutex_lock},   \c{pthread_mutex_unlock},
    \c{sem_wait}, \c{sem_post}
  \item Linux: \c{futex}
  \item Noyau Linux: \c{mutex_lock}, \c{mutex_unlock}
  \item  Xenomai:  \c{rt_sem_p},  \c{rt_sem_v},  \c{rt_mutex_acquire},
    \c{rt_mutex_release}
    \item Win32: \c{EnterCriticalSection}, \c{LeaveCriticalSection}
  \end{itemize} 
\end{frame} 


\subsection{R/W Lock}

\begin{frame}[fragile]{Read/Write Lock}
\note{\url{http://en.wikipedia.org/wiki/Readers-writers\_problem}}

Permet de limiter le phénomène  de latence en dimiminuant le nombre de
sections critiques.

Solution 1 (\emph{reader preference}):
\begin{columns}
  \begin{column} {5cm}
    \begin{lstlisting} 
void read_lock() {
  // mutex protege read_count
  lock(mutex);
  readcount++;
  if (readcount == 1)
    lock(w);
  unlock(mutex);
}
    \end{lstlisting} 
  \end{column}
  \begin{column} {5cm}
    \begin{lstlisting} 
void read_unlock() {
  lock(mutex);
  readcount--;
  if (readcount = 0)
    unlock(w);
  unlock(mutex);
}
void write_lock() {
   lock(w);
}
void write_unlock() {
  unlock(w);
}
    \end{lstlisting} 
  \end{column}
\end{columns}
\end{frame} 

\begin{frame}[fragile]{Read/Write Lock}
  Problème: un accès en écriture doit attendre que toutes les lectures
  soient terminées. Solution 2 (\emph{writer preference}):
  \begin{columns}
    \begin{column} {5cm}
      \begin{lstlisting} 
void read_lock() {
  lock(r);
  lock(mutex);
  readcount++;
  if (readcount == 1)
     lock(w);
  unlock(mutex);
  unlock(r);
  // r n'est pas bloque durant la lecture
}
       \end{lstlisting} 
     \end{column}
     \begin{column} {5cm}
       \begin{lstlisting} 
void read_unlock() {
  lock(mutex);
  readcount--;
  if (readcount == 0)
     unlock(w);
  unlock(mutex);
}
void write_lock() {
   lock(r);
   lock(w);
}
void write_unlock() {
  unlock(w);
  unlock(r);
}
      \end{lstlisting} 
    \end{column}
  \end{columns}
\end{frame} 

\begin{frame}[fragile=singleslide]{Les API}
  \begin{itemize} 
  \item  Posix:  \c{pthread_rwlock_wrlock}, \c{pthread_rwlock_rdlock},
    \c{pthread_rwlock_unlock}
  \item Xenomai: $\emptyset$
  \item  Win32: \c{AcquireSRWLockExclusive}, \c{AcquireSRWLockShared},
    \c{ReleaseSRWLockExclusive}, \c{ReleaseSRWLockShared}
  \end{itemize} 
\end{frame} 

\subsection{Rendez-vous ou barrier}
\begin{frame}[fragile]{Rendez-vous}
  Permet de synchroniser  deux tâches. La première tâche  arrivée à la
  barrière attend la seconde.
\begin{lstlisting} 
void init() {
  lock(m1);
  lock(m2);
}
\end{lstlisting}
Tâche 1:
\begin{lstlisting} 
  unlock(m1);
  lock(m2);
\end{lstlisting} 
Tâche 2:
\begin{lstlisting} 
  unlock(m2);
  lock(m1);
\end{lstlisting} 
\end{frame}

\begin{frame}[fragile=singleslide]{Les API}
  \begin{itemize} 
  \item Posix: \c{pthread_barrier_wait}
  \item Xenomai: $\emptyset$
  \item Win32: \c{EnterSynchronizationBarrier}
  \end{itemize} 
\end{frame} 

\subsection{Condition}

\begin{frame}[fragile]{Conditions}
  Peuvent  être   considérés  comme  des   \emph{rendez-vous}  à  sens
  unique. Si une tâche attend, elle est débloquée, sinon, aucun effet.
  Très utilisée pour le pattern des \cmd{work-thread}
  \begin{columns}
    \begin{column}{5cm}
      \begin{lstlisting} 
void init() {
  lock(m);
}

void wait() {
  lock(m);
}

void signal() {
  unlock(m);
  try_lock(m);
}
      \end{lstlisting}
    \end{column}
    \begin{column}{5.5cm}
      \begin{lstlisting} 
// broadcast debloque
// tous les waiters alors
// que signal en debloque
// uniquement un
void broadcast()  {
  // Plus complexe, il
  // faut un mutex par
  // waiters. 
}
      \end{lstlisting} 
    \end{column}
  \end{columns}
\end{frame} 

\begin{frame}[fragile=singleslide]{Les API}
  \begin{itemize} 
  \item    Posix:    \c{pthread_cond_wait},   \c{pthread_cond_signal},
    \c{pthread_cond_broadcast}
  \item Linux: \c{eventfd}, \c{select}, \c{read}, \c{write}
  \item Noyau Linux: \c{wait_event}, \c{wait_up}
  \item Xenomai: \c{rt_cond_signal}, \c{rt_cond_broadcast}, \c{rt_cond_wait}
  \item              Win32:              \c{WakeAllConditionVariable},
    \c{WakeConditionVariable}, \c{SleepConditionVariableCS}
  \end{itemize} 
\end{frame} 

\subsection{Buffer circulaire et Queue}

\begin{frame}[fragile]{Buffer circulaire et Queue}
  \begin{itemize} 
  \item  Précédement décrit dans  la section  ``Partage d'information
    avec les interruptions''
  \item Fonctionne aussi très bien entre les tâches
  \item Une  des rares structures  permettant d'être partagée à  la fois
    avec une interruption et des tâches
  \item Faire attention à l'allocation dynamique des objets
  \end{itemize}
\end{frame} 

\begin{frame}[fragile=singleslide]{Les API}
  \begin{itemize} 
  \item Posix/Linux: \c{mq_receive}, \c{mq_send}, \c{pipe}, \c{socket}
  \item Noyau Linux: \c{kfifo_put}, \c{kfifo_get}
  \item      Xenomai:     \c{rt_buffer_write},     \c{rt_buffer_read},
    \c{rt_queue_send}, \c{rt_queue_receive}
  \item Win32: \c{GetMessage}, \c{WaitMessage}
  \end{itemize} 
\end{frame} 


\subsection{Spin Lock, Mutex et désactivation des interruptions}

\begin{frame}[fragile]{Spin Lock, Mutex et désactivation des interruptions}
  Lorsque:
  \begin{itemize} 
  \item  Vos  sections critiques  font  intervenir  des  tâches et  des
    interruptions
  \item Votre problème ne concerne  pas un échange de données (et donc
    le buffer circulaire n'est pas une solution)
  \item Vous ne pouvez pas faire autrement
  \end{itemize} 
  alors,   vous  devez   combiner  les   trois   mécanismes  suivants:
  Désactivation des interruptions, Mutex et Spin Lock.

  Le point sur ces trois mécanismes:
  \begin{itemize} 
  \item  Si  une  ressource  est   partagée  entre  une  tâche  et  une
    interruption sur  le même coeur,  il est nécessaire  de désactiver
    les interruptions
  \item  Si une ressource  est partagé  entre deux  tâches sur  un même
    coeur, il est nécessaire d'utiliser un mutex
  \item Si  la ressource est  partagée avec un  autre coeur et  que le
    temps d'utilisation est court, utilisez un Spin Lock.
  \end{itemize}
\end{frame}

\begin{frame}[fragile]{Spin Lock, Mutex et désactivation des interruptions}
  On pourrait imaginer un cas cummulant les trois contraintes:
  \begin{lstlisting} 
disable_interrupts()
mutex_lock(m)
spin_lock(s)
a++
spin_unlock(s)
mutex_unlock(m)
enable_interrupts()
  \end{lstlisting} 
  Globalement, évitez!
\end{frame} 

\subsection{Algorithmes non-bloquants}

\begin{frame}{Algorithmes non-bloquants}
  \begin{itemize} 
  \item Algorithme thread-safe n'utilisant pas de sections ciritques.
  \item Ces  algorithmes utilisent souvant  des instructions atomiques
    proposés par certains processeurs
  \item Par conséquent, ils sont peu portables
  \item Souvent utilisé dans les bases de données
  \end{itemize} 
\end{frame} 

\subsection{Read-Copy-Update (RCU)}

\begin{frame}{Read-Copy-Update (RCU)}
  Type d'algorithme non bloquant:
  \begin{itemize} 
  \item La lecture n'est pas bloquante
  \item On note le nombre de lecteurs
  \item Les modifications s'effectuent sur une copie de l'objet
  \item Les lectures suivante se font sur la nouvelle version de l'objet
  \item Lorsque  le dernier lecteur  a terminé, l'objet  d'origine est
    détruit. Seul subsiste la nouvelle version.
  \end{itemize} 
\end{frame}

\begin{frame}[fragile]{Exemple : Manipulation de listes} 
  \begin{center}
    \begin{lstlisting}[basicstyle=\ttfamily\scriptsize\color{colBasic},commentstyle=\scriptsize\itshape\color{colComments},numbers=none]
typedef struct {
   struct a_t a;
   int count_usage = 0;
   bool obsolete = false;
} rcu_t;
rcu_t *a = malloc(sizeof(rcu_t)); 
    \end{lstlisting}
  \end{center}
  \begin{columns}
    \begin{column}{5cm}
      \begin{lstlisting}[basicstyle=\ttfamily\scriptsize\color{colBasic},commentstyle=\scriptsize\itshape\color{colComments},numbers=none]
void read_a() {
  // lock:
  rcu_t *ptr = a;
  ptr->count_usage++;
  // do something with ptr;
  // unlock:
  ptr->count_usage--;
  if (ptr->obsolete && !ptr->count_usage)
    free(ptr);
}
      \end{lstlisting}
    \end{column}
    \begin{column}{5cm}
      \begin{lstlisting}[basicstyle=\ttfamily\scriptsize\color{colBasic},commentstyle=\scriptsize\itshape\color{colComments},numbers=none]
void write_a() { 
   struct rcu_t *a3 = a;
   struct rcu_t *a2 = malloc(sizeof(rcu_t));
   memcpy(a2, a);
   // modify a2;   
   a = a2;  
   a3->obsolete = true;
   if (!a3->count_usage)
      free(ptr);
}
      \end{lstlisting} 
    \end{column}
  \end{columns}
\end{frame} 

\subsection{Mémoire partagée}

\begin{frame}[fragile=singleslide]{La mémoire partagée}
  \begin{itemize} 
  \item Utilise  le MMU pour partager  une page de  mémoire entre deux
    processus
  \item dans  cette page de  mémoire, il est possible  d'appliquer les
    mécanisme        de       synchronisation        des       threads
    (\man{pthread\_mutexattr\_getpshared(3)})
  \item 
  \end{itemize} 
\end{frame} 

\begin{frame}[fragile=singleslide]{Les API}
  \begin{itemize} 
  \item Posix: \c{shm_open}, \c{mmap}, \c{ftruncate}
  \item Xenomai: \c{rt_heap_create}, \c{rt_heap_bind}
  \item     Win32:     \c{CreateFileMapping},     \c{OpenFileMapping},
    \c{MapViewOfFile}
  \end{itemize} 
\end{frame} 

\subsection{Les signaux}

\begin{frame}[fragile=singleslide]{Les signaux}
  \begin{itemize}
  \item Idée issue de l'immitation des interruption sur l'OS
  \item +/- spécifique aux systèmes Posix
  \item L'histoire a rendu l'API un peu bordélique
  \item     \man{sigaction(2)},     \man{signal(7)},    \man{kill(2)},
    \man{kill(1)}, \man{sigqueue(3)}
  \item Il existe 64 signaux sous Linux
  \item Certain  signaux peuvent  être envoyé à  partir de  la console
    (c'est  le   noyau  qui  traduit   les  touches  en   signaux,  cf
    \man{stty(1)})
  \end{itemize}
\end{frame}

\begin{frame}[fragile=singleslide]{Les signaux}
  \begin{itemize}
  \item  Les  signaux  <  32  sont nommés  et  ont  une  signification
    particulière:
    \begin{columns}
      \begin{column}{3cm}
        \begin{itemize} 
        \item 1: HUP
        \item 2: INT (\c{^C})
        \item 3: QUIT (\c{^\\})
        \item 4: ILL
        \item 5: TRAP
        \item 6: ABRT
        \item 7: BUS
        \item 8: FPE
        \end{itemize}
      \end{column}
      \begin{column}{3cm}
        \begin{itemize} 
        \item 9: KILL
        \item 10: USR1
        \item 11: SEGV
        \item 12: USR2
        \item 13: PIPE
        \item 14: ALRM
        \item 15: TERM
        \item 16: STKFLT
        \end{itemize}
      \end{column}
      \begin{column}{3cm}
        \begin{itemize} 
        \item 17: CHLD
        \item 18: CONT
        \item 19: STOP (\c{^Z})
        \item 20: TSTP
        \item 21: TTIN
        \item 22: TTOU
        \item 23: URG
        \item 24: XCPU
        \end{itemize}
      \end{column}
      \begin{column}{3cm}
        \begin{itemize} 
        \item 25: XFSZ
        \item 26: VTALRM
        \item 27: PROF
        \item 28: WINCH
        \item 29: POLL
        \item 30: PWR
        \item 31: SYS
        \end{itemize}
      \end{column}
    \end{columns}
    \vspace{2ex}
  \item Il existe  un comportement par défaut pour  chaque signal (fin
    de la tâche, suspension, coredump, ignore)
  \item Il  est possible d'associer ces propres  fonctions aux signaux
    (sauf quelques uns)
  \end{itemize}
\end{frame}

\begin{frame}[fragile=singleslide]{Signaux Temps Réels}
\begin{itemize} 
\item Les signaux > 32 sont dit \emph{temps-réel}.
    \begin{itemize}
    \item Plusieurs signaux RT peuvent être en attente
    \item Garantie que les signaux arrivent dans l'ordre dans lesquels
      ils ont été envoyés
    \item Possibilité de passer des valeurs en arguments
    \end{itemize} 
  \item Tombent en désuétude. Remplacés par des \emph{file descriptor}:
    \begin{itemize} 
    \item \man{signalfd(2)}
    \item \man{eventfd(2)}
    \item \man{timerfd\_create(2)}
    \item \man{inotify(7)}
    \end{itemize}  
  \end{itemize} 
\end{frame}





%  \part{Architectures des OS}

\section{Les architectures standard}



% Environs 10 slides -> 30min

% Composition d'un OS multitache
%   Scheduler
%   Gestionnaire de mémoire
%   Des drivers
%   Une API et des services necessitant un "chef d'orchestre": IPC, réseaux, filesystèmes

% Petite partie, a placer avec la virtualisation?
% Noyau monolithique
%   .... mais modulaire
% Micronoyau
% Kernel Hybride
% Hyperviseurs
% Choix des méacanise de communication noyau/user et user/user
% 
% Divrsité de l'API
%   Nombre d'appels système: qqs uns comme Hurd, ou beaucoup comme Linux
%   API stable et definie (Linux) vs API movante (Windows)
%   "Schema de Linux" -> "Linux ne designe que le noyau, etc..."

\section{La virtualisation}





\end{document}

%%% Local Variables:
%%% mode: latex
%%% TeX-master: t
%%% End:
