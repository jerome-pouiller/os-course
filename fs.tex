%
% This document is available under the Creative Commons Attribution-ShareAlike
% License; additional terms may apply. See
%   * http://creativecommons.org/licenses/by-sa/3.0/
%   * http://creativecommons.org/licenses/by-sa/3.0/legalcode
%
% Created: 2012-07-28 20:09:12+02:00
% Main authors:
%     - Jérôme Pouiller <jezz@sysmic.org>
%

\section{Les systèmes de fichiers}

\subsection{L'arborescence}

\begin{frame}[fragile=singleslide]{Généralité}
  Les types de fichiers
  \begin{itemize}
  \item Normal (\emph{touch(1)}, \emph{open(2)}, \emph{mknod(2)})
  \item Répertoire (\emph{mkdir(1)}, \emph{mkdir(2)})
  \item Lien symbolique (\emph{ln(1)}, \emph{symblink(2)})
  \item Pipe nommé (\emph{mkfifo(1)}, \emph{mkfifo(3)}, \emph{mknod(2)})
  \item Socket nommé (\emph{bind(2)}, \emph{mknod(2)})
  \item Fichier périphérique charactère (\emph{mknod(1)}, \emph{mknod(2)})
  \item Fichier périphérique bloc (\emph{mknod(1)}, \emph{mknod(2)})
  \end{itemize}

  Par ailleurs, il  est possible de faire pointer  une nouvelle entrée
  vers  une structure e  fichier exsitante.  Ce mécanisme  est appelle
  \emph{hard link} (\emph{ln(1)}, \emph{link(2)}). Certain système COW
  fonctionnes ainsi.
  \\
  Tous  les  caractères  sont  authorisés  sauf  \c{/}  (réservé  pour
  séparerr les  répertoires) et \c{\0}  (réservé pour indiquer  la fin
  des arguments)
\end{frame}

\begin{frame}[fragile=singleslide]{Arborescence}
  \begin{itemize}
  \item \c{/bin} \c{/sbin} \c{/usr/bin} \c{/usr/sbin}: Binaires
  \item \c{/} et \c{/usr} séparé pour des raison historiques
  \item \c{*/sbin}: Binaire normalement réservée à root
  \item \c{/lib*} \c{/usr/lib*}: Bibliothèque
  \item  \c{/etc}:  Fichiers  de  configuration système.  Beaucoup  se
    terminent en \c{*rc.conf}
  \item \c{/home}, \c{/root} Espace des utilisateurs
  \item \c{/var} Répertoire de travail des applciation systèmes
  \item \c{/var/spool}  Queue de  traitement de certains  démon (mail,
    imprimante, etc...)
  \item \c{/var/log} Logs système
  \item \c{/var/cache} Cache système de cartains outils (index de man,
    version binaire des index apt, debconf, etc...)
  \item \c{/var/run}  et \c{/run}  Fichier de communication  entre les
    services  (fichiers  de  lock,  PIDs, sockets  des  communication,
    identifiants de mémoire partagée, etc...)
  \item \c{/var/lib} Données de travail de certaine bibliothèque (apt,
    ...)
  \item  \c{/var/www}, \c{/var/mail},  ...  Dedié aux  partage web  et
    mails
  \item \c{/usr/share} Donnée  statiques de certains services (icones,
    fonts, internationalisation, configurations statiques, etc...)
  \item \c{/usr/share/man} Pages de man
  \item \c{/usr/share/doc} Autre documentation, Licences
  \item  \c{/usr/include} Headers des  bibliothèques C  (installés par
    les version \c{*-dev} des packets)
  \item  \c{/usr/local} \c{/opt}  Application installée  en  dehors du
    service de packets normal (compilés localement)
  \item \c{/tmp} et  \c{/var/tmp} Répertoire tempsoraire. \c{/tmp} est
    vidé à chaque redémarrage
  \end{itemize}
  Le répertoire \c{/dev}
  \begin{itemize}
  \item Fichiers spéciaux, \emph{file devices}
  \item Communiquent avec des drivers (sous Unix, tout est fichier)
  \item \emph{dd(1)}  permet un control  plus fin et est  souvent plus
    approprié  que  \emph{cat}  et  \emph{echo}  pour  accèder  à  ces
    fichiers
  \item Certains  périphérique nécessite l'utilisation  d'autre appels
    système qui nous verrons plus tard
  \item Dans  tous les cas,  ces appels systèmes passent  utilisent un
    file descriptor comme identifiant
  \item Quelques exemples:
    \begin{itemize}
    \item \c{/dev/ttyS0}: Premier port série
    \item \c{/dev/sda}: Premier disque
    \item \c{/dev/sr0}: Lecteur CD
    \item \c{/dev/mem}: Mémoire physique
    \item \c{/dev/zero}: Périphérique virtuel qui ne donne que des 0
    \item \c{/dev/random}: Source d'entropie
    \item \c{/dev/null}: Trou noir
    \item \c{/dev/psaux}  et \c{/dev/input/*}: Periphériques d'entrées
      (souris, clavier, touchsreen, etc...)
    \item \c{/dev/snd/}: Cartes son
    \item \c{/dev/rtc0}: Horloge (plus spécial à accèder)
    \item  \c{/dev/video0}, \c{/dev/nvidia}:  Webcam,  carte vidéo  ne
      s'accèdent pas directement (nous y reviendrons)
    \end{itemize}
    N'apparaissent pas:
    \begin{itemize}
    \item Les bus (cas très rare et anormaux ou on fait des implementation
      en userland),
    \item Les cartes  et les périphériques réseaux (à  l'heure du cloud et
      des environnement distribué, ca peut avoir son importance)
    \item  Les carte  video n'apparaissent  pas toujours.  Certains driver
      sont implémentés en userspace
    \end{itemize}
  \end{itemize}
  Il s'agit  d'un standard, pour les systèmes  spécialisé, beaucoup de
  ces répertoire n'existeront pas.

  \emph{debootstrap(1)} permet de crée une nouvelle arboresence et d'y
  décompresser   les   fichiers   minimum  au   fonctionnement   d'une
  distribution Debian

  Lorsque  vous  installez  (ou   compilez)  un  paquet,  vous  pouvez
  spécifier de l'installer à partir d'une autre racine.

  Il  est possible  de s'éxecuter  sur une  autre racine  à  l'aide de
  \emph{chroot(1)}   ->   Peut-etre    à   mettre   dans   la   partie
  virtualisation?
\end{frame}

\subsection{Les filesystèmes}

\begin{frame}[fragile=singleslide]{Les filesystèmes}
  \begin{itemize}
  \item Une  partition de disque est \emph{montée}  sur un \emph{point
      de montage}
  \item Un point de montage est un répertoire (vide de préférence)
  \item \emph{mount(1)} \emph{mount(2)} \emph{mount(8)}
  \item Il existe diffŕent type de file systemes: vfat, ntfs, iso9660,
    ext2, ext3, ext4, xfs, btrfs, reiserfs, cramfs, squashfs, jffs2
  \item  Il est  possible de  monter  un fichier  normal plutot  qu'un
    fichier périphérique avec \c{-o loop} (une image disque ou une iso
    par exemple)
  \item \c{tmpfs} est mappé de sur de la mémoire RAM
  \item  \c{procfs}  (monté sur  \c{/proc})  et  \c{sysfs} (monté  sur
    \c{/sys}) sont des file systems virtuels
  \item Il permettent d'obtenir des informations sur l'état du noyau.
  \item Filesystem réseau: nfs, samba
  \item Filesystems plus complexes, implémneté en userland par FUSE (à
    l'aide de \c{/dev/fuse}): sshfs, ftpfs, etc..
  \end{itemize}
\end{frame}

\subsection{Implémentations}

\begin{frame}[fragile=singleslide]{table des patition}
  \begin{itemize}
  \item Inventée par Microsoft à peu près en même temps que la FAT
  \item  CHS (Cylindre-Head-Sector)  est  obsolete, de  nos jours,  on
    accede au disque en  tuilisant leur LBA (Logical Block Addressing)
    (les disque IDE doient implementer cette compatibilité, je ne dois
    pas que ca soit encore present dans le protocole sata )
  \item Les 512 premiers octets représente la table
  \item Les 446 premier  octet contiennent le \emph{Bootcode} (un boot
    loader s'appuyant sur le bios encore utilisé de nos jours)
    %% Ajouter figure
  \item  Ensuite un tableau  de 4  entré de  16 octets  reṕrésention 4
    partitions principales
  \item Chaque entré  contient le type, l'addresse (LBA)  de départ et
    la taille de la partition (répété dans le filesystème)
  \end{itemize}
\end{frame}

\begin{frame}[fragile=singleslide]{Vfat}
  \begin{itemize}
  \item Accronyme de File Allocation Table
  \item Inventé par Microsoft en 1977
  \item Disque divisé en cluster de  $512 * 2^n$ octets (en fait $2^n$
    secteurs de 512 octets)
  \item  La structure  volume ID  est  placée à  l'adresse zero,  elle
    contient la taille des cluster, l'addresse du répertoire racine et
    la taille de la FAT (nous  verrons que la taille de la partition =
    taille de fat en double mots * taille du cluster en octets)
  \item Les répertoires
    \begin{itemize}
    \item Un cluster répertoire  est un vecteur de structures pointant
      sur des fichiers ou d'autres répertoire.
    \item Les entrée contiennent un plus des information comme le nom,
      la taille et les attribut des fichiers
    \item Il  existe des valeurs  spéciales pour indiquer  les entrées
      supprimées et la fin du listing
    \item On peut ainsi parcourir les fichier du disque
    \end{itemize}
  \item Et si le fichier ou le répertoire ne tient pas sur un cluster?
    \begin{itemize}
    \item La FAT intervient
    \item La FAT  est un tableau de d'entiers de  32bits (ou 16bits ou
      12 bits)
    \item Chaque entrée représent un cluster (entrée 1 == cluster 1)
    \item  Si  l'entrée  vaut   0,  le  cluster  est  disponible  pour
      l'allocation
    \item Si  l'entrée vaut  un nombre, le  cluster est occuppé  et la
      valeur de l'entrée pointe sur le cluster suivant
    \item Si l'entrée vaut 0xFFFFFFFF,  le cluster est occupé et c'est
      le dernier cluster du fichier ou du répertoire
    \end{itemize}
  \item Les défauts:
    \begin{itemize}
    \item  La  fragmentation,  particulièrement  vraie si  un  fichier
      grossi progressivement
    \item Le temps d'accès entre la  FAT et les données et l'entrée du
      répertoire (pour modifier la taille)
    \item  La liste simplement  chainée dans  la FAT  qui oblige  à la
      parcourir pout accer à la fin du fichier
    \item Pas de pointeur vers le répertoire parent
    \end{itemize}
  \end{itemize}
  cf. \url{http://www.pjrc.com/tech/8051/ide/fat32.html}
\end{frame}

\begin{frame}[fragile=singleslide]{ext2}
  Inventé par Rémy Card au LIP6
  \begin{itemize}
  \item Deux type de structures principales:
    \begin{itemize}
    \item Des inodes (index-node) (128bits)
    \item Des blocks (paramètrable de l'ordre de 4Ko de nos jours)
    \end{itemize}
  \item Les disque  est divisé en \emph{block group}  (une centaine de
    Mo)
  \item Le système de block group permet de rapprocher les données des
    index
  \item  ... et  de  plus facilement  récupérer  le disque  en cas  de
    problème
  \item Le premier  block groupe se trouve à  l'adresse 1024 (l'espace
    avant est réservé pour un éventuel bootloader)
  \item Chaque \emph{block group} contient
    \begin{itemize}
    \item    Une    copie    du    superblock   (sauf    si    l'opton
      \emph{sparce\_block} est active)
    \item Des donnée concernant  ce block (taille de l'espace d'inode,
      taille  de l'espace  de block,  nombre de  block/inode utilisée,
      etc...)
    \item Un espace (fixe) d'inode
    \item Un espace (fixe) de block
    \item Un pointeur vers un block  de bitmap des inode qui permet de
      connaitre les inodes alloués
    \item Une pointeur vers un block de bitmap des block qui permet de
      connaitre les blocks alloués
    \end{itemize}
  \item  Remarque: Le bitmap  de block  doit tenir  sur un  block, par
    conséquent, il ne  peut y avoir plus de  taille\_de\_bloc * 8 blocs
    par groupe  de bloc. (ausi vrai  dans une moindre  mesure pour les
    inodes). C'est  généralement ca qui va déterminier  la taille d'un
    groupe.
  \item Remarque:  En connaissant, la  taille des block group,  le nom
    d'ino  par block group  et l'offset  de la  table d'inode  dans un
    group, il  est possible d'indexer n'importe quel  inode (idem pour
    les blocks)
  \item Chaque fichier est associé à un inode
  \item Un inode contient:
    \begin{itemize}
    \item Des informations comme  la taille, les date de modification,
      des création et d'accès, le type, etc...
    \item Dans  le cas fichier  spéciaux, toutes les  information sont
      contenus dans l'inode
    \item Un tableau de 12 pointeur vers les blocks contenant le corps
      du fichier (permet d'indexer les 50 premiers Ko)
    \item Un  pointeur vers un  block contenant des pointeurs  vers le
      coprs du fichiers (permet d'indexer les 4Mo suivants)
    \item un pointeur vers un block de second niveau (qui contient des
      pointeurs de pointeurs) (permet d'indexer les 4Go suivant)
    \item  un  pointeur vers  un  block  de  troisième niveau  (permet
      d'indexer les 4To suivant)
      % \item  Le jours  ou ont aura  besoin d'indexer des  fichier de
      %   4Po, on ajoutera  une inférence (limite théorique de l'ordre
      %   de 10^40)
    \item Plus le fichier est  gros, plus le nombre d'indirection sera
      important
    \item  Le système  alloue en  priorité les  fichiers dans  le même
      block group que son inode (limite la fragmentation)
    \end{itemize}
  \item Un fichier peut-être un répertoire
    \begin{itemize}
    \item Il  contient un tableau  de structures contenant  l'inode du
      fichier, la taille du nom et le nom de l'entrée
    \item Un répertoire contient  systématiquement une entrée \c{.} et
      une entrée \c{..}
    \item Une entrée dont l'inode est a zero à été supprimée
    \item Le parcours  des réperoitre se fait en O(n),  un index à été
      ajouté sur ext3 pour le faire en $O(log_2(n))$
    \end{itemize}
  \item Remarque: les  première inodes du disque sont  réservée pour :
    la liste des badblocks, le répertoire racine, le journal, etc...
  \item  cf.   \emph{tune2fs(1)},  \emph{debugfs(1)},  \emph{stat(1)},
    \url{http://www.nongnu.org/ext2-doc/ext2.html}
  \end{itemize}
  % Presnter le fonctionnement de fsck.ext2 (c'est interressant)
  % Exo: cacher des fichier dans de l'ext2 (il faut trouver de l'espace, ecrire, et activer la bitmap)
\end{frame}


\begin{frame}[fragile=singleslide]{Journalisation}
  \begin{itemize}
  \item Provient des technoloies des bases de donnée
  \item Permet de garantir l'atomicité des opération sur le disque
  \item On  écrit d'abord  dans le  journal ce que  l'on se  prépare à
    faire
  \item  Une fois  l'action écrite,  on  écrit ensuite  un message  de
    \emph{commit}
  \item Si le système plante, il  lit le journal, si une operation est
    associée à un commit, il execute l'operation, sinon, on estime que
    les information  concernant l'operation sont  peut-être erronée et
    on execute pas l'action
  \end{itemize}
  Evidement, il faut de temps en temps écrire réelement les donnée sur
  le disque.
  \begin{itemize}
  \item On commence par écrire les données
  \item On s'assure qu'elle ont été correctemnt écrite
  \item  Même si  le système  plante à  ce moment,  on  pourra rejouer
    l'entrée du journal
  \item on supprime l'entrée du journal
  \item Pour des raisons de performance, les entrée du journal ne sont
    pas forcement  écrite dans l'ordre. Il faut  alors faire attention
    aux contrainte de précedances
  \end{itemize}
  \begin{lstlisting}
$ debugfs -R logdump /dev/sda8
  \end{lstlisting}
  \begin{itemize}
  \item Il  est possible de  journaliser toutes les données  écrite ou
    seulement les meta donnée
  \item Dans le  second cas, on garanti que  filesystème sera cohérent
    mais pas les donnée à l'intérieur du fichier
  \item Linux implémente aussi  un more \emph{ordered} qui garanti que
    les donnée sont  mise à jour sur le disque avant  de mettre à jour
    les meta  donnée. Ainsi,  il est possible  de perdre  des données,
    mais pas d'avoir des donnée corrompues.
  \end{itemize}
\end{frame}

\begin{frame}[fragile=singleslide]{Copy On Write}
  \begin{itemize}
  \item Certain file système sont dit Copy-On-Write
  \item Il sont montés read-only
  \item Si  un fichier dit  être modifé, une  copie est faite  (sur un
    autre espace ou en mémoire) et la copie est modifiée.
  \item Les futur accès ce feront sur la copie
  \item Permet de faire démarrer des  système sur CD (avec un plus une
    compression dans ce cas)
  \item Permet  de faire fonctionner plusieurs système  sur une unique
    partition montée en lecture seule (virtualisation)
  \item Permet de démarer des  machine en réseau avec un unique dsique
    partagé
  \end{itemize}
\end{frame}
