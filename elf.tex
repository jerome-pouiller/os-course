%
% This document is available under the Creative Commons Attribution-ShareAlike
% License; additional terms may apply. See
%   * http://creativecommons.org/licenses/by-sa/3.0/
%   * http://creativecommons.org/licenses/by-sa/3.0/legalcode
%
% Created: 2012-07-28 20:09:12+02:00
% Main authors:
%     - Jérôme Pouiller <jezz@sysmic.org>
%

\part{La création d'executables}

\begin{frame}
  \partpage
\end{frame}

\begin{frame}
  \tableofcontents[currentpart]
\end{frame}

%% A reformuler, revoir
\section{Compilation}

\begin{frame}[fragile=singleslide]{Compilation}
  Compilation normale:
  \begin{lstlisting}
host$ gcc -c hello.c -o hello.o
host$ gcc hello.o -o hello
host$ ./hello 1
Hello World
  \end{lstlisting} %$
\end{frame}

\subsection{Fonctionnement de la compilation}

\begin{frame}[fragile=singleslide]{Le format ELF}
  \begin{itemize}
  \item La  plupart des fichiers  manipulés par le compilateur  et les
    outils associés sont au format ELF
  \item Ce format est une série de sections et de tables
  \item  Certaines sections seront chargée en mémoires
  \item  Certaines  section  demande  à etre  simplement  allouées  en
    mémoire
  \item  \man{objdump(1)} permet  d'obtenir des  informations  que un
    fichier ELF
  \item  \man{objcopy(1)}  permet   d'extraire  des  sections  ou  de
    modifier un fichier au format ELF
  \item \man{readelf(1)}  et \man{nm(1)}, mais  \cmd{objdump} contient
    toute leurs fonctionnalités
  \end{itemize}
\end{frame}

\begin{frame}[fragile=singleslide]{La compilation}
  La compilation:
  \begin{itemize}
  \item Execute le préprocesseur (fichier \file{.i}), puis le compilateur
    vérifie  la syntaxe,  le typage,  converti le  code  en assembleur
    (fichier \file{.s}) puis en code objet (fichier \file{.o})
  \item L'option \file{-c} est utilisé.
  \item Le compilateur doit connaitre les signature des fonction (afin
    de vérifier correctement le typage).
  \item Les  fichiers headers  (fichiers \file{.h}) sont  nécessaires pour
    cette phase
  \item  \file{-I} permet  de  spécifier des  chemins supplémentaires  où
    rechercher des fichiers headers (par défaut: \file{/usr/include})
  \item Si un fichier est  inclu entre double-quotes, il est recherché
    dans le même répertoire que le source.
  \item \cmd{-Wall}, \cmd{-Wextra} recommandés
  \item \cmd{-DMACRO} peut etre utilisé
  \end{itemize}
\end{frame}

\begin{frame}[fragile=singleslide]{Les symboles de debug}
  \begin{itemize}
  \item \cmd{-g} permet d'ajouter  une section contenant des symboles de
    debug.
  \item  Utilisé  spar les  debuggueur  pour  donner des  informations
    supplémentaire ou pour faire le lien avec les sources.
  \item Cette section n'est pas chargée en mémoire.
  \item Le format utilisé s'apeller \emph{dwarf} (Debug with Arbitrary
    Record Format).
  \item L'option \cmd{-g} ne change pas le code généré:
    \begin{lstlisting}
$ gcc -g -c main.c -o main-dbg.o
$ gcc -c main.c -o main-rel.o
$ ls -l
-rw-rw-r-- jpo jpo 2125 Aug 3 16:05 main-rel.o
-rw-rw-r-- jpo jpo 3720 Aug 3 16:05 main-dbg.o
$ strip *.o
$ ls -l
-rw-rw-r-- jpo jpo 2096 Aug 3 16:05 main-rel.o
-rw-rw-r-- jpo jpo 2096 Aug 3 16:05 main-dbg.o
    \end{lstlisting}
  \end{itemize}
\end{frame}

\subsection{Edition de liens}

\begin{frame}[fragile=singleslide]{Inlining}
  Le compilateur peut effectuer quelques optimisations:
  \begin{itemize}
  \item Les  fonction et les  variables marquées \c{static}  ne seront
    pas exportés, et donc, pas utilisé par les autres fichier objets.
  \item  Le compilateur peut décider d'\emph{inliner} ces fonctions
  \item Si toutes les appels à une fonctions statique ont été inlinés,
    il peut supprimer la fonction du fichier objet.
  \end{itemize}
\end{frame}

\begin{frame}[fragile=singleslide]{L'édition de liens}
  \begin{itemize}
  \item  Le compilateur  ne connait  pas forcement  les  addresses des
    fonctions et des variables globale marquées \c{extern}
  \item  Les endroits  ayant  besoin de  ces fonctions/variables  sont
    remplacés par  des 0 et une  entrée est ajoutée dans  la table des
    \emph{relocation} du fichier objet (cf. \cmd{objdump -R})
  \item On  appelle le \emph{linker} (\c{gcc} sans  l'option \c{-c}) pour
    résoudre les symboles
  \item Le linker connait toutes les fonctions, et toutes les tables de
    relocation.
  \item Il peut déplacer les  addresses de chargement des fonctions et
    des variables de  manière à les mettre au  plus près des fonctions
    qui  les   appellent  (optimisation  de   l'utilisation  du  cache
    d'instruction)
  \item Une  fois l'agencement des fonctions défini,  le linker résoud
    toutes les entrées des tables de relocations
  \end{itemize}
\end{frame}

\section{Les bibliothèques}

\begin{frame}[fragile=singleslide]{Les bibliothèques}
  \begin{itemize}
  \item  Afin de  simplifier  le déploiment  des  utilitaires, il  est
    possible  d'empaqueter  un  ensemble  de fichier  objet  dans  une
    bibliothèque dite statique (fichier \file{.a}). cf. \man{ar(1)}.
  \item Dans ce cas, cela ne change rien à la procedure de link
  \item Il  est aussi possible d'utiliser  des biliothèques dynamiques
    (fichier \file{.so}). Nous y reviendrons.
  \item  Il  est  possible  de  spécifier  le  chemin  complet  de  la
    bibliothèque (dynamique ou statique)  ou de laisser le compilateur
    la trouver  automatiquement avec  la syntaxe \cmd{-lbrary}.  On peut
    dans ce cas lui précisier des chemins supplémentaire ou rechercher
    la bibliothèque avec \cmd{-L}
  \item  Par  défaut,  le  compilateur  recherchera  les  bibliothèques
    dynamiques, sauf si l'option \cmd{-static} est utilisée
  \item  Le  linker suit  en  fait  une  des recettes  contenues  dans
    \file{/usr/lib/ldscripts/}  (en fonction  des  options passées  au
    linker).   Il   est  possible  de  fabriquer   son  propre  format
    (\cmd{ld -T}) (utile pour générer des firmwares).
  \end{itemize}
\end{frame}

\begin{frame}[fragile=singleslide]{Libtool}
  Problème:
  \begin{itemize}
  \item Les  fichiers objets des bibliothèques  statiques et dynamique
    ne se compilent pas avec les mêmes options
  \item  Certaines architectures ne  permettent pas  les bibliothèques
    dynamiques
  \item La  création de bibliothèques  portables peut devenir  un vrai
    casse tête
  \end{itemize}
  \emph{libtool} est outil permettant de faciliter ce travail.
\end{frame}

\subsection{Les bibliothèques dynamiques}

\begin{frame}[fragile=singleslide]{Les bibliothèques dynamiques}
  L'usage de bibliothèques dynamique permet:
  \begin{itemize}
  \item de ne charger qu'un exemplaire de la bibliothèque pour tout le
    système
  \item de simplifier la redistribution du programme
  \item de simplifier les mises à jour de la bibliothèque
  \end{itemize}
\end{frame}

\begin{frame}[fragile=singleslide]{Les bibliothèques dynamiques}
  Coté bibliothèque:
  \begin{itemize}
  \item  Elle contient  une table  des  symboles exportés  et de  leur
    emplacement (dans la table \emph{.dynsym})
  \item A  la construction, elle  doit être linkée  avec \cmd{-shared}
    pour indiquer  que sont  chargement sera différent  d'un programme
    standard (principalement, aucune fonction main n'est nécessaire)
  \end{itemize}
\end{frame}

\begin{frame}[fragile=singleslide]{Les bibliothèques dynamiques}
  Coté éxecutable:
  \begin{itemize}
  \item Le  linker va ajouter une table  d'indirection (la \emph{.got}
    \emph{Global  Offset Table})  pour tous  les symboles  se trouvant
    dans des bibliothèques
  \item Le linker doit passer  par cette indirection pour appeller une
    fonction exportée par une biliothèque dynamique.
  \item Il ne peut pas faire  tenir ce morceau de code à l'emplacement
    laissé par le compilateur (qui  ne fait pas la différence entre un
    symbole statique et un symbole dynamique)
  \item Il crée  donc un petit morceau de code  qu'il placera dans une
    section \emph{Procedure Linkage Table} (\emph{.plt})
  \item Cette  procedure permet simplement  de brancher vers  la table
    d'indirection.
  \end{itemize}
\end{frame}

\begin{frame}[fragile=singleslide]{Les bibliothèques dynamiques}
  \begin{center}
   \pgfimage[width=7cm]{pics/plt_after}
\end{center}
\begin{lstlisting}
$ objdump -d | grep -A 10 "section \.plt"
    \end{lstlisting}
\end{frame}

\begin{frame}[fragile=singleslide]{Les bibliothèques dynamiques}
  A l'éxécution
  \begin{itemize}
  \item Un interpreteur (normalement \file{/lib/ld.so}) est appellé
  \item  Il charge  les bibliothèques  nécessaires (inscrites  dans la
    table \emph{.dynamic}) en mémoire
  \item  Il  remplit la  GOT  avec  des  pointeurs vers  une  fonction
    permettant la résolution du symbole.
  \item Lorsque ce symbol est appellé la première fois, cette fonction
    est appellée.
  \item La fonction résoud le symbol et place son adresse dans la GOT
  \item Cette méthode s'appelle \emph{lazy resolving}
  \item   Cela   peut   être   désactivé  en   passant   la   variable
    d'environnement \cmd{LD_BIND_NOW}
  \end{itemize}
\end{frame}

\begin{frame}[fragile=singleslide]{Les bibliothèques dynamiques}
\begin{center}
   \pgfimage[width=7cm]{pics/plt_before}
\end{center}
\end{frame}

\begin{frame}[fragile=singleslide]{Les bibliothèques dynamiques}
  \begin{itemize}
  \item Il est possible de  demander le chargement manuel et dynamique
    des    bibliothèques    et    des    symboles    avec    \c{libdl}
    (\man{dl\_open(3)}, \man{dl\_sym(3)})
  \item Afin d'accélérer le  chargement, \file{ld.so} utilise un index de
    bibliothèques présentes  sur le  système.
  \item  Lorsque vous  ajoutez une  bibliothèque, vous  devez appeller
    \man{ldconfig(1)} pour mettre à jour ce cache.
  \item la variable \cmd{LD_LIBRARY_PATH}  permet d'ajouter un chemin de
    recherche de biliothèque pour \file{ld.so}.
  \end{itemize}
\end{frame}

\begin{frame}[fragile=singleslide]{Utilisation de \c{LD_PRELOAD}}
  \begin{itemize}
  \item La  variable d'envionnement \c{LD_PRELOAD}  permet de demander
    le chargement d'une bibliothèque avant les autres
  \item Les symboles exportées par celle-ci seront prioritaires.
  \item Exemple:
    \begin{lstlisting}
unsigned int sleep(unsigned int s) {
    static unsigned int (*real_sleep)(unsigned int s) = NULL;

    if (!real_sleep)
        real_sleep = dlsym(RTLD_NEXT, ``sleep'');

    usleep(5);
    return real_sleep(s);
}
    \end{lstlisting}
  \end{itemize}
\end{frame}

\section{Résultats de la compilation}
% Section: Redistribution? A deplacer apres les outils de compilation?

\begin{frame}[fragile=singleslide]{A qui fournir quoi?}
  \begin{itemize}
  \item La poubelle
    \begin{itemize}
    \item Les dependances
    \item Les objets (ELF)
    \end{itemize}
  \item Les sources
    \begin{itemize}
    \item Les sources
    \item Les headers
    \item Le système de compilation
    \end{itemize}
  \item Les developpeurs externes
    \begin{itemize}
    \item Les headers
    \item Les bibliothèque statique (archve crée avec \man{ar(2)})
    \end{itemize}
  \item L'utilisateur final
    \begin{itemize}
    \item Les bibliothèque dynamique (ELF)
    \item Les binaires (ELF)
    \end{itemize}
  \end{itemize}
\end{frame}

\begin{frame}[fragile=singleslide]{Identifier le résultat}
  Un bon moyen de reconnaitre  les binaires est d'utiliser la commande
  \man{file(1)}:
  \begin{lstlisting}
host$ file ...
hello-dyn:        ELF 64-bit LSB executable, x86-64, version 1 (SYSV), dynamically linked (uses shared libs), for GNU/Linux 2.6.15, not stripped
hello-static: ELF 64-bit LSB executable, x86-64, version 1 (GNU/Linux), statically linked, for GNU/Linux 2.6.15, not stripped
/lib/x86_64-linux-gnu/librt-2.15.so: ELF 64-bit LSB shared object, x86-64, version 1 (SYSV), dynamically linked (uses shared libs), for GNU/Linux 2.6.24, stripped
/usr/lib/x86_64-linux-gnu/librt.a: current ar archive
\end{lstlisting} %$
\end{frame}

\begin{frame}[fragile=singleslide]{Redistribution et licences}
  \begin{itemize}
  \item GPL. Est-ce du travail dérivé?
    \begin{itemize}
    \item Le resultat d'un copier-coller?
    \item Une compilation statique?
    \item Une compilation dynamique?
    \end{itemize}
  \item LGPL
  \end{itemize}
\end{frame}

\section{Makefiles}
% A étoffer?

\begin{frame}[fragile=singleslide]{Makefile}
  \begin{itemize} 
  \item Vieux système (qui a dit ``pourri''?)
  \item Beaucoup d'implémentations différentes
  \item Heureusement,  gmake (GNU Make) est quasiment  le seul utilisé
    de nos jours.
  \item  \cmd{make}  est finalement  un  interpréteur  qui éxecute  le
    fichier  nommé  \file{Makefile}  ce  trouvant dans  le  répertoire
    courant
  \item  Il existe  de nombreuses  alternatives au  Makefile,  mais ce
    dernier  a imposé des  standards qui  se retrouvent  dans beaucoup
    d'autres systèmes (nous en verrons quelques uns plus tard)
  \end{itemize} 
\end{frame} 

\begin{frame}[fragile=singleslide]{Bases}
  \begin{itemize} 
  \item Principe du fichier \file{Makefile}:
    \begin{lstlisting} 
cible: depends
       rules
file.o: file.c dep.h
       gcc file.c -c -o file.o
    \end{lstlisting} 
  \item \cmd{make} crée  un graphe de dépendance et  vérifie les dates
    de modifications  des fichierx afin  de n'éxecuter que  les règles
    nécessaires
\item Notez que les dépendances avec les fichiers \file{.h} doivent être déclarées
  \item Un \file{Makefile} bien écrit ne reconstruit que le nécessaire
    et n'oublie jamais de reconstruire le nécessaire.
  \item  Il  est  possible  spécifier  à  make une  ou  des  cibles  à
    fabriquer. Sinon, il fabriquera la première cible seulement.
  \end{itemize}
\end{frame}

\begin{frame}[fragile=singleslide]{Les variables}
  \begin{itemize} 
  \item Il est possible d'utiliser des variables
    \begin{lstlisting}
CC = gcc
...
        $(CC) file.c -c -o file.o
    \end{lstlisting} %$
  \item Remarquez que la syntaxe des variable est différente du shell.
  \item ... or, nous appellons du shell à partir des \file{Makefile}
  \item  Les  variables  peuvent  être  surchargées sur  la  ligne  de
    commande:
    \begin{lstlisting}
$ make CC=special-gcc
    \end{lstlisting} %$
  \item  Il  existe de  nombreuses  variables standards,  prédéfinies:
    \c{CC}, \c{CFLAGS},  \c{LDFLAGS}, \c{CXX}, \c{CXXFLAGS}, \c{MAKE},
    ...
  \item  Il existe  aussi des  variables locales  aux  règles: \c{$<},
    \c{$@}, ....
  \end{itemize} 
\end{frame}

\begin{frame}[fragile=singleslide]{Makefile}
  \begin{itemize} 
  \item  Il  est  possible  de définir  des  règles
    génériques:
    \begin{lstlisting} 
%.o: %.c
       gcc $< -c -o $@
file.o: dep.h
    \end{lstlisting} 
  \item De plus, il existe de nombreuses règles implicite qui facilite
    le travail
  \item  Utilisez  au maximum  les  règles  implicites facilite  votre
    travail:
    \begin{lstlisting}
host$ vi toto.c
host$ touch Makefile
host$ make toto
    \end{lstlisting} %$
    \item Un bon Makefile est un Makefile court
  \end{itemize}
\end{frame}

\begin{frame}[fragile=singleslide]{Les règles \emph{phony}}
  \begin{itemize} 
  \item Il est possible de définir des règles \emph{virtuelles} (\emph{phony}) pour simplifier certains traitements
    \begin{lstlisting}
all: exec1 exec2
    \end{lstlisting} 
  \item  Parmis  les règles  \emph{phony}  très répandues:  \c{clean},
    \c{distclean},   \c{install},    \c{all},   \c{check},   \c{dist},
    \c{distcheck}, \c{mrproper}...
  \end{itemize} 
\end{frame}
