%
% This document is available under the Creative Commons Attribution-ShareAlike
% License; additional terms may apply. See
%   * http://creativecommons.org/licenses/by-sa/3.0/
%   * http://creativecommons.org/licenses/by-sa/3.0/legalcode
%
% Created: 2012-07-28 20:09:12+02:00
% Main authors:
%     - Jérôme Pouiller <jezz@sysmic.org>
%

\part{Systèmes de compilation}

{
\setbeamertemplate{background canvas}{}
\begin{frame}[plain]
  \partpage
  \begin{textblock}{10}(6,12)
    \begin{quote}
      \rmfamily\textit\textbf\color{darkgray}{\large
      ``Real Programmers don't comment their code. If it was hard to write, it should be hard to understand.''}
      %\vskip3mm\hspace*\fill{\small--- William Shakespeare, Hamlet}
    \end{quote}
  \end{textblock}
\end{frame}
}

\begin{frame}
  \tableofcontents
\end{frame}

% (Re-ajouter l'utilisation des systeme de compilations)


% A placer au debut de la section compilation
\begin{frame}[fragile=singleslide]{Règle d'or}
  \begin{center}
    \huge{Jamais d'espaces dans les chemins de compilation}
  \end{center}
\end{frame}

\subsection{Un projet base de Makefile}

\begin{frame}[fragile=singleslide]{Qu'est-ce qu'un cross-compiler?}
  \begin{itemize}
  \item Permet de compiler vers une cible différente du host
  \item Les binaires produites ne sont pas éxecutable sur le host
  \item La cible de la compilation est généralement décrites dans le nom
  \end{itemize}
  \begin{lstlisting}
$ /opt/arm-sysmic-linux-uclibcgnueabi/usr/bin/arm-buildroot-linux-uclibcgnueabi-gcc source.c
$ PATH+=:/opt/arm-sysmic-linux-uclibcgnueabi/usr/bin/
$ arm-buildroot-linux-uclibcgnueabi-gcc source.c
$ file a.out
a.out: ELF 32-bit LSB executable, ARM, version 1 (SYSV), dynamically linked (uses shared libs), not stripped
  \end{lstlisting}
\end{frame}

\begin{frame}[fragile=singleslide]{Cas générique}
  On utilise les variables prédéfinies de gmake:
  \begin{lstlisting}
make CC=arm-linux-gcc
  \end{lstlisting}
  Mieux dans un sous répertoire séparé:
  \begin{lstlisting}
mkdir arm
cd arm
make -f ../Makefile VPATH=.. CC=arm-linux-gcc
  \end{lstlisting}
  Exemple avec memstat.
\end{frame}

\begin{frame}[fragile=singleslide]{Créer un projet sans autotools}
  Utiliser au maximum les règles implicites facilite votre travail
  \begin{lstlisting}
host$ touch Makefile
host$ make hello
  \end{lstlisting} %$
\end{frame}

\begin{frame}[fragile=singleslide]{Créer un projet sans autotools}
  Utiliser les règles implicites facilite votre travail
  \lstinputlisting{samples/hello/Makefile.1}
  Testons:
\begin{lstlisting}
host$ make CC=arm-linux-gcc CFLAGS=-Wall
\end{lstlisting} %$
\end{frame}

\begin{frame}[fragile=singleslide]{Créer un projet sans autotools}
  \cmd{VPATH} vous permet de géré la compilation \emph{out-of-source}.
  Remarques que pour que \verb+VPATH+ fonctionne correctement, vous devez avoir
  correctement utilisé le quoting pour les directive d'inclusion (\verb+<+ pour
  les entête systèmes et \verb+"+ pour les entêtes du projet)
  % Pas besoin d'ajouter VPATH = . Du coup, c'est la meme chose que Makefile.1
  %\lstinputlisting{samples/hello/Makefile.2}
  Testons:
\begin{lstlisting}
host$ cd build
host$ make -f ../Makefile VPATH=.. CC=arm-linux-gcc
\end{lstlisting} %$
\end{frame}

\begin{frame}[fragile=singleslide]{Créer un projet sans autotools}
  \cmd{gcc} peut  générer les dépendances de vos  fichiers.  On génère
  ainsi  des morceaux  de Makefile  que l'on  inclut. Il  ne  faut pas
  oublier   d'ajouter  la  dépendance   entre  \cmd{hello.d}   et  les
  dépendances de \cmd{hello.c}
  \lstinputlisting{samples/hello/Makefile.3}
  \note{ Voir http://www.makelinux.net/make3/make3-CHP-2-SECT-7.html}
\end{frame}

\begin{frame}[fragile=singleslide]{Créer un projet sans autotools}
  Les  Makefile permettent d'utiliser  des fonctions  de substitutions
  qui  peuvent  nous aider  à  rendre  notre  système plus  générique.
  \lstinputlisting{samples/hello/Makefile.4}
\end{frame}

\begin{frame}[fragile=singleslide]{Créer un projet sans autotools}
  Nous pouvons  ajouter des alias  pour nous aider dans  les commandes
  complexes
  \lstinputlisting[firstline=10]{samples/hello/Makefile.5}
  \note[item]{TODO: Ajouter des \#ifdef CONFIG pour faire dans Kconfig}
  \note[item]{Parler de apt-get source (-b), dpkg -L, dpkg -l, dpkg-buildpackage}
\end{frame}

\subsection{A base d'Autotools}

\begin{frame}[fragile=singleslide]{Historique des Autotools}
  \note[item]{Faire l'historique de configure/Makefile/autotools}
  \begin{enumerate}
  \item Makefile
  \item Makefile + hacks pour effectuer de la configuration
  \item Makefile.in + configure
  \item Makefile.in + configure.ac
  \item Makefile.am + configure.ac
  \end{enumerate}
\end{frame}

\begin{frame}[fragile=singleslide]{Compiler avec autotools}
  \begin{itemize}
  \item C'est le cas le plus courant
  \item Pour une compilation classique:
\begin{lstlisting}
host$ ./configure
host$ make
host% make install
\end{lstlisting} %$
  \item Exemple avec dropbear.
  \item Compilation \emph{out-of-source}. il est nécessaire d'appeller
    le \file{configure} à partir du répertoire de build.
    \begin{lstlisting}
host$ mkdir build
host$ cd build
host$ ../configure
host$ make
host% make install
    \end{lstlisting} %$
  \end{itemize}
\end{frame}

\begin{frame}[fragile=singleslide]{Compiler avec autotools}
  Obtenir de l'aide:
\begin{lstlisting}
host$ ./configure --help
\end{lstlisting} %$

  Parmis les fichiers générés:
  \begin{itemize}
  \item \file{config.log}  contient la sortie  des opération effectuée
    lors de l'appel de \cmd{./configure}.  En particulier, il contient
    la ligne de commande utilisée. Il est ainsi possible de facilement
    dupliquer la configuration.
\begin{lstlisting}
host$ head config.log
\end{lstlisting} %$
  \item     \cmd{config.status}     permet     de    regénérer     les
    Makefile.  \cmd{config.status} est  automatiquement appellé  si un
    Makefile.am est modifié.
  \end{itemize}
  \note{Ajouter un exemple avec tcpdump ou dmalloc}
\end{frame}


\begin{frame}[fragile=singleslide]{Créer un projet avec autotools}
  Fonctionnement des autotools:
  \begin{itemize}
  \item Préparation
\begin{lstlisting}
% apt-get install automake autoconf
\end{lstlisting}
  \item Déclaration de notre programme et de nos sources pour \cmd{automake}
\begin{lstlisting}
$ vim Makefile.am
\end{lstlisting} %$
\begin{lstlisting}
bin_PROGRAMS = hello
hello_SOURCES = hello.c hello.h
\end{lstlisting}
%  \item Les \cmd{autotools}  imposent l'existence de certains fichiers
%    de documentation
%\begin{lstlisting}
%$ touch NEWS README AUTHORS ChangeLog
%\end{lstlisting} %$
  \end{itemize}
\end{frame}

\begin{frame}[fragile=singleslide]{Créer un projet avec autotools}
  \begin{itemize}
  \item  Création  d'un  template  pour \cmd{autoconf}  contenant  les
    macros utiles pour notre projet
\begin{lstlisting}
$ autoscan
$ mv configure.scan configure.ac
$ rm autoscan.log
$ vim configure.ac
\end{lstlisting}
  \item Personnalisation du résultat
\begin{lstlisting}
...
AC_INIT([hello], [1.0], [bug@sysmic.org])
AM_INIT_AUTOMAKE([foreign])
...
\end{lstlisting}
  \item      Génération      du      \file{configure}      et      des
    \file{Makefile.in}. C'est cette version qui devrait être livée aux
    packageurs.
\begin{lstlisting}
$ autoreconf -iv
\end{lstlisting} %$
    \note[item]{Bon, pas de TP la dessus, ca pas très utile}
  \end{itemize}
\end{frame}

\begin{frame}[fragile=singleslide]{Créer un projet avec autotools}
  \begin{itemize}
  \item Compilation
\begin{lstlisting}
$ ./configure --help
$ mkdir build
$ cd build
$ ../configure --host=arm-linux --build=i386 --prefix=$(pwd)/../install
$ make
$ make install
\end{lstlisting} %$
  \end{itemize}
\end{frame}

\begin{frame}[fragile=singleslide]{Créer un projet avec autotools}
  La cible \verb+distcheck+ :
  \begin{enumerate}
  \item Recopie les fichiers référencé dans Autotools
  \item Retire les droits en écriture sur les sources
  \item Lance une compilation \emph{out-of-source}
  \item Installe le projet
  \item Lance la suite de test
  \item Lance un distclean
  \item Vérifie que tous les fichiers créés sont effectivement supprimés
  \item Crée une tarball correctement nommée contenant les sources
  \end{enumerate}
\end{frame}

\begin{frame}[fragile=singleslide]{Créer un projet avec autotools}
  Si \cmd{automake}  est appellé avec  \verb+-gnits+, \verb+distcheck+
  effectue des vérification supplémentaires sur la documentation,
  etc...
  \\[2ex]
  La fonctionnalité \verb+distcheck+ est  le point fort souvent énoncé
  des autotools.
\begin{lstlisting}
$ make distcheck
$ tar tvzf hello-1.0.tar.gz
\end{lstlisting} %$
\end{frame}

%%% A partir de la, je ne sais pas si je le fais

\subsection{A base kmake}

%% A placer apres le système de Makefile?
%% Peut-être ajouter léxemple d'eolane juste avant
\begin{frame}[fragile=singleslide]{Compiler un programme tiers}{Kconfig}
  \begin{itemize}
  \item Système de compilation du noyau
  \item Très bien adapté à la cross-compilation
  \item Pour configurer les options:
    \begin{itemize}
    \item En ncurses
\begin{lstlisting}
host% apt-get install libncurses5-dev
host$ make ARCH=arm CROSS_COMPILE=arm-linux- menuconfig
\end{lstlisting} %$
    \item En Qt3
\begin{lstlisting}
host% apt-get install libqt3-mt-dev
host$ make ARCH=arm CROSS_COMPILE=arm-linux- xconfig
\end{lstlisting} %$
    \end{itemize}
  \item Ne pas oublier d'installer les headers des bibliothèques
    \item Exemple avec busybox
  \end{itemize}
\end{frame}

\begin{frame}[fragile=singleslide]{Compiler un programme tiers}{Kconfig}
  \begin{itemize}
  \item Pour cross-compiler
\begin{lstlisting}
host$ make ARCH=arm CROSS_COMPILE=arm-linux-
\end{lstlisting} %$
  \item Pour compiler \emph{out-of-source}
\begin{lstlisting}
host$ mkdir build
host$ make ARCH=arm CROSS_COMPILE=arm-linux- O=build
\end{lstlisting} %$
  \end{itemize}
\end{frame}

\begin{frame}[fragile=singleslide]{Compiler un programme tiers}{Kconfig}
  Principaux points importants:
  \begin{itemize}
  \item Adapté au environnements embarqué
  \item Adapté aux environnements avec beaucoup de configuration
  \item Initié par le Kernel Linux
  \item  Pas un  système de  compilation réel. Composé de :
    \begin{itemize}
    \item Kconfig, Système de gestion de configuration
    \item  Kmake, règles  Makefile  bien étudiées.  Chaque projet  les
      adapte à ces besoins
    \end{itemize}
  \item Application de la règle: "Pas générique mais simple à hacker"
  \item Dépend principalement de \cmd{gmake}
  \item Mode verbose: \verb+V=1+
  \item Permet d'effectuer des recherche
  \end{itemize}
\end{frame}

\begin{frame}[fragile=singleslide]{Compiler un programme tiers}{Kconfig}
  Test avec busybox:
  \begin{itemize}
  \item Préparation
\begin{lstlisting}
host$ wget http://busybox.net/downloads/busybox-1.18.3.tar.bz2
host$ tar xvjf busybox-1.18.3.tar.bz2
\end{lstlisting} %$
  \item Récupération d'une configuration par défaut
\begin{lstlisting}
host$ make help
host$ make ARCH=arm CROSS_COMPILE=arm-linux- O=build defconfig
\end{lstlisting} %$
  \item Personnalisation de la configuration
\begin{lstlisting}
host% apt-get install libncurses5-dev
host$ make ARCH=arm CROSS_COMPILE=arm-linux- O=build menuconfig
\end{lstlisting} %$
  \end{itemize}
\end{frame}

\begin{frame}[fragile=singleslide]{Compiler un programme tiers}{Kconfig}
  Test avec busybox:
  \begin{itemize}
  \item Compilation
\begin{lstlisting}
host$ make ARCH=arm CROSS_COMPILE=arm-linux-
\end{lstlisting} %$
  \item Installation
\begin{lstlisting}
host$ make ARCH=arm CROSS_COMPILE=arm-linux- install
\end{lstlisting} %$
  \end{itemize}
\end{frame}

\subsection{A base de Cmake}

\begin{frame}[fragile=singleslide]{Cmake}
  \begin{itemize}
  \item Aucune des  solution précédentes ne fonctionne sur  les OS non
    posix (rappel: Cygwin = Couche posix pour Windows)
  \item Cmake ressemble à Automake + Autoconf
  \item Un fichier (CMakeLists.txt) décrit la compilation
    \begin{lstlisting}
cmake_minimum_required (VERSION 2.6)
project (HELLO)
add_executable (hello hello.c)
    \end{lstlisting}
  \item Exemple avec yajl
  \item Cmake génère des fichiers  pour les différent types de système
    de compilation: Makefile, XCode, projet VisualStudio, etc...
    \begin{lstlisting}
$ mkdir build
$ cd build
$ ccmake ..
$ make
    \end{lstlisting}
\end{itemize}
\end{frame}

\begin{frame}[fragile=singleslide]{Cmake}
  Points positifs:
  \begin{itemize}
  \item Portabilité
  \item Syntaxe cohérente (c'est loin d'être le cas de Makefile)
  \item Extensible
  \item  Son  abstration permet  une  prise  en  main facile  pour  un
    débutant
  \end{itemize}
  Sa portabilité rend le niveau d'abstraction de Cmake assez élevé:
  \begin{itemize}
  \item Peut dérouter les habitués
  \item Processus de compilation  complexe à debugguer (c'est aussi le
    cas d'Autotools)
  \item  Faible intégration  avec  les système  de compilation  natifs
    (contrairement à Autotools)
  \item  Certaine action  simple nativement  peuvent  devenir complexe
    dans Cmake
  \item  Nécessite d'installer  cmake  sur la  machine de  compilation
    (Contrairement à Autotools/Kmake)
  \end{itemize}
\end{frame}

\begin{frame}[fragile=singleslide]{Que choisir?}
  \begin{itemize}
  \item  Projet nécessitant  une  bonne integration  avec les  système
    Posix et avec les système Microsoft: Cmake
  \item  Petit  projet, avec  redistribution  restreinte: Makefile  ou
    CMake
  \item Petit projet, mais redistribution large: Autotools
  \item Gros projet de forte complexité: Kmake
  \end{itemize}
\end{frame}
