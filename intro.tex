%
% This document is available under the Creative Commons Attribution-ShareAlike
% License; additional terms may apply. See
%   * http://creativecommons.org/licenses/by-sa/3.0/
%   * http://creativecommons.org/licenses/by-sa/3.0/legalcode
%
% Created: 2012-07-28 10:50:36+02:00
% Main authors:
%     - Jérôme Pouiller <jezz@sysmic.org>
%

%\part{Introduction}

\begin{frame}[fragile=singleslide]{Programme}
  \begin{itemize}
  \item Utilisation des systèmes Unix
    \begin{itemize}
    \item Savoir utiliser l'environnement
    \item  Connaître  quelques  concepts  annexes  (file  descriptors,
      pattern matching, etc...)
    \item Rappeler les concepts de base d'un OS
    \end{itemize}
  \item Les systèmes de fichiers
    \begin{itemize}
      \item Organisation des fichiers
      \item Format des filesystems
    \end{itemize}
  \item La création d'exécutables
    \begin{itemize}
    \item La compilation
    \item Le chargement
    \end{itemize}
  \item La gestion des tâches
    \begin{itemize}
    \item Les systèmes monotâche
    \item Les systèmes multitâches
    \end{itemize}
  \item La gestion de la mémoire
    \begin{itemize}
    \item Les différents segments
    \item Les algorithmes de gestion de la mémoire
    \item Les techniques de debug
    \end{itemize}
  \item Les architectures d'OS
  \end{itemize}
\end{frame}
