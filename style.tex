%
% This document is available under the Creative Commons Attribution-ShareAlike
% License; additional terms may apply. See
%   * http://creativecommons.org/licenses/by-sa/3.0/
%   * http://creativecommons.org/licenses/by-sa/3.0/legalcode
%
% Copyright 2010 Jérôme Pouiller <jezz@sysmic.org>
%

% Configurationde de Beamer

\usetheme{Warsaw}

% Quiris
%\usecolortheme[RGB={238,127,0}]{structure}
% Apollo
%\usecolortheme[RGB={0,56,126}]{structure}
% Sysmic
\usecolortheme[RGB={181,0,0}]{structure}

% Headers et footers
\useoutertheme[subsection=false]{smoothbars}
% Rectangle dans les itemize
\useinnertheme{rectangles}
% Pas de symboles pour la navigation
\setbeamertemplate{navigation symbols}{}

% Logo en filigranne
\setbeamertemplate{background canvas}{
  \tikz {
    \node at (0,0) {};
    \node[inner sep=0pt, opacity=0.2] at (0.5\paperwidth,-0.8\paperheight) {\includegraphics[height=23.4mm,width=128mm]{pics/logo}};
  }
}

% Ajout des numéros de pages
\newcommand*\oldinsertshortitle{}%
\let\oldinsertshorttitle\insertshorttitle%
\renewcommand*\insertshorttitle{\oldinsertshorttitle\hfill\insertframenumber\,/\,\inserttotalframenumber}

\usepackage{ucs}              % La sortie est en UTF8
\usepackage[utf8x]{inputenc}  % L'entrée aussi
\usepackage[T1]{fontenc}      % Pour la césure des mots accentués
\usepackage{lmodern}          % Pour les lettres francaises
\usepackage[french]{babel}    % Pour la sortie francaise
\usepackage{times}
\usepackage{mathptmx}         % Font (don't forget to install latex-font-recommended)

\input{style_tikz}

\ifpdf
  \usepackage{embedfile} 
\else
  \newcommand{\embedfile}[1]{}
\fi

\usepackage[sections,displaymath]{preview}
\PreviewEnvironment{tikzpicture}
\PreviewEnvironment{center}
\PreviewEnvironment{frame}

\usepackage{hyperref} % Doit être chargé avant ntheorem
%\usepackage{realboxes} % Pour \Colorbox
\newcommand{\email}[1]{\href{mailto:#1}{\nolinkurl{<#1>}}}
\newcommand{\man}[1]{\emph{#1}}
\newcommand\file[1]{\lstinline[backgroundcolor=\color{{rgb}{1,1,0.8}},language=]{#1}}
\newcommand\cmd[1]{\lstinline[backgroundcolor=\color{{rgb}{1,1,0.8}},language=]{#1}}
\renewcommand\c[1]{\lstinline[backgroundcolor=\color{{rgb}{1,1,0.8}},language=c]{#1}}

\usepackage[newcommand]{ragged2e}

% \usepackage[]{version} 
% \usepackage{ntheorem} 
% \setlength{\theorempreskipamount}{0.8ex plus 0.9ex minus 0.1ex}
% \setlength{\theorempostskipamount}{0.8ex plus 0.9ex minus 0.1ex}
% \theoremprework{\vspace{3mm}\hrule}
% \theorempostwork{\hrule\vspace{3mm}}
% \theorembodyfont{\itshape}
% \theoremseparator{.}
% \newtheorem{quest}{Question}[section]

% \theorembodyfont{\normalfont}
% \theoremseparator{}
% \theoremstyle{break}
% \newtheorem*{ans}{Réponse}

% \setlength{\theoremindent}{3mm}
% \theoremseparator{:}
% \theorembodyfont{\itshape}
% \theoremstyle{plain}
% \newtheorem*{man}{Documentation utile}
% \theorembodyfont{\normalfont}
% \newtheorem*{hint}{Remarque}
% \newtheorem*{note}{Note pédagogique}

% \usepackage[vmargin=25mm,hmargin=15mm]{geometry}          % Marges peronnalisées

% Quelques règlage de mise en page
%\renewcommand{\baselinestretch}{1.2} % taille de l'interligne
\setlength{\parindent}{0pt}
\setlength{\parskip}{0.9ex plus 0.5ex minus 0.2ex}

%\usepackage{fancyhdr}        % Fancy page headers
%\lhead{}                     % Top-Left
%\chead{}                     % Top-Center
%\rhead{}                     % Top-Right
%\lfoot{}                     % Bottom-Left
%\cfoot{}                     % Bottom-Center
%\rfoot{}                     % Bottom-Right
%\pagestyle{fancy}

\usepackage{listings}         % Pour mettre en page du code source
\usepackage{color}            % Pour les lien en couleur dans le pdf
\definecolor{colBg}        {rgb}{1,1,0.8}
\definecolor{colKeys}      {rgb}{0,0,1}
\definecolor{colComments}  {rgb}{1,0,0}
\definecolor{colString}    {rgb}{0,0.5,0}
\definecolor{colBasic}     {rgb}{0,0,0}
\definecolor{colIdentifier}{rgb}{0,0,0}
\lstset{%                     % Basic style
  numbers=left,%
  stepnumber=10,%
  numberstyle=\scriptsize,%
%
  basicstyle=\ttfamily\normalsize\color{colBasic},%
  commentstyle=\normalsize\itshape\color{colComments},%
  identifierstyle=\color{colIdentifier},%
  keywordstyle=\bf\ttfamily\color{colKeys},%
  stringstyle=\color{colString},%
  backgroundcolor=\color{colBg},%
%
%  mathescape=true,%
  extendedchars=false,%
%  tabsize=4,%
  columns=flexible,%
  fontadjust=true,%
  frame=lines,%
  showspaces=false,%
  showstringspaces=false,%
%
  emptylines=1,%
  breaklines=true,%
  breakautoindent=true,%     % Inutile avec breaklines=false
  literate={é}{{\'e}}1 {è}{{\`e}}1 {ô}{{\^o}}1 {à}{{\`a}}1 {ç}{{\c{c}}}1 
}
\lstset{language=}        % ... En C++

\definecolor{darkgreen}{rgb}{0, 0.7, 0}
\definecolor{darkgreen2}{rgb}{0, 0.5, 0}
\definecolor{red2}{rgb}{0.8, 0, 0}
\lstdefinelanguage{diff} {
    morecomment=[f][\color{darkgreen}][0]{+},
    morecomment=[f][\color{red}][0]{-},
    morecomment=[f][\itshape\color{darkgreen2}][0]{+++},
    morecomment=[f][\itshape\color{red2}][0]{---},
    moredelim=[l][\color{cyan}]{\ \@\@},
    moredelim=*[l][\color{blue}]{\@\@},
}

\hypersetup{colorlinks=true,plainpages=false,urlcolor=blue,linkcolor=}






